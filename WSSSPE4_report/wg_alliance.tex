%%%%%%%%%%%%%%%%%%%%%%%%%%%%%%%%%%%%%%%%%%%%%%%%%%%%%%%%%%%%
\subsection{Software Sustainability Alliance}
\label{sec:alliance}
%%%%%%%%%%%%%%%%%%%%%%%%%%%%%%%%%%%%%%%%%%%%%%%%%%%%%%%%%%%%

\note{Neil and Jean wrote this}

The Software Sustainability Alliance working group aims to establish an alliance between organizations interested in improving the sustainability of research software. Such an alliance would ease the collaboration between member organizations to improve the sharing of expertise, resources and best practices. We are currently seeking feedback on potential member organizations, as well as the aims and scope of this alliance.

\subsubsection{Participants}

\begin{itemize}
\item Neil Chue Hong <N.ChueHong@software.ac.uk>
\item Jean Salac <jeansalac@virginia.edu>
\item Radovan Bast <radovan.bast@uit.no>
\item Lorraine Hwang <ljhwang@ucdavis.edu>
\item Karthik Ram <karthik.ram@berkeley.edu>
\item Peter Elmer <Peter.Elmer@cern.ch>
\end{itemize}


\subsubsection{Working group objective}

The overall objective of the Software Sustainability Alliance working group is to develop the steps necessary for establishing an alliance between organizations willing to engage in mutually reinforcing activities to advance the sustainability of software used in research. These organizations are funded groups or teams that aim to advance research software sustainability beyond their local university or community. More specifically, this working group aims to define the scope of this alliance and provide clear distinction with WSSSPE. We also seek to understand the incentives for members of this alliance and identify its key activities.

\subsubsection{Gap or challenge}

Currently, point-to-point collaboration exists between organizations, but this inadvertently results in competition or redundancy within the sustainable software community. An alliance of software sustainability organizations would ease inter-organization collaboration and the promotion of software sustainability. This alliance would also improve the pooling of competencies and the sharing of expertise. Furthermore, with the international scope of this alliance, it could support an organization who wants to hold an event in a country where another organization exists.

\subsubsection{Relevant people and resources}

Our working group is looking for ideas for which organizations should be consulted or invited to join the Software Sustainability Alliance. We are also looking for feedback on the aims and scope for joining the Software Sustainability Alliance.

\subsubsection{Plans}

Throughout the scoping period, our working group will continue to refine the Software Sustainability Alliance as we collect feedback on potential members, aims and scopes. We will also update our website to reflect these changes. Next steps and further tasks will be determined based on the feedback from the scoping period.

\subsubsection{SMART steps}

The first SMART steps of our working group are outlined below:
\begin{itemize}
\item Create a draft text to explain context and incentives for joining the Software Sustainability Alliance
\item Come up with a list of potential invitees
\item Update www.softwaresustainability.org to provide information
\item Draft a letter to go out to potential alliance members
\item Create a timeline and framework for initial consultation
\end{itemize}

\subsubsection{More information \& joining instructions}

If interested, visit \url{http://softwaresustainability.org/} or email Neil Chue Hong (N.ChueHong@software.ac.uk) and Jean Salac (jeansalac@virginia.edu)
