%%%%%%%%%%%%%%%%%%%%%%%%%%%%%%%%%%%%%%%%%%%%%%%%%%%%%%%%%%%%
\subsection{Software engineering processes tailored for research software}
\label{sec:soft-eng}
%%%%%%%%%%%%%%%%%%%%%%%%%%%%%%%%%%%%%%%%%%%%%%%%%%%%%%%%%%%%

%\note{Anshu to write this}

%Introduction to group here, including the overall objective of work in this area.
This working group concerned itself with identifying processes that are not adequately
covered by general software engineering. Verification and testing is
the first subtopic that we address.  
% there is a growing awareness about the lack of trust
% and reproducibility in many computational results. 
%\katznote{this paragraph repeats some of the previous one - they probably should be merged}
Computational science and engineering applications have many moving
parts that need to interoperate with one another. The accuracy and
reliability of results produced by the scientific software depends not
only on the individual components behaving correctly, but also on the
validity of their interactions. 
% Therefore, a rigorous verification
% process and a robust testing regime are critical requirements for
% scientific software.  There have been instances where inadequate
% verification has resulted in publication of wrong results and later
% retraction of publications.
As scientific understanding grows
the corresponding computational software models are refined, leading
to more complex codes. Increasing complexity makes them more prone to
defects, not only in individual code units, but also in interaction
among units. Therefore, a strong verification process combined with a
rigorous testing regime plays a critical role in the prevention of
generating incorrect scientific results. However, most science teams
struggle to find a good solution for themselves. Causes range from
lack of exposure to the practices, to distrust of adopting practices
because they do not meet the needs of the teams developing such
software.  The current focus of our effort is on testing because it is the first step
towards building a software processes that can lead to provenance and
reproducibility, the hallmarks of quality science. 

\subsubsection{Participants}
\begin{itemize}
\item Mark Abraham
\item Anshu Dubey 
\item Hans Fangohr
\item Dominic Kempf
\item Eric Seidel
\end{itemize}

\subsubsection{Working group objective}
Scientific computing software lags behind commercial software in
adoption of software engineering practices. This
gap is particularly acute in the area of software testing,
verification, and validation, where the standard practices are
simultaneously inadequate and overly onerous.
This group aims to close the gap.

\subsubsection{Gap or challenge}
%What is the gap or challenge being addressed?
Computational science code developers often lack of exposure to
regular testing and its benefits. Good developers will test their code to
verify that it operates as expected; however, they may not appreciate
that without regular testing, defects can be introduced
inadvertently. An even bigger challenge is that those who understand
the importance of regular testing do not often find much help from
software engineering literature. There is a significant gap between
the testing gospel and its applicability to computational science. This
gap leads to frustration and abandonment of the good with the bad. Some
relevant literature exists, in particular experiences from
practitioners in computational science who developed their own
solutions. However, this literature is scattered among many different
forums, and can be challenging to find. Our working group aims to
address this gap by curating the existing content and contributing
content where none exists.

\subsubsection{Relevant people and resources}

%What people, groups, or resources are needed.
The working group will benefit from wide participation by developers
of large computational science codes. The reason is that the
management of such codes becomes intractable without adopting some
software process and a testing regime. The experiences and
customizations vary, and the community will benefit from hearing about
as many as possible. The seed resources required are fairly
minimal. We have started a git repository for collecting the
existing references. That, and a few volunteers reading through the
references is all we need in the beginning. As we gather more
knowledge and pinpoint gaps, we may need more resources to reach out
to a wider group of developers to collect more information.

\subsubsection{Plans}

%What tasks will the working group undertake
This working group will (1) conduct a literature survey to gauge the extent
of awareness of the issue in general, (2) generate content useful for
the community where needed, and (3) curate the collected and added
content for the use of the community.


\subsubsection{SMART steps}

%What are the first SMART steps proposed?
Our first few SMART steps are : 
\begin{itemize}
\item Create a channel in wssspe.slack.com -- {\em done}
\item Create github repository
  {\url{http://github.com/wssspe/testing-in-science}} -- {\em done}
\item Find and gather existing publications in repository -- {\em ongoing}
\item Review and summarize material. Decide whether we consider this
  sufficient. If yes, then we will put brief report together and
  conclude the working group.
\item If we do not consider the material adequate, we start research
  and gather methodologies for testing in science 
\item Write a document accessible to computational science and
  software engineering community, and publish document at a citeable forum.
\end{itemize}

\subsubsection{More information \& joining instructions}

%How could a reader get more information or get more involved?
Readers interested in getting more information should get in touch
with a member of the working group. The working group
has a channel in the WSSSPE Slack. The channel is called
{\it wg-testing-in-science}.
(See \S\ref{sec:slack} for instructions on how to join the Slack WSSSPE team.)
Additionally, a git
repository (\url{https://github.com/WSSSPE/WG-Best-Practices.git}) exists for contributing content and reference to, and
curation of the existing literature on this topic.