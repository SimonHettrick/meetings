%%%%%%%%%%%%%%%%%%%%%%%%%%%%%%%%%%%%%%%%%%%%%%%%%%%%%%%%%%%%
\subsection{Software best practices for undergraduates}
\label{sec:best-practices-undergrads}
%%%%%%%%%%%%%%%%%%%%%%%%%%%%%%%%%%%%%%%%%%%%%%%%%%%%%%%%%%%%

\note{Jonah Miller to write this}

% Introduction to group here, including the overall objective of work
% in this area.

This working group was motivated by the perceived prevalence of
so-called ``dark code'' in scientific communities: code written by
individual researchers in an unsustainable way that is never shared
with the larger community. Participants recalled their own experience
working with colleagues who write dark code or inheriting
unsustainable dark code from collaborators.

The question is, then, how to prevent researchers from writing dark
code? The participants hypothesized that the best strategy is to catch
researchers while they are still in training and teach them software
best-practices. Therefore, this working group was formed with the goal
of developing courses on software best practices aimed at
undergraduate students studying domain science. The program might be
similar to a software carpentry or data carpentry workshop but with a
focus for domain scientists.

\subsubsection{Participants}

The participants in this working group are/were Jonah Miller,
Aleksandra Nenadic, Raniere Silva, Francisco Queiroz, Hans Fangohr,
and Prabhjyot Sing. Aleksandra Nenadic and Raniere Silva are involved
in software and data carpentry.

\subsubsection{Working group objective}

Implementation of a course aimed at each domain science is a long term
goal. However, a short-term, achievable goal, is to develop a
curriculum for a course aimed at a single domain science. Since the
participants in the working group have expertise in physics, this is a
natural target.

\subsubsection{Gap or challenge}

The challenge is that people are writing unsustainable scientific
software without ever learning that there is a better way. Here we
focus on training in good software design and engineering in this
context.

\subsubsection{Relevant people and resources}

The development of a successful curriculum relies on the expertise of
software engineers to describe the best practices, domain experts to
describe model problems and work-flow, and instructors to formulate
the pedagogy. Ideally these people will be brought together for a
short workshop or hackathon with the goal of drafting the
curriculum. Later, organizational partners will be required to
actually implement the program.

\subsubsection{Plans}

The working group will seek funding to organize a hackathon to develop
the curriculum and seek the required expertise to invite. The
hackathon will be implemented and a draft of the curriculum will be
written within the next year.

\subsubsection{SMART steps}

The working group will accomplish or has accomplished the following
smart steps:

\begin{enumerate}
\item Create channels of communication including a mailing list, a
  slack channel, and a git repository (done)
\item Review literature on best and good-enough practices for
  scientific computing as well as literature on pedagogy for
  scientific and research computing (by February 1, 2017)
\item Organize a telecon to organize a workshop to write the
  curriculum and decide how to bring in outside experts (by February 1,
  2017)
\item Write a draft of the curriculum (by April 1, 2017)
\item Invite the broader community to contribute to a more complete
  curriculum (by April 30, 2017)
\end{enumerate}

\subsubsection{More information \& joining instructions}

Anyone interested in contributing should get in touch via one of the
following methods:

\begin{itemize}
\item Join our google group/mailing list
  \cite{WSSSPEUndergradGoogleGroup}.
\item Join the WSSSPE Slack \cite{WSSSPESlack} and sign into the
  \texttt{\#wg-undergraduatecourse} channel and say hi.
\item Ask to join the WSSSPE4-undergraduate-course organization
  \cite{WSSSPEUndergradGithub}.
\end{itemize}