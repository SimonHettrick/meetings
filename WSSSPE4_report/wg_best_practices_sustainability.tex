%%%%%%%%%%%%%%%%%%%%%%%%%%%%%%%%%%%%%%%%%%%%%%%%%%%%%%%%%%%%
\subsection{Verifying best practices \& metrics for sustainable research software}
\label{sec:best-practices-sustainable}
%%%%%%%%%%%%%%%%%%%%%%%%%%%%%%%%%%%%%%%%%%%%%%%%%%%%%%%%%%%%

\note{Ray to write this}

Many open source projects for research software document their best practices that contributed to the sustainability of the software.  Many open source projects also document software metrics they use to define their research software project’s success.  This project endeavors to aggregate several sources of these best practices into a consolidated list of best practices, as and to also do the same for software metrics in another consolidated list.  Our team will then create a workflow to evaluate how open source projects stand up to these lists.

\subsubsection{Participants}

\begin{itemize}
\item Ray Idaszak <rayi@renci.org>
\item August Muench <august.muench@aas.org>
\item Jonah Miller <jmiller@perimeterinstitute.ca>
\item Lorraine Hwang <ljhwang@ucdavis.edu>
\end{itemize}

\subsubsection{Working group objective}

The overall objective of the Verify Best Practices and Metrics for Sustainable Software working group is to take the outputs of the WSSSPE efforts to identify best practices for creation of sustainable software for science/academia, and also the outputs of WSSSPE efforts to identify metrics for sustainable software for science/academia and cross-reference these with current open source research software that successfully uses modern software engineering.  This will allow us to identify gaps on both sides.  This approach will also allow the group to  hypothesize how successful open source projects can be further improved, verify that recommended approaches for software engineering for science/academia are sufficient and valid, and that metrics for software engineering for science/academia are relevant and useful.  The group has a specific objective of getting a good cross-sampling of disparate software including community model codes, community cyberinfrastructure, community analytic tools, etc.

\subsubsection{Gap or challenge}

The gap addressed by the project is described by the following:
\begin{itemize}
\item Are the suggested software engineering for science/academia best practices verified when evaluated against open source research software projects that currently use modern software engineering?
\item Are the suggested metrics for science/academia verified to be relevant when evaluated against open source research software projects that currently use modern software engineering?
\item Do research groups have role models to follow for successful best practices in research software?
\end{itemize}

\subsubsection{Relevant people and resources}

The people spearheading this effort at the time of the writing of this workshop report are Ray Idaszak <rayi@renci.org>, August Muench <august.muench@aas.org>, Jonah Miller <jmiller@perimeterinstitute.ca>, Lorraine Hwang <ljhwang@ucdavis.edu>, Peter Elmer <peter.elmer@cern.ch>, and Hans Fangohr <fangohr@soton.ac.uk>. 

\noindent
Resources include research projects suggested from WSSSPE community including:
\begin{itemize}
\item FLASH -- \url{http://flash.uchicago.edu/site/flashcode/}
\item GROMACS -- http://www.gromacs.org/
\item CIG -- \url{http://geodynamics.org/}
\item CUAHSI -- \url{https://www.cuahsi.org/}
\item CSDMS -- \url{https://csdms.colorado.edu/wiki/Main_Page}
\item OntoSoft -- \url{http://www.ontosoft.org/}
\item Anything -- from ASCL (astronomy) \href{https://ui.adsabs.harvard.edu/#search/q=bibstem%3A%22ASCL%22&sort=citation_count%20desc%2C%20bibcode%20desc}{ ASCL entries sorted by citation count}
\item Yt-project -- (astronomy) \url{http://yt-project.org}
\item Einstein Toolkit -- (physics) \url{https://einsteintoolkit.org/}
\item SpEC -- (physics) \url{https://www.black-holes.org/SpEC.html}
\item pyCLOUDY -- (astrophysics) \url{http://ascl.net/1304.020}
\end{itemize}

\noindent
Additional resources include these sources for best practices to be explored: 

\begin{itemize}
\item Good Enough Practices in Scientific Computing, Wilson et al. 2016

\begin{itemize}
\item \url{https://swcarpentry.github.io/good-enough-practices-in-scientific-computing/}
\item \url{http://arxiv.org/abs/1609.00037}
\end{itemize}

\item Best Practices for Scientific Computing, Wilson et al. 2014
\begin{itemize}
\item \url{http://dx.doi.org/10.1371/journal.pbio.1001745}
\item 8 high level categories 
\end{itemize}

\item Butterfly: a paradigm towards stable bio & neuro informatics tools development
\begin{itemize}
\item \url{http://www.frontiersin.org/10.3389/conf.fnins.2015.91.00009/event_abstract}
\end{itemize}

\item Working towards Sustainable Software for Science: Practice and Experiences
\begin{itemize}
\item JORS collection
\item \url{http://openresearchsoftware.metajnl.com/collections/special/working-towards-sustainable-software-for-science/}
\end{itemize}

\item CIG Software Development Best Practices
\begin{itemize}
\item \url{https://geodynamics.org/cig/dev/best-practices/}
\item 6 categories; 3 Levels of detail within each: minimum, standard, target
\end{itemize}

\item Best Practices for Scientific Computing
\begin{itemize}
\item \url{http://arxiv.org/abs/1210.0530v1}
\end{itemize}

\item Software Sustainability Institute
\begin{itemize}
\item \url{https://www.software.ac.uk}
\end{itemize}

\item WSSSPE papers from past WSSSPE workshops
\begin{itemize}
\item \url{http://wssspe.researchcomputing.org.uk/}
\end{itemize}

\item JOSS Peer Review & rOpenSci onboarding
\begin{itemize} 
\item \url{http://joss.theoj.org/about#reviewer_guidelines} 
\item E.g., \url{https://github.com/openjournals/joss-reviews/issues/43}
\item \url{https://github.com/ropensci/onboarding}
\item \url{http://ropensci.org/blog/2016/03/28/software-review}
\end{itemize}

\item RSE 2016 Talk:
\begin{itemize}
\item \href{http://www.rse.ac.uk/conf2016_talks#T1.1}{InterMine: Best Practices for Open Source Software}
\end{itemize}

\item ANNIS corpus linguistic analysis tool
\begin{itemize}
\item \url{http://corpus-tools.org/annis/} (humanities code)
\end{itemize}

\end{itemize}


\subsubsection{Plans}

Our plans are for each of the Verify Best Practices and Metrics for Sustainable Research Software to volunteer to take on one to two projects each, and as incentive the projects that individual take on can be one they are interested in.  This would lead to an estimated 5-10 projects to be evaluated within this project.  In terms of increasing this sample size, it is our hope that once we document the workflow by which these evaluations are performed, others can be self-sufficient in following this workflow and contribute their own evaluations of software they wish to evaluate and have interest in.

\subsubsection{SMART steps}

\begin{itemize}
\item Create mailing list for our project titled:  Verify Best Practices and Metrics for Sustainable Research Software.
\item Identify where to obtain representative recommended software engineering best practices and metrics within the WSSSPE4 community including single points-of-contact.  Obtain said best practices and metrics in raw form.
\item Create GitHub site for project.
\item Identify where to obtain representative successful open source projects in some manageable number of domains that also successfully use modern software engineering.  The minimum number of projects per domain should be 1. Obtain said list of representative projects.

\begin{itemize}
\item Example sources: Astronomy source code library; high citation software papers; CIG software list; personal domain knowledge.
\end{itemize}

\item Create a case study sheet made of questions to be posed about existing research groups.
\item Perform review.
\item Aggregate results and document.  
\item Document workflow of getting to these results so it can be repeated for additional domains.
\end{itemize}


\subsubsection{More information \& joining instructions}

To join this group or obtain more information about it, please send an email to: \href{wssspe4-verify-best-practices@googlegroups.com}{Verifying best practices \& metrics for sustainable research software.}
