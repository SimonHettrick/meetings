%%%%%%%%%%%%%%%%%%%%%%%%%%%%%%%%%%%%%%%%%%%%%%%%%%%%%%%%%%%%
\subsection{Social science for scientific software}
\label{sec:social}
%%%%%%%%%%%%%%%%%%%%%%%%%%%%%%%%%%%%%%%%%%%%%%%%%%%%%%%%%%%%

\note{from Stuart}

%Introduction to group here, including the overall objective of work in this area.

This working group was motivated by the goal of building better connections between academic researchers who are studying topics in or relating to software sustainability with practitioners, managers, and administrators who are working in the area of software sustainability. In any domain, bringing research and practice closer together a mutually beneficial goal, but also has its challenges. This group met to discuss existing research projects and findings that might be relevant for RSEs and others in the software sustainability domain, then identified several gaps and challenges to tackle in bringing research and practice together.

\subsubsection{Participants}

\begin{itemize}
  \item Stuart Geiger <stuart@stuartgeiger.com>
  \item Lorraine Hwang <ljhwang@ucdavis.edu>
  \item Robert McDonald <rhmcdona@indiana.edu>
\end{itemize}

\subsubsection{Working group objective}

To give social scientists and practitioners working on scientific software, software sustainability, and open source and a better understanding of each other's work, as well as help them connect and coordinate on specific projects of mutual interest.

\subsubsection{Gap or challenge}

There is more and more academic research being done on topics related to software sustainability, including work on software engineering practices and management of open source projects. However, academic research in general is often siloed for many reasons, and work on topics relevant to software sustainability is no exception. There are many academic studies and projects which may be relevant for practitioners working in this area, and many projects and initiatives by practitioners that may be relevant for social scientists. However, there is a gap between these two research and practice. Furthermore, we also recognize that academic social science and research software engineering are not monolithic, and there is a need to connect people who care about research-driven best practices inside of these two domains with each other as well.


\subsubsection{Relevant people and resources}

People: a network of invested social scientists and research software engineers who care about research on software engineering practices
Resources: a mailing list and a central repository for collecting literature and documenting lessons learned

\subsubsection{Plans}

%What tasks will the working group undertake
The working group planned to briefly survey existing literature to get a better sense of the academic research landscape, facilitate some initial dialog between academic researchers and practitioners at WSSSPE4, then identify needed actions that would be mutually beneficial to researchers and practitioners. This plan resulted in the creation of the following SMART steps listed in the next section.

\subsubsection{SMART steps}

\begin{itemize}
\item Collect research questions RSEs have for social scientists (done)
\item Create a mailing list for people interested in this topic (done) \url{https://groups.google.com/forum/#!forum/researchsoftwarestudies}
\item Create a central repository for collecting literature and documenting lessons learned
\item Publicize mailing list and collect (more) research questions/topics/literature/researchers
\item Write up synthesis document 
\end{itemize}

\subsubsection{More information \& joining instructions}

Join the mailing list at \url{https://groups.google.com/forum/#!forum/researchsoftwarestudies}
