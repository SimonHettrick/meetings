\documentclass[11pt, oneside]{amsart}
\pdfoutput=1

\usepackage{amsmath,amssymb,amsmath}

\usepackage{color}
\usepackage{dcolumn}
\usepackage{float,caption}
\usepackage{graphicx}

\usepackage[T1]{fontenc}
\usepackage[utf8]{inputenc}
\usepackage{lmodern}

\usepackage[activate={true,nocompatibility},final,tracking=true,kerning=true,spacing=true,factor=1100,stretch=10,shrink=10]{microtype}
\microtypecontext{spacing=nonfrench}
\usepackage{xspace}

\usepackage{rotating}
\usepackage{subfigure}
\usepackage{psfrag}
\usepackage{tabularx}
\usepackage[hyphens]{url}
\usepackage{wrapfig}
\usepackage{longtable}
\usepackage{verbatim}
\usepackage{booktabs,multicol,multirow}

%better printing of numbers
\usepackage[english]{babel}
\usepackage{textcomp}
\usepackage{csquotes}

% The following three lines are used for displaying footnote in tables.
\usepackage{footnote}
\makesavenoteenv{tabular}
\makesavenoteenv{table}


\usepackage{enumitem}
\setlist{leftmargin=7mm}

%\setcounter{secnumdepth}{3}
%\setcounter{tocdepth}{3}


\usepackage[bookmarks, bookmarksopen, bookmarksnumbered]{hyperref}
\usepackage[all]{hypcap}
\urlstyle{rm}

\definecolor{orange}{rgb}{1.0,0.3,0.0}
\definecolor{violet}{rgb}{0.75,0,1}
\definecolor{darkgreen}{rgb}{0,0.6,0}
\definecolor{cyan}{rgb}{0.2,0.7,0.7}
\definecolor{blueish}{rgb}{0.2,0.2,0.8}
\definecolor{darkblue}{rgb}{0.1,0.1,0.9}

\newcommand{\todo}[1]{{\color{blue}$\blacksquare$~\textsf{[TODO: #1]}}}
\newcommand{\note}[1]{ {\textcolor{blueish}    { ***Note:      #1 }}}
\newcommand{\katznote}[1]{ {\textcolor{magenta}    { ***Dan:      #1 }}}
\newcommand{\gabnote}[1]{ {\textcolor{cyan}    { ***Gabrielle:     #1 }}}
\newcommand{\nchnote}[1]{  {\textcolor{orange}      { ***Neil: #1 }}}

% Don't use tt font for urls
\urlstyle{rm}

\usepackage[letterpaper, margin=1in]{geometry}
% You can use a baselinestretch of down to 0.9
\renewcommand{\baselinestretch}{0.96}

\sloppypar

\begin{document}

\title[]{Report on the Fourth Workshop on Sustainable Software for Science: Practice and Experiences (WSSSPE4)}

\author{Daniel S.\ Katz$^{(1)}$,
Kyle E.\ Niemeyer$^{(2)}$,
Sandra Gesing$^{(3)}$,
Lorraine Hwang$^{(4)}$,
Author order to be determined based on contributions
}

%
\thanks{{}$^{(1)}$ \hspace{-1ex}National Center for Supercomputing Applications (NCSA) \&
Electrical and Computer Engineering (ECE) Department \&
School of Information Sciences (iSchool),
University of Illinois at Urbana--Champaign, IL, USA; d.katz@ieee.org}
%
\thanks{{}$^{(2)}$ School of Mechanical, Industrial, and Manufacturing Engineering,
Oregon State University, OR, USA; kyle.niemeyer@oregonstate.edu}
%
\thanks{{}$^{(3)}$ \note{info needed}}
%
\thanks{{}$^{(4)}$ \note{info needed}}


\begin{abstract}
This report records and discusses the Fourth Workshop on Sustainable
Software for Science: Practice and Experiences (WSSSPE4). The report
includes a description of the keynote presentation of the workshop,
which % TODO
% It also summarizes a set of lightning
% talks in which speakers highlighted to-the-point lessons and challenges
% pertaining to sustaining scientific software.
% The final and main contribution of the report is a summary of the
% discussions, future steps, and future organization for a set of self-organized
% working groups on topics including developing pathways to funding scientific
% software; constructing useful common metrics for crediting software
% stakeholders; identifying principles for sustainable software engineering
% design; reaching out to research software organizations around the world; and
% building communities for software sustainability. For each group, we include a
% point of contact and a landing page that can be used by those who want to join
% that group's future activities. The main challenge left by the workshop is to
% see if the groups will execute these activities that they have scheduled, and
% how the WSSSPE community can encourage this to happen.

\end{abstract}



\maketitle
\newpage

%%%%%%%%%%%%%%%%%%%%%%%%%%%%%%%%%%%%%%%%%%%%%%%%%%%%%%%%%%%%
\section{Introduction} \label{sec:intro}
%%%%%%%%%%%%%%%%%%%%%%%%%%%%%%%%%%%%%%%%%%%%%%%%%%%%%%%%%%%%

The Fourth Workshop on Sustainable Software for Science: Practice and Experiences
(WSSSPE4)\footnote{\url{http://wssspe.researchcomputing.org.uk/wssspe4/}} was
held over 2 1/2 days on 12--14 September 2016 in Manchester, England, UK.
This location and date was selected so that WSSSPE4 immediately preceded the
First Research Software Engineers (RSE) Conference.
Previous events in the WSSSPE series include
WSSSPE1\footnote{\url{http://wssspe.researchcomputing.org.uk/wssspe1/}}~\cite{WSSSPE1-pre-report,WSSSPE1},
held in conjunction with SC13;
WSSSPE1.1\footnote{\url{http://wssspe.researchcomputing.org.uk/wssspe1-1/}}, a
focused workshop organized jointly with the SciPy
conference\footnote{\url{https://conference.scipy.org/scipy2014/participate/wssspe/}};
WSSSPE2\footnote{\url{http://wssspe.researchcomputing.org.uk/wssspe2/}}~\cite{WSSSPE2-pre-report,WSSSPE2},
held in conjunction with SC14;
WSSSPE2.1\footnote{\url{http://wssspe.researchcomputing.org.uk/wssspe2-1/}}, a
focused workshop organized again jointly with
SciPy\footnote{\url{http://scipy2015.scipy.org/ehome/115969/286469/}};
and WSSSPE3\footnote{\url{http://wssspe.researchcomputing.org.uk/wssspe3/}}~\cite{WSSSPE3},
held in Boulder, Colorado, USA.
Note that the first WSSSPE workshop was named
``Working towards
Sustainable Software for Science: Practice and Experiences,'' which remains the meaning
of the WSSSPE group, but the workshops after that were named
``Workshop on Sustainable
Software for Science: Practice and Experiences.'' Together these reflect
that WSSSPE is both a community and a set of workshops.

\todo{Introduction}

The WSSSPE4 workshop included multiple mechanisms for participation and
encouraged team building around solutions. WSSSPE4 strongly encouraged participation
of early-career scientists, postdoctoral researchers, graduate students,
early-stage researchers, and those from underrepresented groups,
with funds provided to the conference organizers by the National Science
Foundation (NSF), the Gordon and Betty Moore Foundation, the Alfred P.~Sloan Foundation, and the Software
Sustainability Institute (SSI) to support the travel of potential participants
who would not otherwise be able to attend the workshop. These
funds allowed 29 additional people to attend and participate. A subset of the
organizing committee reviewed 44 applications for travel support and
competitively selected the awardees, including 11 students and five early-career
researchers. In addition, the travel award subcommittee tried to increase the
diversity of applicants by forwarding the notice of travel support to
organizations including PyLadies, Django Girls, Women Who Code, and
Women in HPC.\todo{add links}
\todo{did we reach out to any others?}

WSSSPE4 also included two professional event organizers/facilitators from
KnowInnovation who helped the organizing committee plan the workshop agenda,
and during the workshop, they actively engaged participants with various
tools, activities, and reminders.

This workshop also introduced a Code of Conduct (CoC)
\footnote{\url{http://wssspe.researchcomputing.org.uk/wssspe4/code-of-conduct/}}
for both WSSSPE4 and interactions in the greater WSSSPE community (e.g., on
Twitter, in email lists, in the Slack team). The WSSSPE4 CoC is based on the
FORCE11 conference CoC~\cite{FORCE11:CoC}, in turn based on the Code4Lib
CoC~\cite{Code4Lib:CoC}.
The main guidelines of the CoC are:
\begin{quote}
    WSSSPE events are community events intended for networking and collaboration
    as well as learning. We value the participation of every member of the
    community and want all attendees to have an enjoyable and fulfilling
    experience. Accordingly, all attendees are expected to show respect and
    courtesy to other attendees throughout the event and in interactions online
    associated with the event.

    The WSSSPE event organizers are dedicated to providing a harassment-free
    experience for everyone, regardless of gender, gender identity and
    expression, age, sexual orientation, disability, physical appearance,
    body size, race, ethnicity, religion (or lack thereof), technology choices,
    or other group status.

    To make clear what is expected, everyone taking part in WSSSPE events and
    discussions---speakers, helpers, organizers, and participants---is required
    to conform to the following Code of Conduct.

    \begin{itemize}
    \item Communicate appropriately for a professional audience including
    people of many different backgrounds. Sexual language and imagery are not
    appropriate for any event.

    \item Be kind to others. Do not insult or put down other attendees. Be
    mindful of jargon, which can sometimes exclude others from engaging in the
    discussion.

    \item Behave professionally. Remember that harassment and sexist, racist,
    ageist, or exclusionary behavior are not appropriate.
    \end{itemize}
\end{quote}

The CoC was introduced at the beginning of WSSSPE4, with the CoC subcommittee
and a general email address introduced for reporting concerns or incidents, or
asking questions.  There was one concern mentioned to the CoC subcommittee
after the first half day,
regarding how part of the workshop was being run, and we changed the workshop
to address this.

This report is based on the events at the workshop and the submitted materials.
There events included discussion of the WSSSPE mission and vision, a keynote,
a set of presentations, a panel, and a number of working groups.  About a day of the
workshop was spent with participants in the working groups, which occurred in parallel
with each other.

%%%%%%%%%%%%%%%%%%%%%%%%%%%%%%%%%%%%%%%%%%%%%%%%%%%%%%%%%%%%
\section{Calls for participation} \label{sec:preworkshop}
%%%%%%%%%%%%%%%%%%%%%%%%%%%%%%%%%%%%%%%%%%%%%%%%%%%%%%%%%%%%

WSSSPE4 was based on the work done in WSSSPE1, WSSSPE2, and WSSSPE3, but aimed
at producing working groups that better continued working after the workshop ended.
In addition, based on feedback after WSSSPE3, it became clear that WSSSPE attendees
had two different motivations in participating.  One motivation was to make a better future
for research software, and the other was to immediately do better research software development.
This led to the idea of WSSSPE4 being partially divided into two tracks:

\begin{quote}
    \textbf{Track 1 -- Building a sustainable future for open-use research
    software} has the goals of defining a vision of the future of open-use
    research software, and in the workshop, initiating the activities that are
    needed to get there. The idea of this track is to first think about where
    we want to be 5 to 10 years from now, without being too concerned with
    where we are today, and then to determine how we can move to this future.

    \textbf{Track 2 -- Practices \& experiences in sustainable scientific software}
    has the goal of improving the quality of today's research software and the
    experiences of its developers by sharing practices and experiences.
    This track is focused on the current state of scientific software and what
    we can do to improve it in the short term, starting with where we are today.
\end{quote}

The call for participation requested
idea papers, which present implementable proposals related to Track 1;
position papers, which are longer, not previously published papers related to
Track 2 specifically discussing what we can do to improve sustainable scientific
software in the short term, starting with where we are today;
experience papers, longer papers related to Track 2 that discuss current
practices and experiences and how they have been used to improve the quality of
today's research software and/or the experiences of its developers;
demos, brief papers describing a tool or
process that would be demonstrated, relevant to Track 2 that improves the quality of today's research
software and\slash or the experiences of its developers; and
extended abstracts describing lightning talks,
where each author could make a brief statement about work that either had been
done or was needed.

There were~24 lightning talks submitted; after a peer-review process, 16
were accepted, as discussed further in Section~\ref{sec:lightning}.
\todo{update, add paper stats too}

%
%\begin{quote}
\begin{itemize}
\renewcommand{\labelenumi}{\textbf{\theenumi}.}
\setlength{\rightmargin}{1em}

\item Development and Community
\begin{itemize}
    \item Best practices for developing sustainable software
    \item Models for funding specialist expertise in software collaborations
    \item Software tools that aid sustainability
    \item Academia/industry interaction
    \item Refactoring/improving legacy scientific software
    \item Engineering design for sustainable software
    \item Metrics for the success of scientific software
    \item Adaptation of mainstream software practices for scientific software
\end{itemize}

\item Professionalization
\begin{itemize}
    \item Career paths
    \item RSE as a brand
    \item RSE outside of the UK or Europe
    \item Increase incentives in publishing, funding and promotion for better software
\end{itemize}

\item Training
\begin{itemize}
    \item Training for developing sustainable software
    \item Curriculum for software sustainability
\end{itemize}

\item Credit
\begin{itemize}
    \item Making the existing credit and citation ecosystem work better for software
    \item Future credit and citation ecosystem
    \item Software contributions as a part of tenure review
    \item Case studies of receiving credit for software contributions
    \item Awards and recognition that encourage sustainable software
\end{itemize}

\item Software publishing
\begin{itemize}
    \item Journals and alternative venues for publishing software
    \item Review processes for published software
\end{itemize}

\item Software discoverability/reuse
\begin{itemize}
    \item Proposals and case studies
\end{itemize}

\item Reproducibility and testing
\begin{itemize}
    \item Reproducibility in conferences and journals
    \item Best practices for code testing and code review
\end{itemize}

\end{itemize}

\todo{need to mention/link to the proceedings somewhere, though maybe not here.}

%%%%%%%%%%%%%%%%%%%%%%%%%%%%%%%%%%%%%%%%%%%%%%%%%%%%%%%%%%%%
\section{Keynote}\label{sec:keynote}
%%%%%%%%%%%%%%%%%%%%%%%%%%%%%%%%%%%%%%%%%%%%%%%%%%%%%%%%%%%%

\note{need a person to write this}

The keynote was given by Patricia Lago and entitled ``\todo{title}''.

%%%%%%%%%%%%%%%%%%%%%%%%%%%%%%%%%%%%%%%%%%%%%%%%%%%%%%%%%%%%
\section{Mission and vision}\label{sec:mission}
%%%%%%%%%%%%%%%%%%%%%%%%%%%%%%%%%%%%%%%%%%%%%%%%%%%%%%%%%%%%

\note{Lorraine and Dan will write this}

\todo{write info about process}

After this process, the final statements about WSSSPE are:

{\bf Mission}
WSSSPE is an international community-driven organization that promotes sustainable research software by addressing challenges related to the full lifecycle of research software through shared learning and community action.

{\bf Vision}
We envision a world where research software is accessible, robust, sustained, and recognized as a scholarly research product critical to the advancement of knowledge, learning, and discovery.

{\bf Focus areas}
WSSSPE promotes sustainable research software by positively impacting:
\begin{itemize}
\item {\bf Principles and Best Practices}. Promoting best practices in sustainable software
\item {\bf Careers}. Developing and supporting career paths in research software development and engineering
\item {\bf Learning}. Engaging in activities to promote peer learning and interaction
\item {\bf Credit}. Ensuring recognition of research software as an intellectual contribution equal to other research products
\end{itemize}

{\bf Definitions}

Sustainable software has the capacity to endure such that it will continue to be available in the future, on new platforms, meeting new needs.

The research software lifecycle includes:
\begin{itemize}
\item acquiring and assembling resources (including funding and people) into teams and communities
\item developing software
\item using software
\item recognizing contributions to and of software
\item maintaining software
\end{itemize}


%%%%%%%%%%%%%%%%%%%%%%%%%%%%%%%%%%%%%%%%%%%%%%%%%%%%%%%%%%%%
\section{Papers and demos} \label{sec:papers}
%%%%%%%%%%%%%%%%%%%%%%%%%%%%%%%%%%%%%%%%%%%%%%%%%%%%%%%%%%%%

\note{Sandra Gesing will write this}

Intro and description \katznote{not sure enum list is right here}
%
\begin{enumerate}
\item \textbf{Authors: \emph{title}}.
Description

\end{enumerate}


%%%%%%%%%%%%%%%%%%%%%%%%%%%%%%%%%%%%%%%%%%%%%%%%%%%%%%%%%%%%
\section{Lightning talks} \label{sec:lightning}
%%%%%%%%%%%%%%%%%%%%%%%%%%%%%%%%%%%%%%%%%%%%%%%%%%%%%%%%%%%%
\begin{comment}
\note{
\href{http://wssspe.researchcomputing.org.uk/wssspe4/agenda/}{Slides.}}
\end{comment}

\note{Kyle Niemeyer will write this}

Intro and description \katznote{not sure enum list is right here}
%
\begin{enumerate}
\item \textbf{Authors: \emph{title}}.
Description

\end{enumerate}

%%%%%%%%%%%%%%%%%%%%%%%%%%%%%%%%%%%%%%%%%%%%%%%%%%%%%%%%%%%%
\section{Panel discussion} \label{sec:panel}
%%%%%%%%%%%%%%%%%%%%%%%%%%%%%%%%%%%%%%%%%%%%%%%%%%%%%%%%%%%%

\todo{description of panel discussion on best practices} \note{asked Simon to write}

%%%%%%%%%%%%%%%%%%%%%%%%%%%%%%%%%%%%%%%%%%%%%%%%%%%%%%%%%%%%
\section{Working groups} \label{sec:WGs}
%%%%%%%%%%%%%%%%%%%%%%%%%%%%%%%%%%%%%%%%%%%%%%%%%%%%%%%%%%%%

\todo{describe process for assembling people into working groups: vision, gaps, projects exercise, then soapboxes}

Sections: \todo{copy wg\_open\_research\_index for each, add input command for each new file}

\begin{itemize}
\item Verifying best practices \& metrics for sustainable research software [Ray Idaszak]
\item Software Sustainability Alliance [Neil \& Jean]
\item Scientific Software Prototyping Infrastructure (S2PI) [Santiago Nunez-Corrales] 
\item CodeMeta [Alice Allen]
\item White paper on developing sustainable software [Sandra Gesing]
\item Social Science of Scientific Software [Stuart Geiger]
\item Software Best Practices for Undergraduates [Jonah Miller]
\item Meaningful metrics for sustainable software [Emily Chen]
\item Coordinating access to CI for research software [Mark Abraham]
\item Software engineering processes tailed for research software [Anshu Dubey]
\item Open research index [Dan]
\item Letters of evaluation for computational scientists [Wolfgang Bangerth]
\end{itemize}


%%%%%%%%%%%%%%%%%%%%%%%%%%%%%%%%%%%%%%%%%%%%%%%%%%%%%%%%%%%%
\subsection{Verifying best practices \& metrics for sustainable research software}
\label{sec:best-practices-sustainable}
%%%%%%%%%%%%%%%%%%%%%%%%%%%%%%%%%%%%%%%%%%%%%%%%%%%%%%%%%%%%

\note{Ray to write this}

Many open source projects for research software document their best practices that contributed to the sustainability of the software.  Many open source projects also document software metrics they use to define their research software project’s success.  This project endeavors to aggregate several sources of these best practices into a consolidated list of best practices, as and to also do the same for software metrics in another consolidated list.  Our team will then create a workflow to evaluate how open source projects stand up to these lists.

\subsubsection{Participants}

\begin{itemize}
\item Ray Idaszak <rayi@renci.org>
\item August Muench <august.muench@aas.org>
\item Jonah Miller <jmiller@perimeterinstitute.ca>
\item Lorraine Hwang <ljhwang@ucdavis.edu>
\end{itemize}

\subsubsection{Working group objective}

The overall objective of the Verify Best Practices and Metrics for Sustainable Software working group is to take the outputs of the WSSSPE efforts to identify best practices for creation of sustainable software for science/academia, and also the outputs of WSSSPE efforts to identify metrics for sustainable software for science/academia and cross-reference these with current open source research software that successfully uses modern software engineering.  This will allow us to identify gaps on both sides.  This approach will also allow the group to  hypothesize how successful open source projects can be further improved, verify that recommended approaches for software engineering for science/academia are sufficient and valid, and that metrics for software engineering for science/academia are relevant and useful.  The group has a specific objective of getting a good cross-sampling of disparate software including community model codes, community cyberinfrastructure, community analytic tools, etc.

\subsubsection{Gap or challenge}

The gap addressed by the project is described by the following:
\begin{itemize}
\item Are the suggested software engineering for science/academia best practices verified when evaluated against open source research software projects that currently use modern software engineering?
\item Are the suggested metrics for science/academia verified to be relevant when evaluated against open source research software projects that currently use modern software engineering?
\item Do research groups have role models to follow for successful best practices in research software?
\end{itemize}

\subsubsection{Relevant people and resources}

The people spearheading this effort at the time of the writing of this workshop report are Ray Idaszak <rayi@renci.org>, August Muench <august.muench@aas.org>, Jonah Miller <jmiller@perimeterinstitute.ca>, Lorraine Hwang <ljhwang@ucdavis.edu>, Peter Elmer <peter.elmer@cern.ch>, and Hans Fangohr <fangohr@soton.ac.uk>. 

\noindent
Resources include research projects suggested from WSSSPE community including:
\begin{itemize}
\item FLASH -- \url{http://flash.uchicago.edu/site/flashcode/}
\item GROMACS -- http://www.gromacs.org/
\item CIG -- \url{http://geodynamics.org/}
\item CUAHSI -- \url{https://www.cuahsi.org/}
\item CSDMS -- \url{https://csdms.colorado.edu/wiki/Main_Page}
\item OntoSoft -- \url{http://www.ontosoft.org/}
\item Anything -- from ASCL (astronomy) \href{https://ui.adsabs.harvard.edu/#search/q=bibstem%3A%22ASCL%22&sort=citation_count%20desc%2C%20bibcode%20desc}{ ASCL entries sorted by citation count}
\item Yt-project -- (astronomy) \url{http://yt-project.org}
\item Einstein Toolkit -- (physics) \url{https://einsteintoolkit.org/}
\item SpEC -- (physics) \url{https://www.black-holes.org/SpEC.html}
\item pyCLOUDY -- (astrophysics) \url{http://ascl.net/1304.020}
\end{itemize}

\noindent
Additional resources include these sources for best practices to be explored: 

\begin{itemize}
\item Good Enough Practices in Scientific Computing, Wilson et al. 2016

\begin{itemize}
\item \url{https://swcarpentry.github.io/good-enough-practices-in-scientific-computing/}
\item \url{http://arxiv.org/abs/1609.00037}
\end{itemize}

\item Best Practices for Scientific Computing, Wilson et al. 2014
\begin{itemize}
\item \url{http://dx.doi.org/10.1371/journal.pbio.1001745}
\item 8 high level categories 
\end{itemize}

\item Butterfly: a paradigm towards stable bio & neuro informatics tools development
\begin{itemize}
\item \url{http://www.frontiersin.org/10.3389/conf.fnins.2015.91.00009/event_abstract}
\end{itemize}

\item Working towards Sustainable Software for Science: Practice and Experiences
\begin{itemize}
\item JORS collection
\item \url{http://openresearchsoftware.metajnl.com/collections/special/working-towards-sustainable-software-for-science/}
\end{itemize}

\item CIG Software Development Best Practices
\begin{itemize}
\item \url{https://geodynamics.org/cig/dev/best-practices/}
\item 6 categories; 3 Levels of detail within each: minimum, standard, target
\end{itemize}

\item Best Practices for Scientific Computing
\begin{itemize}
\item \url{http://arxiv.org/abs/1210.0530v1}
\end{itemize}

\item Software Sustainability Institute
\begin{itemize}
\item \url{https://www.software.ac.uk}
\end{itemize}

\item WSSSPE papers from past WSSSPE workshops
\begin{itemize}
\item \url{http://wssspe.researchcomputing.org.uk/}
\end{itemize}

\item JOSS Peer Review & rOpenSci onboarding
\begin{itemize} 
\item \url{http://joss.theoj.org/about#reviewer_guidelines} 
\item E.g., \url{https://github.com/openjournals/joss-reviews/issues/43}
\item \url{https://github.com/ropensci/onboarding}
\item \url{http://ropensci.org/blog/2016/03/28/software-review}
\end{itemize}

\item RSE 2016 Talk:
\begin{itemize}
\item \href{http://www.rse.ac.uk/conf2016_talks#T1.1}{InterMine: Best Practices for Open Source Software}
\end{itemize}

\item ANNIS corpus linguistic analysis tool
\begin{itemize}
\item \url{http://corpus-tools.org/annis/} (humanities code)
\end{itemize}

\end{itemize}


\subsubsection{Plans}

Our plans are for each of the Verify Best Practices and Metrics for Sustainable Research Software to volunteer to take on one to two projects each, and as incentive the projects that individual take on can be one they are interested in.  This would lead to an estimated 5-10 projects to be evaluated within this project.  In terms of increasing this sample size, it is our hope that once we document the workflow by which these evaluations are performed, others can be self-sufficient in following this workflow and contribute their own evaluations of software they wish to evaluate and have interest in.

\subsubsection{SMART steps}

\begin{itemize}
\item Create mailing list for our project titled:  Verify Best Practices and Metrics for Sustainable Research Software.
\item Identify where to obtain representative recommended software engineering best practices and metrics within the WSSSPE4 community including single points-of-contact.  Obtain said best practices and metrics in raw form.
\item Create GitHub site for project.
\item Identify where to obtain representative successful open source projects in some manageable number of domains that also successfully use modern software engineering.  The minimum number of projects per domain should be 1. Obtain said list of representative projects.

\begin{itemize}
\item Example sources: Astronomy source code library; high citation software papers; CIG software list; personal domain knowledge.
\end{itemize}

\item Create a case study sheet made of questions to be posed about existing research groups.
\item Perform review.
\item Aggregate results and document.  
\item Document workflow of getting to these results so it can be repeated for additional domains.
\end{itemize}


\subsubsection{More information \& joining instructions}

To join this group or obtain more information about it, please send an email to: \href{wssspe4-verify-best-practices@googlegroups.com}{Verifying best practices \& metrics for sustainable research software.}

%%%%%%%%%%%%%%%%%%%%%%%%%%%%%%%%%%%%%%%%%%%%%%%%%%%%%%%%%%%%
\subsection{CodeMeta}
\label{sec:CodeMeta}
%%%%%%%%%%%%%%%%%%%%%%%%%%%%%%%%%%%%%%%%%%%%%%%%%%%%%%%%%%%%

\note{Alice to write this}

This group, which included people who had previously been working on the CodeMeta project\footnote{\url{http://codemeta.github.io/}}, formed to address the proposed project,
 ``Define a standard for metadata to be used in current repos including authors, dependencies, ... repo url.''

\subsubsection{Participants}

\begin{itemize}
\item Gabrielle Allen <gdallen@illinois.edu>
\item Stephan Druskat <stephan.druskat@hu-berlin.de>
\item Iain Emsley <iain.emsley@oerc.ox.ac.uk>
\item Carole Goble <carole.goble@manchester.ac.uk>
\item Chris Gwilliams <gwilliamsc@cardiff.ac.uk>
\item Rafael Jimenez <rafael.jimenez@elixir-europe.org>
\item Frank L\"{o}ffler <knarf@cct.lsu.edu>
\item Kyle Niemeyer <Kyle.Niemeyer@oregonstate.edu>
\item Thomas Robitaille <thomas.robitaille@gmail.com>
\item Rob Welch <py12rw@leeds.ac.uk>
\item Alice Allen <aallen@ascl.net>
\end{itemize}

\subsubsection{Working group objective}

The primary objective of this working group was to find ways to help the CodeMeta project come to fruition. CodeMeta seeks in part to create a ``Rosetta Stone'' for software metadata to facilitate retaining such metadata between repositories, services, registries, indexers, publishers, citation managers, and other entities that create, ingest, use, and/or store metadata about software. The project also wants to establish a JSON-LD schema as a tool for making metadata machine-readable~\cite{CodeMeta_schema}.

\subsubsection{Gap or challenge}

Research software is often not shared; that that is shared may not have much metadata associated with it, and that which does exist often does not travel further than the website on which the software resides. CodeMeta wants to incentivize software developers to release their software, encourage the development of metadata for it, enable credit assignment and citation of research software and increase its discoverability, more easily track dependencies, and enable reuse of software metadata, all goals that WSSSPE attendees have great interest in supporting. 

\subsubsection{Relevant people and resources}

The CodeMeta team members (as of the WSSSPE4 meeting) are listed below; the CodeMeta project leads are shown in bold, and team members who also part of the WSSSPE working group are shown in italics :

\begin{itemize}
\item {\bf Carl Boettiger}, UC Berkeley
\item {\bf Matt Jones}, NCEAS
\item {\bf Arfon Smith}, Space Telescope Science Institute
\item {\bf Abby Mayes}, Mozilla Science Lab
\item Yolanda Gil, USC ISI
\item Peter Slaughter, NCEAS
\item Patricia Cruse, DataCite
\item Neil Chue Hong SSI
\item Merc\`{e} Crosas, Harvard IQSS
\item Martin Fenner, DataCite
\item Mark Hahnel, figshare
\item Luke Coy, RIT \& MSL
\item {\em Kyle Niemeyer}, Oregon State
\item Krzysztof Nowak, Zenodo
\item Daniel S. Katz, NCSA
\item {\em Carole Goble}, University of Manchester
\item Ashley Sands, UCLA
\item {\em Alice Allen}, ASCL

\end {itemize}

Resources already established by the CodeMeta team can be found online: 

\begin{itemize}
\item Main website and meeting information (\url{http://codemeta.github.io/})
\item Github repository (\url{https://github.com/codemeta/codemeta})
\item List of milestones (\url{https://github.com/codemeta/codemeta/milestones})
\item Gitter discussion site (\url{https://gitter.im/codemeta/codemeta})
\end {itemize}

Other resources to draw on are the software metadata vocabularies of Schema.org\footnote{\url{https://schema.org/SoftwareSourceCode}} and DOAP\footnote{\url{https://github.com/ewilderj/doap}}, and the FORCE11 Software Citation Principles~\cite{Smith:2016sc}. Relevant documents on research software, including the Guidelines for persistently identifying software using DataCite~\cite{Gent2015} are available on the UK's Research Data Network's Research Software web page\footnote{\url{https://research-data-network.readme.io/docs/research-software}}.

\subsubsection{Plans}

Our plans are for the working group to engage with those already working on CodeMeta, to examine the existing CodeMeta crosswalk table to see what improvements and additions might be made, and to determine how to engage the community and provide ongoing social engagement and structure. Further, the group wants to assist in the implementation of the specifications and, by providing an outsider's view, contribute suggestions for more understandable project documentation. Finally, the WG seeks a better way or ways to present the crosswalk table to make it more easily understood and consumable by research software communities. 

\subsubsection{SMART steps}

With a comparatively large working group, the use of the established CodeMeta Github repository's issue tracking, and the quick responsiveness of CodeMeta lead Matt Jones to a barrage of questions, comments, and logged issues, several of our SMART steps were completed at WSSSPE; these are identified below in italics.

\begin{itemize}
\item {\em Write to the managers of CodeMeta project to start dialog with the two groups} 
\item {\em Post questions generated by our discussion as issues on the CodeMeta repository}
\item Look at the crosswalk table and Schema.org data elements for common elements
\item Identify text in the project documentation that is unclear and suggest changes
\item Define a list of services to be added to the crosswalk table and match terms
\item {\em Identify two roles and information on CodeMeta that would be useful to these people for engaging with the project}
\item Determine a better way or ways to present the crosswalk table
\item Create a mailing list for WSSSPE CodeMeta participants
\end{itemize}


\subsubsection{More information \& joining instructions}

Notes for the working group were taken in real time in a Google document\footnote{\url{https://docs.google.com/document/d/1UxlHIoBRgVWB8NAXYf4Q0yS7PAqa-EYwFfsNuTKSPbs/edit}} and include email replies in response to WG questions from one of the CodeMeta project leads.

Because of the work done at WSSSPE4, the CodeMeta project README file\footnote{\url{https://github.com/codemeta/codemeta/blob/master/README.md}} was greatly expanded to include a description of the project that is geared to those with little or no prior knowledge of the project, a list of contributors, information on how one can get involved, a brief project history and who is managing the project, and links to additional information. Though a Google group mailing list has been established for the working group, the easiest way to engage with the CodeMeta project is through its Github repository\footnote{\url{https://github.com/codemeta/codemeta}}. 


%%%%%%%%%%%%%%%%%%%%%%%%%%%%%%%%%%%%%%%%%%%%%%%%%%%%%%%%%%%%
\subsection{Software best practices for undergraduates}
\label{sec:best-practices-undergrads}
%%%%%%%%%%%%%%%%%%%%%%%%%%%%%%%%%%%%%%%%%%%%%%%%%%%%%%%%%%%%

\note{Jonah Miller to write this}

% Introduction to group here, including the overall objective of work
% in this area.

This working group was motivated by the perceived prevalence of
so-called ``dark code'' in scientific communities: code written by
individual researchers in an unsustainable way that is never shared
with the larger community. Participants recalled their own experience
working with colleagues who write dark code or inheriting
unsustainable dark code from collaborators.

The question is, then, how to prevent researchers from writing dark
code? The participants hypothesized that the best strategy is to catch
researchers while they are still in training and teach them software
best-practices. Therefore, this working group was formed with the goal
of developing courses on software best practices aimed at
undergraduate students studying domain science. The program might be
similar to a software carpentry or data carpentry workshop but with a
focus for domain scientists.

\subsubsection{Participants}

\begin{itemize}
  \item Jonah Miller <jmiller@perimeterinstitute.ca>
  \item Aleksandra Nenadic <a.nenadic@manchester.ac.uk>
  \item Raniere Silva <raniere.silva@software.ac.uk>
  \item Francisco Queiroz <chico@puc-rio.br>
  \item Hans Fangohr <fangohr@soton.ac.uk>
  \item Prabhjyot Sing <prabhjyot10@gmail.com>
\end{itemize}

\noindent Aleksandra Nenadic and Raniere Silva are involved
in software and data carpentry.

\subsubsection{Working group objective}

Implementation of a course aimed at each domain science is a long term
goal. However, a short-term, achievable goal, is to develop a
curriculum for a course aimed at a single domain science. Since the
participants in the working group have expertise in physics, this is a
natural target.

\subsubsection{Gap or challenge}

The challenge is that people are writing unsustainable scientific
software without ever learning that there is a better way. Here we
focus on training in good software design and engineering in this
context.

\subsubsection{Relevant people and resources}

The development of a successful curriculum relies on the expertise of
software engineers to describe the best practices, domain experts to
describe model problems and work-flow, and instructors to formulate
the pedagogy. Ideally these people will be brought together for a
short workshop or hackathon with the goal of drafting the
curriculum. Later, organizational partners will be required to
actually implement the program.

\subsubsection{Plans}

The working group will seek funding to organize a hackathon to develop
the curriculum and seek the required expertise to invite. The
hackathon will be implemented and a draft of the curriculum will be
written within the next year.

\subsubsection{SMART steps}

The working group will accomplish or has accomplished the following
smart steps:

\begin{enumerate}
\item Create channels of communication including a mailing list, a
  slack channel, and a git repository (done)
\item Review literature on best and good-enough practices for
  scientific computing as well as literature on pedagogy for
  scientific and research computing (by February 1, 2017)
\item Organize a telecon to organize a workshop to write the
  curriculum and decide how to bring in outside experts (by February 1,
  2017)
\item Write a draft of the curriculum (by April 1, 2017)
\item Invite the broader community to contribute to a more complete
  curriculum (by April 30, 2017)
\end{enumerate}

\subsubsection{More information \& joining instructions}

Anyone interested in contributing should get in touch via one of the
following methods:

\begin{itemize}
\item Join our google group/mailing list
  \cite{WSSSPEUndergradGoogleGroup}.
\item Join the WSSSPE Slack \cite{WSSSPESlack} and sign into the
  \texttt{\#wg-undergraduatecourse} channel and say hi.
\item Ask to join the WSSSPE4-undergraduate-course organization
  \cite{WSSSPEUndergradGithub}.
\end{itemize}
%%%%%%%%%%%%%%%%%%%%%%%%%%%%%%%%%%%%%%%%%%%%%%%%%%%%%%%%%%%%
\subsection{Meaningful metrics for sustainable software}
\label{sec:metrics}
%%%%%%%%%%%%%%%%%%%%%%%%%%%%%%%%%%%%%%%%%%%%%%%%%%%%%%%%%%%%

\note{content from Emily}

%Introduction to group here, including the overall objective of work in this area.

Meaningful Metrics for Sustainable Scientific Software aims to increase the visibility on the quality of scientific software, facilitate the reusability of scientific software, and promote the best software practices by standardizing metrics via interviews with scientific software developers. This working group believes improving the current software metrics system will increase software sustainability. Currently, there are inefficiencies regarding software duplication, sustainability, and selection, as well as others, within the scientific software community. In order to address these inefficiencies, Meaningful Metrics for Sustainable Scientific Software aims to create a goal-oriented method to collecting productive metrics by focusing on the developer side of software.

\subsubsection{Participants}

%members of the working group

\begin{itemize}
\item Emily Chen <echen35@illinois.edu>
\item Patricia Lago <pdotlago@gmail.com>
\item Udit Nangia <unangi2@illinois.edu>
\item Tengyu Ma <tengyuma10717@gmail.com>
\item Aseel Aldabjan <a.dabjan@hotmail.com>
\end{itemize}

\subsubsection{Working group objective}

%Specific things the working group wants to accomplish in the context of the larger objective.

In the context of the larger objective, Meaningful Metrics for Sustainable Scientific Software hopes to find an efficient solution for streamlining the process of collecting and utilizing metrics to benefit software sustainability, as well as minimize the current inefficiencies within the scientific software community. Finding meaningful metrics will improve software evaluation and comparison, thus reducing the effort spent on seeking scientific software.

\subsubsection{Gap or challenge}

%What is the gap or challenge being addressed?

The gap being addressed is the lack of a standard for collecting and presenting metrics. This gap delays workflow and creates a multitude of tedious tasks, including searching for the best-fit software, unknowingly duplicating software, and other related busy-work. The need for metric standardization stems from the abundance of scientific, both ?dark? and open source, software and the difficulties of ensuring the software is sustainable. 

\subsubsection{Relevant people and resources}

%What people, groups, or resources are needed.

Relevant people and resources include scientific software developers, researchers that use scientific software, scientific software funding institutions. 

\subsubsection{Plans}

%What tasks will the working group undertake

This working group plans on interviewing scientific software developers to form metrics from the goals they have for their software. 

\subsubsection{SMART steps}

%What are the first SMART steps proposed?

The SMART steps Meaningful Metrics for Scientific Software has proposed are:
\begin{enumerate}
\item Identify the population of scientific software developers who are willing to be interviewed
\item Define the interview questions and organize them into categories
\item Interview the participants and map the survey results on goals
\item Convert the goals to feasible, meaningful metrics
\item Analyze the collected metrics
\end{enumerate}

\subsubsection{More information \& joining instructions}

%How could a reader get more information or get more involved?

For more information, please contact Emily Chen at <echen35@illinois.edu>.

%%%%%%%%%%%%%%%%%%%%%%%%%%%%%%%%%%%%%%%%%%%%%%%%%%%%%%%%%%%%
\subsection{Open Research Index}
\label{sec:open-research-index}
%%%%%%%%%%%%%%%%%%%%%%%%%%%%%%%%%%%%%%%%%%%%%%%%%%%%%%%%%%%%

\note{Dan to write this}

Introduction to group here, including the overall objective of work in this area.

\subsubsection{Participants}

members of the working group

\subsubsection{Working group objective}

Specific things the working group wants to accomplish in the context of the larger objective.

\subsubsection{Gap or challenge}

What is the gap or challenge being addressed?

\subsubsection{Relevant people and resources}

What people, groups, or resources are needed.

\subsubsection{Plans}

What tasks will the working group undertake

\subsubsection{SMART steps}

What are the first SMART steps proposed?

\subsubsection{More information \& joining instructions}

How could a reader get more information or get more involved?

%%%%%%%%%%%%%%%%%%%%%%%%%%%%%%%%%%%%%%%%%%%%%%%%%%%%%%%%%%%%
\section{Survey responses \label{sec:survey}}
%%%%%%%%%%%%%%%%%%%%%%%%%%%%%%%%%%%%%%%%%%%%%%%%%%%%%%%%%%%%

\note{Lorraine will write this}

\todo{discuss survey responses; maybe put detailed responses in an appendix}


%%%%%%%%%%%%%%%%%%%%%%%%%%%%%%%%%%%%%%%%%%%%%%%%%%%%%%%%%%%%
\section{Conclusions} \label{sec:conclusions}
%%%%%%%%%%%%%%%%%%%%%%%%%%%%%%%%%%%%%%%%%%%%%%%%%%%%%%%%%%%%

In WSSSPE4, we

\todo{discuss Twitter activity}


%%%%%%%%%%%%%%%%%%%%%%%%%%%%%%%%%%%%%%%%%%%%%%%%%%%%%%%%%%%%
\section*{Acknowledgments} \label{sec:acks}
%%%%%%%%%%%%%%%%%%%%%%%%%%%%%%%%%%%%%%%%%%%%%%%%%%%%%%%%%%%%

% NSF grant for WSSSPE4
This material is based upon work supported by the National Science
Foundation (ACI-1648293), by the Gordon and Betty Moore Foundation  (GBMF\#5620), and by the Alfred P.~Sloan Foundation (G-2016-7214).

%
\todo{feel free to add stuff here}

\newpage
\appendix
%%%%%%%%%%%%%%%%%%%%%%%%%%%%%%%%%%%%%%%%%%%%%%%%%%%%%%%%%%%%
\section{Organizing committee}  \label{sec:orgcom}
%%%%%%%%%%%%%%%%%%%%%%%%%%%%%%%%%%%%%%%%%%%%%%%%%%%%%%%%%%%%
%\todo{Do we want email addresses here?}

The following is the list of WSSSPE4 organizers.

{\scriptsize
\begin{longtable}{lll}
Daniel S. Katz &  University of Chicago \& Argonne National Laboratory, USA\\
Gabrielle Allen &  University of Illinois Urbana-Champaign, USA\\
Neil Chue Hong &  Software Sustainability Institute.  University of Edinburgh, UK\\
Sou-Cheng (Terrya) Choi &  Illinois Institute of Technology \& University of Chicago, USA\\
Sandra Gesing &  University of Notre Dame,  USA\\
Lorraine Hwang &   University of California, Davis, USA\\
Manish Parashar &  Rutgers University, USA\\
Erin Robinson &  Foundation for Earth Science, USA (local organizer)\\
Matthew Turk &  University of Illinois Urbana-Champaign, USA\\
Colin C. Venters &  University of Huddersfield, UK

\end{longtable}
}


%%%%%%%%%%%%%%%%%%%%%%%%%%%%%%%%%%%%%%%%%%%%%%%%%%%%%%%%%%%%
\section{Attendees}  \label{sec:attendees}
%%%%%%%%%%%%%%%%%%%%%%%%%%%%%%%%%%%%%%%%%%%%%%%%%%%%%%%%%%%%
The following is a list of attendees at WSSSPE4.

{\scriptsize
\begin{longtable}{lll}
\input{participants}
\end{longtable}
}

%%%%%%%%%%%%%%%%%%%%%%%%%%%%%%%%%%%%%%%%%%%%%%%%%%%%%%%%%%%%
\section{Travel award recipients}  \label{sec:awardees}
%%%%%%%%%%%%%%%%%%%%%%%%%%%%%%%%%%%%%%%%%%%%%%%%%%%%%%%%%%%%
%\todo{Do we want email addresses here?}
The following table contains the list of WSSSPE4 travel award recipients, where
\textsuperscript{*} and \textsuperscript{\textdagger} indicate students and
early-career researchers, respectively.

{\scriptsize
\begin{longtable}{lll}
Alice Allen & Astrophysics Source Code Library\\
Steven Brandt &  Louisiana State University\\
Jeffrey Carver &  University of Alabama\\
Emily Chen & NCSA, University of Illinois\\
Sou-Cheng Choi & NORC at the University of Chicago \&  Illinois Institute of Technology\\
Yuhan Ding &  Illinois Institute of Technology\\
Lorraine Hwang &  CIG,  UC Davis\\
Ray Idaszak &  RENCI, University of North Carolina at Chapel Hill\\
Frank L\"{o}ffler &  Louisiana State University\\
Abigail Cabunoc Mayes &  Mozilla Science Lab\\
Pate Motter &  University of Colorado\\
Kyle Niemeyer &  Oregon State University\\
Birgit Penzenstadler &  California State University Long Beach\\
Bernadette Randles &  UCLA\\
%Ilian Todorov &  STFC Daresbury Laboratory\\
Nic Weber &  University of Washington iSchool

\end{longtable}
}

%%%%%%%%%%%%%%%%%%%%%%%%%%%%%%%%%%%%%%%%%%%%%%%%%%%%%%%%%%%%
\section{Program committee}  \label{sec:progcom}
%%%%%%%%%%%%%%%%%%%%%%%%%%%%%%%%%%%%%%%%%%%%%%%%%%%%%%%%%%%%
The following table lists the WSSSPE4 program committee.

{\scriptsize
\begin{longtable}{lll}
\input{program_committee}
\end{longtable}
}

%%%%%%%%%%%%%%%%%%%%%%%%%%%%%%%%%%%%%%%%%%%%%%%%%%%%%%%%%%%%%
\section{Best Practices Group Discussion}
\label{sec:appendix_best_practices}
%%%%%%%%%%%%%%%%%%%%%%%%%%%%%%%%%%%%%%%%%%%%%%%%%%%%%%%%%%%%

\todo{add POC here}

\subsection{Group Members}
\katznote{add affils for all, please}

\begin{itemize}
\item Abani Patra -- University at Buffalo
\item Sandra Gesing -- University of Notre Dame
\item Neil Chue Hong -- Software Sustainability Institute
\item Gregory Tucker -- University of Colorado at Boulder
\item Birgit Penzens -- California State University Long Beach
\item Abigail Cabunoc Mayes -- Mozilla Foundation
\item Jeff Carver -- University of Alabama
\item Frank L\"{o}ffler -- Louisiana State University 
\item Colin Venter --  University of Huddersfield
\item Lorraine Hwang -- UC Davis 
\item Sou-Cheng Choi -- NORC at the University of Chicago \&  Illinois Institute of Technology
\item Suresh Marru -- Indiana University
\item Don Middleton -- NCAR 
\item Daniel Katz --  University of Chicago \& Argonne National Laboratory
\item Kyle Niemeyer -- Oregon State University
\item Jeffrey Carver -- University of Alabama
\item Dan Gunter -- LBNL
\item Alexander Konovalov -- \choinote{TBD}
\item Tom Crick --  \choinote{TBD}

\end{itemize}

\subsection{Summary of Discussion}

Core questions that will need to be explored are in knowledge management, 
(transitions between people), reliability (reproducibility), usability, and how a software tool becomes part of the core workflow of well identified users (stakeholders)
relating to tool success and hence sustainability. \katznote{prev sentence is complex and awkward} Ideas 
that may need to be explored include:
\begin{itemize}

\item Requirements engineering to create tools with immediate uptake;

\item When should software ``die''?

\item Catering to disruptive developments in environment, e.g., new hardware,
new methodology;

\item Dimensions of sustainability -- economic, technical, environmental,
declining interest in primary application area), \katznote{not sure what the
prev. comment goes with} social.

\end{itemize}

Sustainability requires community participation in code development and/or a
wide adoption of software. The larger the community base is using a piece of
software, the better are the funding possibilities and thus also the
sustainability options. Additionally,  developers’ commitment to an application is
essential and experience shows that software packages with an evangelist
imposing strong inspiration and discipline are more likely to achieve
sustainability. While a single person can push sustainability to a certain
level, open source software also needs sustained commitment from the developer
community. Such sustained commitments include diverse tasks and roles, which can
be fulfilled by diverse developers with different knowledge levels. Besides
developing software and appropriate software management with measures for
extensibility and scalability of the software, active (expertise) support for
users via a user forum with a quick turnaround is crucial. The barrier to entry
for the community as users as well as developers has to be as low as possible.

\subsection{Description of Opportunity, Challenges, and Obstacles}

The opportunity lies in collaboration on a white paper, which will be revisited
regularly for further improvements, to enhance knowledge of the state of best
practices, resulting in a peer-reviewed paper. We would like to reach a wide
community by doing this. But these are also the challenges and obstacles -- to
get everyone to contribute to the paper and to reach the community.

\subsection{Key Next Steps}

The key next steps are to write an introduction, reach out to the co-authors,
and to agree on the scope of the white paper.

\subsection{Plan for Future Organization}

Sandra Gesing and Abani Patra are the main editors and will organize the overall
communication and the paper. Sections will be assigned to diverse co-authors.

\subsection{What Else is Needed?}

At the moment we do not see any further requirements.

\subsection{Key Milestones and Responsible Parties}
\begin{itemize}
\item 15 Nov: Introduction and scope finished (Abani Patra/Sandra Gesing)
\item 15 Nov: Sections assigned (Abani Patra/Sandra Gesing)
\item 31 Jan: Analyzing funding possibilities for survey
\item 31 Jan: First version of each section
\item 15 Feb: Distribution to the WSSSPE community
\item 31 Mar: Final version of the white paper
\item 30 Apr: Submission to a peer-reviewed journal?
\end{itemize}

\subsection{Description of Funding Needed}
We might need funding for a journal publication (open-access options).


\bibliographystyle{vancouver}

\bibliography{wssspe}
\end{document}
