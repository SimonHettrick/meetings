\documentclass[11pt, oneside]{amsart}
\pdfoutput=1

\usepackage{amsmath,amssymb,amsmath}

\usepackage{color}
\usepackage{dcolumn}
\usepackage{float,caption}
\usepackage{graphicx}

\usepackage[T1]{fontenc}
\usepackage[utf8]{inputenc}
\usepackage{lmodern}

\usepackage[activate={true,nocompatibility},final,tracking=true,kerning=true,spacing=true,factor=1100,stretch=10,shrink=10]{microtype}
\microtypecontext{spacing=nonfrench}
\usepackage{xspace}

\usepackage{rotating}
\usepackage{subfigure}
\usepackage{psfrag}
\usepackage{tabularx}
\usepackage[hyphens]{url}
\usepackage{wrapfig}
\usepackage{longtable}
\usepackage{verbatim}
\usepackage{booktabs,multicol,multirow}

%better printing of numbers
\usepackage[english]{babel}
\usepackage{textcomp}
\usepackage{csquotes}

% The following three lines are used for displaying footnote in tables.
\usepackage{footnote}
\makesavenoteenv{tabular}
\makesavenoteenv{table}


\usepackage{enumitem}
\setlist{leftmargin=7mm}

%\setcounter{secnumdepth}{3}
%\setcounter{tocdepth}{3}


\usepackage[bookmarks, bookmarksopen, bookmarksnumbered]{hyperref}
\usepackage[all]{hypcap}
\urlstyle{rm}

\definecolor{orange}{rgb}{1.0,0.3,0.0}
\definecolor{violet}{rgb}{0.75,0,1}
\definecolor{darkgreen}{rgb}{0,0.6,0}
\definecolor{cyan}{rgb}{0.2,0.7,0.7}
\definecolor{blueish}{rgb}{0.2,0.2,0.8}
\definecolor{darkblue}{rgb}{0.1,0.1,0.9}

\newcommand{\todo}[1]{{\color{blue}$\blacksquare$~\textsf{[TODO: #1]}}}
\newcommand{\note}[1]{ {\textcolor{blueish}    { ***Note:      #1 }}}
\newcommand{\katznote}[1]{ {\textcolor{magenta}    { ***Dan:      #1 }}}
\newcommand{\gabnote}[1]{ {\textcolor{cyan}    { ***Gabrielle:     #1 }}}
\newcommand{\nchnote}[1]{  {\textcolor{orange}      { ***Neil: #1 }}}

% Don't use tt font for urls
\urlstyle{rm}

\usepackage[letterpaper, margin=1in]{geometry}
% You can use a baselinestretch of down to 0.9
\renewcommand{\baselinestretch}{0.96}

\sloppypar

\begin{document}

\title[]{Report on the Fourth Working towards Sustainable Software for Science: Practice and Experiences (WSSSPE4) workshop}

\author{Daniel S.\ Katz$^{(1)}$,
Kyle E.\ Niemeyer$^{(2)}$,
}

%
\thanks{{}$^{(1)}$ \hspace{-1ex}Computation Institute,
University of Illinois at Urbana--Champaign, IL, USA; d.katz@ieee.org}
%
\thanks{{}$^{(2)}$ School of Mechanical, Industrial, and Manufacturing Engineering,
Oregon State University, OR, USA; kyle.niemeyer@oregonstate.edu}
%


\begin{abstract}
This report records and discusses the Fourth Working towards Sustainable
Software for Science: Practice and Experiences (WSSSPE4) workshop. The report
includes a description of the keynote presentation of the workshop,
which % TODO
% It also summarizes a set of lightning
% talks in which speakers highlighted to-the-point lessons and challenges
% pertaining to sustaining scientific software.
% The final and main contribution of the report is a summary of the
% discussions, future steps, and future organization for a set of self-organized
% working groups on topics including developing pathways to funding scientific
% software; constructing useful common metrics for crediting software
% stakeholders; identifying principles for sustainable software engineering
% design; reaching out to research software organizations around the world; and
% building communities for software sustainability. For each group, we include a
% point of contact and a landing page that can be used by those who want to join
% that group's future activities. The main challenge left by the workshop is to
% see if the groups will execute these activities that they have scheduled, and
% how the WSSSPE community can encourage this to happen.

\end{abstract}



\maketitle
\newpage

%%%%%%%%%%%%%%%%%%%%%%%%%%%%%%%%%%%%%%%%%%%%%%%%%%%%%%%%%%%%
\section{Introduction} \label{sec:intro}
%%%%%%%%%%%%%%%%%%%%%%%%%%%%%%%%%%%%%%%%%%%%%%%%%%%%%%%%%%%%

The Fourth Workshop on Sustainable Software for Science: Practice and Experiences
(WSSSPE4)\footnote{\url{http://wssspe.researchcomputing.org.uk/wssspe4/}} was
held on 12--14 September 2016 in Manchester, England, UK.
This location and date was selected so that WSSSPE4 immediately preceded the
First Conference of Research Software Engineers.
Previous events in the WSSSPE series include
WSSSPE1\footnote{\url{http://wssspe.researchcomputing.org.uk/wssspe1/}}~\cite{WSSSPE1-pre-report,WSSSPE1},
held in conjunction with SC13;
WSSSPE1.1\footnote{\url{http://wssspe.researchcomputing.org.uk/wssspe1-1/}}, a
focused workshop organized jointly with the SciPy
conference\footnote{\url{https://conference.scipy.org/scipy2014/participate/wssspe/}};
WSSSPE2\footnote{\url{http://wssspe.researchcomputing.org.uk/wssspe2/}}~\cite{WSSSPE2-pre-report,WSSSPE2},
held in conjunction with SC14;
WSSSPE2.1\footnote{\url{http://wssspe.researchcomputing.org.uk/wssspe2-1/}}, a
focused workshop organized again jointly with
SciPy\footnote{\url{http://scipy2015.scipy.org/ehome/115969/286469/}};
and WSSSPE3\footnote{\url{http://wssspe.researchcomputing.org.uk/wssspe3/}}~\cite{WSSSPE3},
held in Boulder, Colorado, USA.
Note that the WSSSPE series changed its name from ``Workshop on Sustainable
Software for Science: Practice and Experiences'' to ``Working towards
Sustainable Software for Science: Practice and Experiences'' to better reflect
the movement towards a community in addition to the associated workshop series.

\todo{Introduction}

The WSSSPE4 workshop included multiple mechanisms for participation and
encouraged team building around solutions. WSSSPE4 strongly encouraged participation
of early-career scientists, postdoctoral researchers, graduate students,
early-stage researchers, and those from underrepresented groups,
with funds provided to the conference organizers by the National Science
Foundation (NSF), the Alfred P.~Sloan Foundation, and the Software
Sustainability Institute (SSI) to support the travel of potential participants
who would not otherwise be able to attend the workshop. These
funds allowed 29 additional people to attend and participate. A subset of the
organizing committee reviewed 44 applications for travel support and
competitively selected the awardees, including 11 students and five early-career
researchers. In addition, the travel award subcommittee tried to increase the
diversity of applicants by forwarding the notice of travel support to
organizations including PyLadies, Django Girls, Women Who Code, and
Women in HPC.\todo{add links}
\todo{did we reach out to any others?}

WSSSPE4 also included two professional event organizers/facilitators from
KnowInnovation who helped the organizing committee plan the workshop agenda,
and during the workshop, they actively engaged participants with various
tools, activities, and reminders.

This report is based on
\todo{what report is based on}

%%%%%%%%%%%%%%%%%%%%%%%%%%%%%%%%%%%%%%%%%%%%%%%%%%%%%%%%%%%%
\section{Calls for participation} \label{sec:preworkshop}
%%%%%%%%%%%%%%%%%%%%%%%%%%%%%%%%%%%%%%%%%%%%%%%%%%%%%%%%%%%%

WSSSPE4 was based on the work done in WSSSPE1, WSSSPE2, and WSSSPE3, but aimed
at \todo{what aimed at?}
as the calls for participation said:

two tracks:

\begin{quote}
    call for participation
\end{quote}

\todo{update}
The calls for participation requested papers describing lightning talks,
where each author could make a brief statement about work that either had been
done or was needed;
idea papers, which present implementable proposals related to Track 1;
position papers, which are longer, not previously published papers related to
Track 2 specifically discussing what we can do to improve sustainable scientific
software in the short term, starting with where we are today;
experience papers, longer papers related to Track 2 that discuss current
practices and experiences and how they have been used to improve the quality of
today’s research software and/or the experiences of its developers;
and demos, brief papers describing a 10--15 minute demonstration of a tool or
process relevant to Track 2 that improves the quality of today's research
software and\slash or the experiences of its developers.

There were~24 lightning talks submitted; after a peer-review process, 16
were accepted, as discussed further in Section~\ref{sec:lightning}.
\todo{update}

%
%\begin{quote}
\begin{itemize}
\renewcommand{\labelenumi}{\textbf{\theenumi}.}
\setlength{\rightmargin}{1em}

\item Development and Community
\begin{itemize}
    \item Best practices for developing sustainable software
    \item Models for funding specialist expertise in software collaborations
    \item Software tools that aid sustainability
    \item Academia/industry interaction
    \item Refactoring/improving legacy scientific software
    \item Engineering design for sustainable software
    \item Metrics for the success of scientific software
    \item Adaptation of mainstream software practices for scientific software
\end{itemize}

\item Professionalization
\begin{itemize}
    \item Career paths
    \item RSE as a brand
    \item RSE outside of the UK or Europe
    \item Increase incentives in publishing, funding and promotion for better software
\end{itemize}

\item Training
\begin{itemize}
    \item Training for developing sustainable software
    \item Curriculum for software sustainability
\end{itemize}

\item Credit
\begin{itemize}
    \item Making the existing credit and citation ecosystem work better for software
    \item Future credit and citation ecosystem
    \item Software contributions as a part of tenure review
    \item Case studies of receiving credit for software contributions
    \item Awards and recognition that encourage sustainable software
\end{itemize}

\item Software publishing
\begin{itemize}
    \item Journals and alternative venues for publishing software
    \item Review processes for published software
\end{itemize}

\item Software discoverability/reuse
\begin{itemize}
    \item Proposals and case studies
\end{itemize}

\item Reproducibility and testing
\begin{itemize}
    \item Reproducibility in conferences and journals
    \item Best practices for code testing and code review
\end{itemize}

\end{itemize}

%%%%%%%%%%%%%%%%%%%%%%%%%%%%%%%%%%%%%%%%%%%%%%%%%%%%%%%%%%%%
\section{Keynote}\label{sec:keynote}
%%%%%%%%%%%%%%%%%%%%%%%%%%%%%%%%%%%%%%%%%%%%%%%%%%%%%%%%%%%%



%%%%%%%%%%%%%%%%%%%%%%%%%%%%%%%%%%%%%%%%%%%%%%%%%%%%%%%%%%%%
\section{Lightning talks} \label{sec:lightning}
%%%%%%%%%%%%%%%%%%%%%%%%%%%%%%%%%%%%%%%%%%%%%%%%%%%%%%%%%%%%
\begin{comment}
\note{
\href{http://wssspe.researchcomputing.org.uk/wssspe4/agenda/}{Slides.}}
\end{comment}

Intro and description
%
\begin{enumerate}
\item \textbf{Names: \emph{title}}.
Description

\end{enumerate}

%%%%%%%%%%%%%%%%%%%%%%%%%%%%%%%%%%%%%%%%%%%%%%%%%%%%%%%%%%%%
\section{Panel discussion} \label{sec:panel}
%%%%%%%%%%%%%%%%%%%%%%%%%%%%%%%%%%%%%%%%%%%%%%%%%%%%%%%%%%%%

\todo{description of panel discussion on best practices}

%%%%%%%%%%%%%%%%%%%%%%%%%%%%%%%%%%%%%%%%%%%%%%%%%%%%%%%%%%%%
\section{Working groups} \label{sec:WGs}
%%%%%%%%%%%%%%%%%%%%%%%%%%%%%%%%%%%%%%%%%%%%%%%%%%%%%%%%%%%%

\todo{go through subsections that have both content here and appendix, and add pointer from text in the subsection here to the corresponding appendix}

%%%%%%%%%%%%%%%%%%%%%%%%%%%%%%%%%%%%%%%%%%%%%%%%%%%%%%%%%%%%
\subsection{Open Research Index}
\label{sec:open-research-index}
%%%%%%%%%%%%%%%%%%%%%%%%%%%%%%%%%%%%%%%%%%%%%%%%%%%%%%%%%%%%

\note{Dan to write this}

Introduction to group here, including the overall objective of work in this area.

\subsubsection{Participants}

members of the working group

\subsubsection{Working group objective}

Specific things the working group wants to accomplish in the context of the larger objective.

\subsubsection{Gap or challenge}

What is the gap or challenge being addressed?

\subsubsection{Relevant people and resources}

What people, groups, or resources are needed.

\subsubsection{Plans}

What tasks will the working group undertake

\subsubsection{SMART steps}

What are the first SMART steps proposed?

\subsubsection{More information \& joining instructions}

How could a reader get more information or get more involved?

%%%%%%%%%%%%%%%%%%%%%%%%%%%%%%%%%%%%%%%%%%%%%%%%%%%%%%%%%%%%
\section{Conclusions} \label{sec:conclusions}
%%%%%%%%%%%%%%%%%%%%%%%%%%%%%%%%%%%%%%%%%%%%%%%%%%%%%%%%%%%%

In WSSSPE4, we


%%%%%%%%%%%%%%%%%%%%%%%%%%%%%%%%%%%%%%%%%%%%%%%%%%%%%%%%%%%%
\section*{Acknowledgments} \label{sec:acks}
%%%%%%%%%%%%%%%%%%%%%%%%%%%%%%%%%%%%%%%%%%%%%%%%%%%%%%%%%%%%

% NSF grant for WSSSPE4
This material is based upon work supported by the National Science
Foundation under grant ACI-1648293, and by the Alfred P.~Sloan Foundation
under grant G-2016-7214.

%
\todo{feel free to add stuff here}

\newpage
\appendix
%%%%%%%%%%%%%%%%%%%%%%%%%%%%%%%%%%%%%%%%%%%%%%%%%%%%%%%%%%%%
\section{Organizing committee}  \label{sec:orgcom}
%%%%%%%%%%%%%%%%%%%%%%%%%%%%%%%%%%%%%%%%%%%%%%%%%%%%%%%%%%%%
%\todo{Do we want email addresses here?}

The following is the list of WSSSPE4 organizers.

{\scriptsize
\begin{longtable}{lll}
Daniel S. Katz &  University of Chicago \& Argonne National Laboratory, USA\\
Gabrielle Allen &  University of Illinois Urbana-Champaign, USA\\
Neil Chue Hong &  Software Sustainability Institute.  University of Edinburgh, UK\\
Sou-Cheng (Terrya) Choi &  Illinois Institute of Technology \& University of Chicago, USA\\
Sandra Gesing &  University of Notre Dame,  USA\\
Lorraine Hwang &   University of California, Davis, USA\\
Manish Parashar &  Rutgers University, USA\\
Erin Robinson &  Foundation for Earth Science, USA (local organizer)\\
Matthew Turk &  University of Illinois Urbana-Champaign, USA\\
Colin C. Venters &  University of Huddersfield, UK

\end{longtable}
}


%%%%%%%%%%%%%%%%%%%%%%%%%%%%%%%%%%%%%%%%%%%%%%%%%%%%%%%%%%%%
\section{Attendees}  \label{sec:attendees}
%%%%%%%%%%%%%%%%%%%%%%%%%%%%%%%%%%%%%%%%%%%%%%%%%%%%%%%%%%%%
The following is a list of attendees at WSSSPE4.

{\scriptsize
\begin{longtable}{lll}
\input{participants}
\end{longtable}
}

%%%%%%%%%%%%%%%%%%%%%%%%%%%%%%%%%%%%%%%%%%%%%%%%%%%%%%%%%%%%
\section{Travel award recipients}  \label{sec:awardees}
%%%%%%%%%%%%%%%%%%%%%%%%%%%%%%%%%%%%%%%%%%%%%%%%%%%%%%%%%%%%
%\todo{Do we want email addresses here?}
The following table contains the list of WSSSPE4 travel award recipients, where
\textsuperscript{*} and \textsuperscript{\textdagger} indicate students and
early-career researchers, respectively.

{\scriptsize
\begin{longtable}{lll}
Alice Allen & Astrophysics Source Code Library\\
Steven Brandt &  Louisiana State University\\
Jeffrey Carver &  University of Alabama\\
Emily Chen & NCSA, University of Illinois\\
Sou-Cheng Choi & NORC at the University of Chicago \&  Illinois Institute of Technology\\
Yuhan Ding &  Illinois Institute of Technology\\
Lorraine Hwang &  CIG,  UC Davis\\
Ray Idaszak &  RENCI, University of North Carolina at Chapel Hill\\
Frank L\"{o}ffler &  Louisiana State University\\
Abigail Cabunoc Mayes &  Mozilla Science Lab\\
Pate Motter &  University of Colorado\\
Kyle Niemeyer &  Oregon State University\\
Birgit Penzenstadler &  California State University Long Beach\\
Bernadette Randles &  UCLA\\
%Ilian Todorov &  STFC Daresbury Laboratory\\
Nic Weber &  University of Washington iSchool

\end{longtable}
}


%%%%%%%%%%%%%%%%%%%%%%%%%%%%%%%%%%%%%%%%%%%%%%%%%%%%%%%%%%%%%
\section{Best Practices Group Discussion}
\label{sec:appendix_best_practices}
%%%%%%%%%%%%%%%%%%%%%%%%%%%%%%%%%%%%%%%%%%%%%%%%%%%%%%%%%%%%

\todo{add POC here}

\subsection{Group Members}
\katznote{add affils for all, please}

\begin{itemize}
\item Abani Patra -- University at Buffalo
\item Sandra Gesing -- University of Notre Dame
\item Neil Chue Hong -- Software Sustainability Institute
\item Gregory Tucker -- University of Colorado at Boulder
\item Birgit Penzens -- California State University Long Beach
\item Abigail Cabunoc Mayes -- Mozilla Foundation
\item Jeff Carver -- University of Alabama
\item Frank L\"{o}ffler -- Louisiana State University 
\item Colin Venter --  University of Huddersfield
\item Lorraine Hwang -- UC Davis 
\item Sou-Cheng Choi -- NORC at the University of Chicago \&  Illinois Institute of Technology
\item Suresh Marru -- Indiana University
\item Don Middleton -- NCAR 
\item Daniel Katz --  University of Chicago \& Argonne National Laboratory
\item Kyle Niemeyer -- Oregon State University
\item Jeffrey Carver -- University of Alabama
\item Dan Gunter -- LBNL
\item Alexander Konovalov -- \choinote{TBD}
\item Tom Crick --  \choinote{TBD}

\end{itemize}

\subsection{Summary of Discussion}

Core questions that will need to be explored are in knowledge management, 
(transitions between people), reliability (reproducibility), usability, and how a software tool becomes part of the core workflow of well identified users (stakeholders)
relating to tool success and hence sustainability. \katznote{prev sentence is complex and awkward} Ideas 
that may need to be explored include:
\begin{itemize}

\item Requirements engineering to create tools with immediate uptake;

\item When should software ``die''?

\item Catering to disruptive developments in environment, e.g., new hardware,
new methodology;

\item Dimensions of sustainability -- economic, technical, environmental,
declining interest in primary application area), \katznote{not sure what the
prev. comment goes with} social.

\end{itemize}

Sustainability requires community participation in code development and/or a
wide adoption of software. The larger the community base is using a piece of
software, the better are the funding possibilities and thus also the
sustainability options. Additionally,  developers’ commitment to an application is
essential and experience shows that software packages with an evangelist
imposing strong inspiration and discipline are more likely to achieve
sustainability. While a single person can push sustainability to a certain
level, open source software also needs sustained commitment from the developer
community. Such sustained commitments include diverse tasks and roles, which can
be fulfilled by diverse developers with different knowledge levels. Besides
developing software and appropriate software management with measures for
extensibility and scalability of the software, active (expertise) support for
users via a user forum with a quick turnaround is crucial. The barrier to entry
for the community as users as well as developers has to be as low as possible.

\subsection{Description of Opportunity, Challenges, and Obstacles}

The opportunity lies in collaboration on a white paper, which will be revisited
regularly for further improvements, to enhance knowledge of the state of best
practices, resulting in a peer-reviewed paper. We would like to reach a wide
community by doing this. But these are also the challenges and obstacles -- to
get everyone to contribute to the paper and to reach the community.

\subsection{Key Next Steps}

The key next steps are to write an introduction, reach out to the co-authors,
and to agree on the scope of the white paper.

\subsection{Plan for Future Organization}

Sandra Gesing and Abani Patra are the main editors and will organize the overall
communication and the paper. Sections will be assigned to diverse co-authors.

\subsection{What Else is Needed?}

At the moment we do not see any further requirements.

\subsection{Key Milestones and Responsible Parties}
\begin{itemize}
\item 15 Nov: Introduction and scope finished (Abani Patra/Sandra Gesing)
\item 15 Nov: Sections assigned (Abani Patra/Sandra Gesing)
\item 31 Jan: Analyzing funding possibilities for survey
\item 31 Jan: First version of each section
\item 15 Feb: Distribution to the WSSSPE community
\item 31 Mar: Final version of the white paper
\item 30 Apr: Submission to a peer-reviewed journal?
\end{itemize}

\subsection{Description of Funding Needed}
We might need funding for a journal publication (open-access options).


\bibliographystyle{vancouver}

\bibliography{wssspe}
\end{document}
