%%%%%%%%%%%%%%%%%%%%%%%%%%%%%%%%%%%%%%%%%%%%%%%%%%%%%%%%%%%%
\section{Training Working Group Discussion}
\label{sec:appendix_training}
%%%%%%%%%%%%%%%%%%%%%%%%%%%%%%%%%%%%%%%%%%%%%%%%%%%%%%%%%%%%

Nick Jones\footnote{email: \href{mailto:nick.jones@nesi.org.nz}{nick.jones@nesi.org.nz}} will serve as the point of contact for this working group, and be responsible for ensuring timely progress of the planned actions.

\subsection{Group Members}
\begin{itemize}
\item Nick Jones -- New Zealand eScience Infrastructure
\item Iain Larmour -- Engineering \& Physical Sciences Research Council (UK)
\item Erin Robinson -- Foundation for Earth Science
\end{itemize}

\subsection{Summary of Discussion}


While little training focuses specifically on this outcome \katznote{sustainable software?}, a variety of training activities increase researcher awareness of and engagement with software professionals and software engineering practices. Research Software Engineers are being recognized as critical contributors to high quality research; the pathway to acquire and master the relevant skills isn't yet clear; equally those skills required by researchers in general are also not commonly understood nor routinely developed.

The group's discussion explored a rapidly growing array of training that is seen to contribute to sustainable software. The offerings are diverse, including: self paced online modules focused around specific tools; single and multiple day training workshops that raise awareness of a tool chain to support collaborative and shared software development within a research workflow; block courses specializing on particular methods, technologies, and applications; academic programs at undergraduate and masters levels; doctoral training programs that in part contain requisite skills training activities.

While some of this training focuses on applying software engineering practices within the context of research, meeting the values and goals of research are less often incorporated as explicit learning outcomes. With software (and similarly, data) often the only tangible artifact of a research method or protocol, the dependency between software applications, and the quality of research adds complexity to the learners journey. In recognition of the longer term investment required by researchers to integrate such skills into their research practices, many activities are focusing on emotionally engaging researchers and cohorts, to build a sense of shared purpose beyond the obvious goal of technical skill acquisition.

In reviewing current training activities, the group identified a variety of perspectives seen as useful in positioning activities in ways to better communicate why and when best to apply each activity. Training can be categorized on a variety of spectra, with content and delivery focused from: programming to research; basic to advanced; technical to emotional; informal to formal; self paced to participative. A few attempts have been made to situate a cross section of training activities within such dimensions, creating easier means of communicating the value of any specific opportunity and the pathways across opportunities over time.

Evaluation of training delivery and outcomes is seen as a weakness common to most non academic training activities. Opportunities for measuring success in delivering training start simply with collecting a Net Promoter Score, which lets those delivering training know whether attendees are likely to recommend the training to others. In looking at the longer term outcomes for the learner, frameworks such as Blooms Taxonomy and Kirkpatrick Evaluations offer possible approaches.

In this latter case of formal evaluation, ownership of evaluation as a component of career development for any researcher appears mostly absent. While academic research institutions have professional development centers to support research staff, the skills taught which might impact on sustainable software are limited at best, and lack a clear and coherent development pathway.

Coordination of these training projects will depend on buy in from a broad range of training program and activity leaders, suggesting a key opportunity lies in identifying and bringing together these people on a regular basis.

\subsection{Description of Opportunity, Challenges, and Obstacles}

Software skills are needed by an increasing array of researchers and fields. The training arc is not well-defined, with a sometimes baffling array of training opportunities responding to various facets of skill deficit and need. Given this current complexity, coordination across training projects would create common frames of reference, communicating and integrating activities to better serve the needs of researchers. 

Building this community could lift the maturity of training projects and capabilities, enabling more advanced approaches to address key gaps in evaluation, career development, and a lift in the standard of research practices. 

In aiming at these opportunities it will be necessary to find the means to support those involved in leading training activities to allocate time to coordination activities, which will often sit beyond their current scope of responsibility. 

These activities are also distributed globally, with no single country or region offering a comprehensive set of capabilities and initiatives. Any coordination activity will therefore need to raise the profile of the opportunity gap with relevant research funders and policy makers.

\subsection{Key Next Steps}

The goal of the following next steps is to quickly test whether there is interest in establishing a community committed to increasing the degree of coordination across training projects.

\begin{enumerate}
    \item Hold a virtual meeting by December 2015, to bring together a broader group of interest in this topic, with specific goals to:
    \begin{enumerate}
        \item Identify programs with existing activity aimed at integrating across training projects.
        \item Identify training projects with an interest in participating in coordination efforts.
        \item Identify funding opportunities to bring together training program and project leaders to identify shared goals for future coordination of activities.
        \item Agree a communications plan to qualify whether programs, projects, and funders are interested in engaging and committing to ongoing activities.
    \end{enumerate}
    \item Review progress within 3 months, to establish next steps, if any.
\end{enumerate}

\subsection{Plan for Future Organization}

Continue to track progress by posting comments to WSSSPE3 issue.

\subsection{What Else is Needed?}

If the group moves from early stage formation into working towards shared goals, expertise will likely be required in pedagogy and training evaluation

\subsection{Key Milestones and Responsible Parties}
\begin{enumerate}
    \item During October, Nick Jones and Erin Robinson to draft WSSSPE3 report back. 
    \item Before December 2015, Nick Jones and Erin Robinson to call a meeting of the broader group, to review key next steps. 
    \item 1st quarter 2016 - if willing parties identified, draft workshop proposal and identify relevant forum, including future WSSSPE events
\end{enumerate}

\subsection{Description of Funding Needed}

Workshop/RCN travel funding to bring together key program, project, and funder representatives from across North America, EU, UK, Australasia. In addition, funding to support work on better defining the landscape of training activities, the useful perspectives in communicating the value of the varied training projects, and the possible pathways through training activities over time.
