\subsection{Research Programming Expertise -- how to fund it and sustain it.}



\subsubsection{Temporary direct paste-in}

What is the summary of the opportunity/challenge and key obstacles?
The issue is how to fund and create a career path for Research Software Engineer’s. RSE’s are going to be key to promoting co-development, which seems like a highly efficient mechanism for scientific software development.
Funding is “hard” for individual PIs to find support for RSEs. How can we make it easier for folks to access RSEs, and in so doing create an RSE niche in the University.
Goal: Demonstrate value of RSEs to both funding agencies and institutions, leading to creation of positions for RSEs.
 We have defined some models where RSEs have been successfully funded, and where they have contributed to scientific software development, and believe this can be extended more broadly.

Next Steps?
White paper, add groups outside the UK that are RSE groups.
Poll for where there are existing RSE groups; how they are funded. James will post existing RSE groups on University College London - RSDG website and ask others to identify themselves. Use this information to collect more information on successful funding models.
Promoting communication and even resource sharing among RSE groups.
Recognition that this is a spectrum, and subject to interpretation, i.e. professionals versus development out of hobby or necessity.
Will we work on the next steps?
Lindsay, Mark, and James will continue to push this agenda forward by turning the notes into a white paper. We will work as peers.
How will we organize ourselves going forward?
Mark will be the scribe, and will work to be sure the notes are translates into a white paper draft as we go forward.
What else do we need?
We would like to create a list of other RSE organizations. How do we define and identify these organizations? How will we support this activity?
How can we credibly document the impact of RSEs on research progress; not just how things go faster, but what discoveries are enabled faster, and what setbacks were avoided because good engineering practices were followed.
What are key Milestones?
Prepare a white paper to solidify our understanding of the issue, and to bring it more into the public consciousness.
Create a forum to promote information and resource sharing across RSEs.
Maybe host an RSE meeting for exchange of thoughts and experiences.
What is the size shape and structure of funding needed?
Support would be required to create a resource sharing structure for RSEs
Support would be required to hold a meeting of RSE groups.



\subsubsection{Why it is important}

Research Software Engineers -- those who contribute to science and scholarship
through computer programming -- are an important part of the team
needed to deliver 21st century research. However, existing academic structures
and systems of funding do not effectively fund and sustain these skills.
The resulting high levels of turnover, and inappropriate incentivisation,
are a significant contributing factor to low levels of reliability and
readability observed in scientific software.

\subsubsection{Fit with related activities}

One solution to this problem would be a successful solution to the problems of
obtaining academic credit for programming work. If this were solved, traditional
academic funding models and career tracks would properly sustain this
contribution to research.

\subsubsection{Discussion}
\todo{short-ish text here}

\subsubsection{Plans}
\todo{short text here - not bullets}

\subsubsection{Landing Page}
\todo{link to landing page}
