\subsection{Funding Research Programmer Expertise}
\label{RSE}

%\katznote{I don't think this section title is exactly right - are there some other options?  Also, the title of Appendix~\ref{sec:appendix_funding_spec_expert} should match, whatever is chosen.  Maybe `Funding Research Programmer Expertise'?}

%\subsubsection{Why it is important}

Research Software Engineers -- those who contribute to science and scholarship
through software development -- are an important part of the team
needed to deliver 21st century research. However, existing academic structures
and systems of funding do not effectively fund and sustain these skills.
The resulting high levels of turnover and inappropriate incentives
are a significant contributing factor to low levels of reliability and
readability observed in scientific software. Moreover, the absence of skilled and experienced 
developers retards progress in key projects, and at times causes important projects to fail completely.

Effective development of software for advanced research requires
that researchers work closely with scientific software developers who understand
the research domain sufficiently to build meaningful software at a reasonable pace. 
This requires a collaborative approach -- where developers who are fully engaged/invested 
in the research context are co-developing software with domain academics.

\subsubsection{Fit with related activities}

The solution we envision entails creating an environment where software developers 
are a stable part of the research team. Such an environment mitigates the risk
of losing a key developer at a critical moment in a projects lifetime, and provides
the benefits of building a store of institutional knowledge about specific projects as well as about software 
development for today's research. Our vision is to find a way to promote a University/research institute environment where
software developers are stable components of research project teams. 

One strategy to promote stability is implementing a mechanism for developers to obtain academic
credit for software development work. With such a mechanism in place, traditional academic funding
models and career tracks could properly sustain individuals for whom software development is their
primary contribution to research. A contributing factor to the problem with the current academic reward system is the
devastating effect on an academic
publication record resulting from time in industry; such postings often develop exactly the skills that research software
engineers need, yet returns to university positions following an industry role are penalized by the current structures.
Retention of senior developers is hard, because these people are highly in demand by the economy. However, people who have a
PhD in science and enter industry, may desire to return for diverse reasons, and should be welcomed back.

While development of new mechanisms in the current academic reward system is a worthy aspirational goal, such a dramatic
change in this structure does not seem likely in a time scale relevant to this working group. Accordingly, our working party
sought alternate solutions that may be achievable within the context of existing academic structures. The group felt that
developing dedicated research software engineering roles within the University, and finding stable funding for those individuals is the most promising mechanism for creating a stable software development staff.

Measures of impact and success for research programming groups, as well as for individual research software engineers, will
be required in order to make the case to the University for continued funding. Research software engineers will not be measured by publications, we hope, but by other measures. Middle-author publications are common for RSEs. Most RSEs welcome co-authorship on papers where the PIs think that the contribution deserves it.

\subsubsection{Discussion}

It is hard for an individual PI in a university or college to support dedicated research software engineering resources, as
the need for, and funding for, these activities is intermittent within the research cycle. To sustain this capacity, therefore, it is necessary to aggregate this work across multiple research groups.

One solution is to fund dedicated software engineering roles for major research software projects at national laboratories
or other non-educational institutions. This solution is in place and working well for many well-used scientific codebases.
However, this strategy has limited application, as much of the body of software is created and maintained in research
universities. Therefore, we argue that research institutions should develop hybrid academic-technical tracks for this
capacity, where employees in this track work with more than one PI, rather than the traditional RA role within a single group.
This could be coordinated centrally, as a core facility, perhaps within research computing organizations which have
traditionally supported university cyberinfrastructure, library organizations, or research offices. Alternatively, these
groups could be organizationally closer to research groups, sitting within academic departments. The most effective model
will vary from institution to institution, but the mandate and ways of working should be similar.

Having convinced ourselves that this would be a positive innovation, we were then faced with the specific question of how to
fund the initiation of this activity. A self-sustaining research software group will support itself through collaborations
with PIs in the normal grant process, with PIs choosing to fund some amount of research software engineering effort through grants in
the usual way. However, to bootstrap such a function to a level where it has sufficient reputation and client base to be self
-sustaining will generally require seed investment.

This might come from universities themselves (this was the model that led to the creation of the group in University College
London), but more likely, seed funding needs to come from research councils (as with the Research Software
Engineering Fellowship provided by the UK Engineering and Physical Sciences Research Council). We therefore recommend that
funding organizations consider how they might provide such seed funding.

Success, appropriately measured, will help make the case to such funding bodies for further investment. One might expect that metrics such as improved productivity, software adoption rates, and grant success rates would be sufficient arguments in favor of such a model. However, useful measurement of code cleanliness, and the resulting productivity gains, is an unsolved problem in empirical software engineering. To measure ``what did not go wrong'' because of an intervention is particularly hard. 

We finally noted that the institutional case for such groups is made easier by having successful examples to point to. In the
UK, a collective effort to identify the research software engineering community, with individuals clearly stating ``I'm a research software engineer,'' has been important to the campaign. It will be useful to the global effort to similarly identify emerging research software organizations, and also, importantly, to identify longer-running research software groups, which have in some cases had a long running \emph{sui-generis} existence, but which now can be identified as part of a wider solution. There remains the problem of how to "sell" the value of this investment to investigators within the university. This is an issue best addressed by the individual organizations that embark on the plan. 

For more details on the discussion, see Appendix~\ref{sec:appendix_funding_spec_expert}.

\subsubsection{Plans}

The first step in moving this strategy forward is to gather a list of groups that self-identify as research software engineering groups, and to reach out to other organizations to see if there may be a widespread community of RSEs who do not identify themselves as such at this time. We will collect information as to the organizational models under which these groups function, and how they are funded. For example, how many research universities currently fund people in the RSE track, whether they bear the the RSE moniker or not. Are these developers paid by the University or through a program supported by research grants/individual PIs? How did they bootstrap the developer track to get this started? How successful is the university in getting investigators to pay for fractional RSEs?  We will author a report describing our findings, should funding be available to conduct the investigation. 

\subsubsection{Landing Page}

To find more information about this group, or join it, see \url{http://www.rse.ac.uk/groups}.
