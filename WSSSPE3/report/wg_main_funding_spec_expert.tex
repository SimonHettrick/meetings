\subsection{Research Programming Expertise -- how to fund it?}
\label{RSE}

\subsubsection{Why it is important}

Research Software Engineers -- those who contribute to science and scholarship
through computer programming -- are an important part of the team
needed to deliver 21st century research. However, existing academic structures
and systems of funding do not effectively fund and sustain these skills.
The resulting high levels of turnover, and inappropriate incentivisation,
are a significant contributing factor to low levels of reliability and
readability observed in scientific software.

For effective development of software for advanced research, it is necessary
that researchers can work with scientific programmers who are able to understand
the research domain sufficiently to be able to build meaningful software. This
requires a collaborative approach -- programmers fully engaged in the research
context, co-developing software with domain academics.

\subsubsection{Fit with related activities}

One solution to this problem would be a successful solution to the problems of
obtaining academic credit for programming work. If this were solved, traditional
academic funding models and career tracks would properly sustain this
contribution to research. However, in the absence of guaranteed success in that
stream, this working party seeks to look to an alternate solution; the
development and funding of dedicated research programming roles. We also noted
that measures of impact and success for research programming
groups, as well as for individual research software engineers, will be required
in order to make the case for continued funding. Research programmers will not
be measured by publications, we hope, but by other measures. Middle-author
publications are common for RSEs. Most RSEs welcome co-authorship on papers
where the PIs consider the contribution deserves it.

A contributing factor to the problem is the devastating effect on an academic
publication record resulting from time in industry; such postings often
develop exactly the skills that research programmers need, yet returns to
university positions following an industry role are penalised by the current
structures. Retention of senior programmers is hard, because these people
are highly in demand by the economy. However,
people who have a PhD in science and enter industry,
may desire to return for diverse reasons, and should be welcomed back.

\subsubsection{Discussion}

It is hard for an individual PI in a University or College
to support dedicated research programming
resources, as the need for, and funding for, these activities is intermittent
within the research cycle. To sustain this capacity, therefore, it is
necessary to aggregate this work across multiple research groups.

One solution is to fund dedicated programming roles for major research software
projects at national laboratories or other non-educational institutions. This
solution is working well for many well-used scientific codebases.
However, much of the body of software for which this expertise is needed sits
within the longer tail of research in universities. Therefore, we argue that
institutions need to develop hybrid academic-technical roles for this capacity
who work with more than one PI, rather than the traditional RA role within a
single group. This could be centrally, perhaps within research computing
organisations which have traditionally supported university cyberinfrastructure,
library organisations, or research offices. Alternatively, these groups could
be organisationally closer to research groups, sitting within academic
departments. The most
effective model will vary from institution to institution,
but the mandate and ways of working should be similar.

Having convinced ourselves that this would be a positive innovation, we were
then faced with the specific question of how to fund the initiation of this
activity. A self-sustaining research software group will support itself
through collaborations with PIs in the normal grant process, with PIs choosing
to fund some amount of research programming effort through grants in the usual
way. However, to bootstrap such a function to a level where it has the
reputation and client base to be self-sustaining requires seed investment.

This might come from unversities themselves (this was the model that led to the
creation of the group in University College London), but more likely, seed
funding needs to come from research councils (as with the Research Software
Engineering Fellowship provided by the UK Engineering and Physical Sciences
Research Council). We therefore recommend that funding organisations consider
how they might provide such seed funding.

Success, appropriately measured, will help make the case for further investment.
However, measurement of code cleanliness, and the resulting productivity gains,
is an unsolved problem in empirical software engineering. To measure
``what did not go wrong'' because of an intervention is particularly hard.

We finally noted that the institutional case for such groups is made easier
by having successful examples to point to. In the UK, a collective effort to
identify this community, with individuals clearly stating ``I'm a research
software engineer'', has been important to the campaign. It will be useful to
the global effort to similarly identify emerging research software organisations,
and also, importantly, to identify longer-running research software groups,
which have in some cases had a long running \emph{sui-generis} existence, but which
now can be identified as part of a wider solution.

\subsubsection{Plans}

We will therefore seek to gather a list of groups which are prepared to
self-identify as research software engineering groups. We will collect information
as to the organisational models under which these groups function, and how they
are funded.

\subsubsection{Landing Page}

http://www.rse.ac.uk/groups
