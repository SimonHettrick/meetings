%%%%%%%%%%%%%%%%%%%%%%%%%%%%%%%%%%%%%%%%%%%%%%%%%%%%%%%%%%%%
\subsection{Software Credit Working Group}
\label{sec:software-credit}
%%%%%%%%%%%%%%%%%%%%%%%%%%%%%%%%%%%%%%%%%%%%%%%%%%%%%%%%%%%%

%\subsubsection{Why it is important}

Modern scientific and engineering research often relies considerably on software, 
but currently no standard mechanism exists for citing software or receiving 
credit for developing software akin to receiving credit via citations for 
writing papers. Ensuring that developers of such scientific software receive 
credit for their efforts will encourage additional creation and maintenance. 
Standardizing software citations offers one route to establishing such a
citation and credit mechanism. Software is currently eligible for DOI
assignment, but DOI metadata fields are not well tuned for software compared to
publications. Some software providers apply for DOIs but it is still not widely
adopted. Also, there is no mechanism to cite software dependencies within
software in the same way papers cite supporting prior work.

\subsubsection{Fit with related activities}

Publishing Software Working Group (\S\ref{sec:publishing-software}): publishing
a software paper offers one existing mechanism for receiving credit, and further
developing new publishing concepts for software will strengthen our activities.

A number of groups external to WSSSPE (although with some overlapping members)
are also focused on aspects of software credit, including the FORCE11 Software
Citation Working Group (see plans for coordination below). In addition, a
Software Credit workshop\footnote{London Software Credit workshop:
\url{http://www.software.ac.uk/software-credit}} convened in London on October
19, following the conclusion of WSSSPE3. See
Appendix~\ref{sec:appendix_SW_credit} for more detailed discussion of related
activities.

\subsubsection{Discussion}

The group discussed a number of topics related to software credit, including a
contributorship taxonomy, software citation metadata, standards for citing
software in publications, and increasing the value of software in academic
promotion and tenure reviews. Although initial discussions both prior to and
during WSSSPE3 focused on contribution taxonomy and dividing credit, discussing
as an example the Entertainment Identifier Registry~\cite{EIDR} used in the
entertainment industry, the group decided to prioritize software citation. This
decision was motivated by the idea that standardizing citations for software
would introduce some initial credit for developers, and later the quantification
of credit could be refined based on concepts such as transitive
credit~\cite{wssspe2_katz,Katz:2014_tc}.

The majority of the remaining discussion focused on standardizing (1) the
metadata necessary for software to be cited and (2) the mechanism for citing
software in publications. Moreover, discussions also oriented around the
indexing of software citations necessary for establishing a software citation
network either integrated with the existing paper citation ecosystem or
complementary to it. See Appendix~\ref{sec:appendix_SW_credit} for a more
detailed summary of the working group's discussion on these topics.

\subsubsection{Plans}

The group already merged with the FORCE11 Software Citation Working Group
(SCWG), and their efforts will focus (over the next six to nine months) on
developing a document describing principles for software citation. Following the
publication of that document, the group will focus on outreach to key
groups (e.g., journals, publishers, indexers, professional societies).
Longer-term plans include working with indexers to ensure that software
citations are indexed and pursuing an open\slash community indexer; these
activities may be organized by future FORCE11 working groups.

\subsubsection{Landing Page}

Since near-term efforts will be shifting to the FORCE11-SCWG, we direct
interested readers to that group's existing landing page\footnote{FORCE11-SCWG
landing page,
\url{https://www.force11.org/group/software-citation-working-group}} and GitHub
repository\footnote{FORCE11-SCWG GitHub page,
\url{https://github.com/force11/force11-scwg}}.
