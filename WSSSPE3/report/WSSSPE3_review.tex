\documentclass[11pt, oneside]{amsart}
\pdfoutput=1

\usepackage{amsmath}
\usepackage{amssymb}

\usepackage{color}
\usepackage{dcolumn}
\usepackage{float}
\usepackage{graphicx}
\usepackage[utf8]{inputenc}
\usepackage[T1]{fontenc}
\usepackage{lmodern}
\usepackage{multirow}
\usepackage{rotating}
\usepackage{subfigure}
\usepackage{psfrag}
\usepackage{tabularx}
\usepackage[hyphens]{url}
\usepackage{wrapfig}
\usepackage{longtable}
\usepackage{verbatim}
\usepackage{booktabs,multicol}

% The following three lines are used for displaying footnote in tables.
\usepackage{footnote}
\makesavenoteenv{tabular}
\makesavenoteenv{table}


\usepackage{enumitem}
\setlist{leftmargin=7mm}

%\setcounter{secnumdepth}{3}
%\setcounter{tocdepth}{3}


\usepackage[bookmarks, bookmarksopen, bookmarksnumbered]{hyperref}
\usepackage[all]{hypcap}
\urlstyle{rm}

\definecolor{orange}{rgb}{1.0,0.3,0.0}
\definecolor{violet}{rgb}{0.75,0,1}
\definecolor{darkgreen}{rgb}{0,0.6,0}
\definecolor{cyan}{rgb}{0.2,0.7,0.7}
\definecolor{blueish}{rgb}{0.2,0.2,0.8}

\newcommand{\todo}[1]{{\color{blue}$\blacksquare$~\textsf{[TODO: #1]}}}
\newcommand{\note}[1]{ {\textcolor{blueish}    { ***Note:      #1 }}}
\newcommand{\katznote}[1]{ {\textcolor{magenta}    { ***Dan:      #1 }}}
\newcommand{\clunenote}[1]{ {\textcolor{orange}    { ***Tom:      #1 }}}
\newcommand{\gabnote}[1]{ {\textcolor{cyan}    { ***Gabrielle:     #1 }}}
\newcommand{\nchnote}[1]{  {\textcolor{orange}      { ***Neil: #1 }}}
\newcommand{\manishnote}[1]{  {\textcolor{violet}     { ***Manish: #1 }}}
\newcommand{\davidnote}[1]{  {\textcolor{darkgreen}      { ***David: #1 }}}
\newcommand{\colinnote}[1]{ {\textcolor{red}    {***Colin: #1 }}}
\newcommand{\choinote}[1]{ {\textcolor{orange}    {***Choi: #1 }}}

% Don't use tt font for urls
\urlstyle{rm}

% 15 characters / 2.5 cm => 100 characters / line
% Using 11 pt => 94 characters / line
\setlength{\paperwidth}{216 mm}
% 6 lines / 2.5 cm => 55 lines / page
% Using 11pt => 48 lines / pages
\setlength{\paperheight}{279 mm}
\usepackage[top=2.5cm, bottom=2.5cm, left=2.5cm, right=2.5cm]{geometry}
% You can use a baselinestretch of down to 0.9
\renewcommand{\baselinestretch}{0.96}

\sloppypar

\begin{document}

\title[]{Report on the Third Workshop on Sustainable Software for Science: Practice and Experiences (WSSSPE3)}

\author{tbd by writing and organizing}

%\author{Daniel S. Katz$^{(1)}$, Sou-Cheng T. Choi$^{(2)}$, Nancy Wilkins-Diehr$^{(3)}$, Neil Chue Hong$^{(4)}$,
%\\Colin C. Venters$^{(5)}$, James Howison$^{(6)}$, Frank Seinstra$^{(7)}$, Matthew Jones$^{(8)}$,
%\\Karen Cranston$^{(9)}$, Thomas L. Clune$^{(10)}$, Miguel de Val-Borro$^{(11)}$, Richard Littauer$^{(12)}$}
%%
%\thanks{{}$^{(1)}$ Computation Institute, 
%University of Chicago \& Argonne National Laboratory, Chicago, IL, USA; \url{dsk@uchicago.edu}}
%%
%\thanks{{}$^{(2)}$ NORC at the University of Chicago and Illinois Institute of Technology, Chicago, IL, USA; \url{sctchoi@uchicago.edu}}
%%
%\thanks{{}$^{(3)}$ University of California-San Diego, San Diego, CA, USA; \url{wilkinsn@sdsc.edu}}
%%
%\thanks{{}$^{(4)}$ Software Sustainability Institute, 
%University of Edinburgh, Edinburgh, UK; \url{N.ChueHong@software.ac.uk}}
%%
%\thanks{{}$^{(5)}$ University of Huddersfield, School of Computing and Engineering, Huddersfield, UK; \url{C.Venters@hud.ac.uk}}
%%
%\thanks{{}$^{(6)}$ University of Texas at Austin, Austin, TX, USA; \url{jhowison@ischool.utexas.edu}}
%%
%\thanks{{}$^{(7)}$ Netherlands eScience Centre, Amsterdam, Netherlands; \url{F.Seinstra@esciencecenter.nl}}
%%
%\thanks{{}$^{(8)}$ National Center for Ecological Analysis and Synthesis, Santa Barbara, CA, USA; \url{jones@nceas.ucsb.edu}}
%%
%\thanks{{}$^{(9)}$ National Evolutionary Synthesis Center, Durham, NC, USA; \url{karen.cranston@nescent.org}}
%%
%\thanks{{}$^{(10)}$ NASA Goddard Space Flight Center, Greenbelt, MD, USA; \url{Thomas.L.Clune@nasa.gov}}
%%
%\thanks{{}$^{(11)}$ Department of Astrophysical Sciences, 
%Princeton University, Princeton, NJ, USA; \url{mdevalbo@astro.princeton.edu}}
%%
%\thanks{{}$^{(12)}$ University of Saarland, Germany; \url{richard.littauer@gmail.com}}
%%
 

\begin{abstract}
\todo{need to update this abstract for WSSSPE3}
This technical report records and discusses the Second Workshop on Sustainable
Software for Science: Practice and Experiences (WSSSPE2). 
%The workshop used an
%alternative submission and peer-review process, which led to a set of papers
%divided across five topic areas: 
The report includes a description of the alternative, experimental submission
and review process, two workshop keynote presentations, a series of lightning
talks, a discussion on sustainability, and five discussions from the topic areas
of exploring sustainability; software development experiences; credit \&
incentives; reproducibility \& reuse \& sharing; and code testing \& code
review. For each topic, the report includes a list of tangible actions that were
proposed and that would lead to potential change.
%
The workshop recognized that reliance on scientific software
is pervasive in all areas of world-leading research today. The workshop
participants then proceeded to explore different perspectives on the concept of
sustainability. Key enablers and barriers of sustainable scientific software
were identified from their experiences. In addition,
recommendations with new requirements such as software credit files and software
prize frameworks were outlined for improving practices in sustainable software
engineering.
%
There was also broad consensus that formal
training in software development or engineering was rare among the
practitioners. Significant strides need to be made in building a sense of
community via training in software and technical practices, on increasing their
size and scope, and on better integrating them directly into graduate education
programs.
%
Finally, journals can define and publish policies to improve reproducibility, whereas
reviewers can insist that authors provide sufficient information and access to
data and software to allow them reproduce the results in the paper. Hence a list of
criteria is compiled for  journals to provide to reviewers so as to make it easier to
review software submitted for publication as a ``Software Paper.''

\end{abstract}


\maketitle
%\newpage

%%%%%%%%%%%%%%%%%%%%%%%%%%%%%%%%%%%%%%%%%%%%%%%%%%%%%%%%%%%%
\section{Introduction} \label{sec:intro}
%%%%%%%%%%%%%%%%%%%%%%%%%%%%%%%%%%%%%%%%%%%%%%%%%%%%%%%%%%%%

%\katznote{example comment by Dan}
%
%\gabnote{example comment by Gabrielle}
%
%\nchnote{example comment by Neil}
%
%\manishnote{example comment by Manish}
%
%\davidnote{example comment by David}

The Third Workshop on Sustainable Software for Science: Practice and Experiences
(WSSSPE3)\footnote{\url{http://wssspe.researchcomputing.org.uk/wssspe3/}} was
held on 28--29 September 2015 in Boulder, Colorado, USA. Previous events in the
WSSSPE series are
WSSSPE1\footnote{\url{http://wssspe.researchcomputing.org.uk/wssspe1/}}~\cite{WSSSPE1-pre-report,WSSSPE1},
held in conjunction with SC13,
WSSSPE1.1\footnote{\url{http://wssspe.researchcomputing.org.uk/wssspe1-1/}}, a
focused workshop organized in July 2014 jointly with the SciPy
conference\footnote{\url{https://conference.scipy.org/scipy2014/participate/wssspe/}},
WSSSPE2\footnote{\url{http://wssspe.researchcomputing.org.uk/wssspe2/}}~\cite{WSSSPE2-pre-report,WSSSPE2},
held in conjunction with SC14, and
WSSSPE2.1\footnote{\url{http://wssspe.researchcomputing.org.uk/wssspe2-1/}}, a
focused workshop organized in July 2015 again jointly with
SciPy\footnote{\url{http://scipy2015.scipy.org/ehome/115969/286469/}}.

Progress in scientific research is dependent on the quality and accessibility of
software at all levels. Hence it is critical to address challenges related to
the development, deployment, maintenance, and overall sustainability of reusable
software as well as education around software practices. These challenges can be
technological, policy based, organizational, and educational, and are of
interest to developers (the software community), users (science disciplines),
software-engineering researchers, and researchers studying the conduct of
science (science of team science, science of organizations, science of science
and innovation policy, and social science communities). The WSSSPE1 workshop
engaged the broad scientific community to identify challenges and best practices
in areas of interest to creating sustainable scientific software. WSSSPE2
invited the community to propose and discuss specific mechanisms to move towards
an imagined future practice for software development and usage in science and
engineering, but WSSSPE2 didn't have a good way to enact those mechanisms, or to
encourage the attendees to follow through on their intentions.

The workshop included multiple mechanisms for participation and encouraged team
building around solutions. It strongly encouraged participation of early-career
scientists, postdoctoral researchers, and graduate students, with funds provided
to the conference organizers by the Moore Foundation, the National Science
Foundation (NSF), and the Software Sustainability Institute, to support the
travel of potential participants who would not otherwise be able to attend,
younger participants, and those from underrepresented groups. These funds allowed
16 additional participants to attend.

This report is based on collaborative notes taken with during the workshop,
which were linked from the GitHub issues that represented the potential and
actual working
groups\footnote{\url{https://github.com/issues?q=label\%3A\%22WSSSPE3+activity\%22}}.
Overall, the report discusses the organization work done before the workshop
(\S\ref{sec:preworkshop}); the keynotes (\S\ref{sec:keynote}); a series of
lightning talks (\S\ref{sec:lightning}), intended to give an opportunity for
attendees to quickly highlight an important issue or a potential solution. The
report also gives summaries of action plans proposed by the eleven working
groups (\S\ref{sec:WGs}), and some conclusions (\S\ref{sec:conclusions}).
Finally, the report includes a list of registered attendees
(Appendix~\ref{sec:attendees}) and longer descriptions of the activities that
occurred in each the working groups
(Appendices~\ref{sec:appendix_best_practices}--\ref{sec:appendix_user_community}.)



%%%%%%%%%%%%%%%%%%%%%%%%%%%%%%%%%%%%%%%%%%%%%%%%%%%%%%%%%%%%
\section{Calls for Participation} \label{sec:preworkshop}
%%%%%%%%%%%%%%%%%%%%%%%%%%%%%%%%%%%%%%%%%%%%%%%%%%%%%%%%%%%%

WSSSPE3 was based on the work done in WSSSPE1 and WSSSPE2, but aimed at starting
a process to make progress in sustainable software, as the calls for
participation said:

\begin{quote} The WSSSPE1 workshop engaged the broad scientific community to
identify challenges and best practices in areas relevant to sustainable
scientific software. WSSSPE2 invited the community to propose and discuss
specific mechanisms to move towards an imagined future practice of software
development and usage in science and engineering. WSSSPE3 will organize
self-directed teams that will collaborate prior to and during the workshop to
create vision documents, proposals, papers, and action plans that will help the
scientific software community produce software that is more sustainable,
including developing sustainable career paths for community members. These teams
are intended to lead into working groups that will be active after the workshop,
if appropriate, working collaboratively to achieve their goals, and seeking
funding to do so if needed. \end{quote}

The first call for participation requested lightning talks, where authors could
make a brief statement about work that either had been done or was needed, with
the goal of contributing to the discussion of one or more working groups. There
was 24 lightning talks submitted, and, after a peer-review process, 16 of these
were accepted, as discussed further in Section~\ref{sec:lightning}.

The first call also discussed the potential action topics that came out of
WSSSPE2, and requested additional suggestions. The combination of existing and
new topics led to the following 18 potential topics that were advertised in the
call for participation:


\begin{quote}
\begin{itemize}
\renewcommand{\labelenumi}{\textbf{\theenumi}.}
\setlength{\rightmargin}{1em}

\item Development and Community
\begin{itemize}
\item Writing a white paper/review paper about best practices in developing
sustainable software
\item Documenting successful models for funding specialist expertise in software
collaborations
\item Creating and curating catalogs for software tools that aid sustainability
(perhaps categorized by domain, programming languages, architectures, and/or
functions, e.g., for code testing, documentation)
\item Documenting case studies for academia/industry interaction
\item Determining effective strategies for refactoring/improving legacy
scientific software
\item Determining principles for engineering design for sustainable software
\item Create a set of guidance giving examples of specific metrics for the
success of scientific software in use, why they were chosen, what they are
useful to measure, and any challenges/pitfalls; then publish this as a white
paper
\end{itemize}

\item Training
\begin{itemize}
\item Writing a white paper on training for developing sustainable software, and
coordinating multiple ongoing training-oriented projects
\item Developing curriculum for software sustainability, and ideas about where
such curriculum would be presented, such as a summer training institute
\end{itemize}

\item Credit
\begin{itemize}
\item Hacking the credit and citation ecosystem (making it work, or work better,
for software)
\item Developing a taxonomy of contributorship/guidelines for including software
contributions in tenure review
\item Documenting case studies of receiving credit for software contributions
\item Developing a system of awards and recognitions to encourage sustainable software
\end{itemize}

\item Publishing
\begin{itemize}
\item Developing a categorization of journals that publish software papers
(building on existing work), and case studies of alternative publishing
mechanisms that have been shown to improve software discoverability/reuse e.g.,
popular blogs/websites
\item Determining what journals that publish software paper should provide to
their reviewers (e.g., guidelines, mechanisms, metadata standards, etc.)
\end{itemize}

\item Reproducibility and Testing
\begin{itemize}
\item Building a toolkit that could allow conference organizers to easily add a
reproducibility track
\item Documenting best practices for code testing and code review
\end{itemize}

\item Documentation
\begin{itemize}
\item Develop landing pages on the WSSSPE website (or elsewhere) that enable the
community to easily find up-to-date information on a WSSSPE topic (e.g.,
software credit, scientific software metrics, testing scientific software)
\end{itemize}

\end{itemize}
\end{quote}

%%%%%%%%%%%%%%%%%%%%%%%%%%%%%%%%%%%%%%%%%%%%%%%%%%%%%%%%%%%%
\section{Keynote} \label{sec:keynote} 
%%%%%%%%%%%%%%%%%%%%%%%%%%%%%%%%%%%%%%%%%%%%%%%%%%%%%%%%%%%%
\todo{need to update for WSSSPE3. 
Lead: S.-C. Choi. 
PDF slides at \url{http://tinyurl.com/qbbqgsj}.
Youtube Video at \url{http://tinyurl.com/q45kfcn}.
}

WSSSPE3 began with a keynote speech delivered by Professor Matthew Turk from the
Department of Astronomy, University of Illinois titled \emph{Why Sustain
Scientific Software?}. Turk is a prolific practitioner of scientific software
and has extensive experiences working on large collaborative projects employing
modern computing tools. He is also a co-organizer and champion of all the WSSSPE
events so far.

In his keynote address, Turk recapped the course of development of WSSSPE
workshops alongside with his career from a postdoc to an academic over the past
few years. The first WSSSPE workshop was at the Supercomputing conference (SC13)
in 2013, but he observed that the notion of sustainable scientific software drew
in an audience beyond supercomputing. In the following year, WSSSPE1.1 at SciPy
had speakers talking about how software has been sustained inside the scientific
Python community. WSSSPE2 at SC14 had breakout group discussions coming up with
actionable items, and WSSSPE2.1 at SciPy 2015 was similar. Turk noted the
different atmosphere of the surrounding large conferences, despite similar
WSSSPE participants.

WSSSPE3 broke free of the traditional Supercomputing Conference environment this
year, and in Turk's words, this spoke to the fact that scientific software comes
from many different types of inquiries, deployment, strategies for maintenance,
users; and ways of measuring the value of a piece of software. It appeared to
Turk that the supercomputing conferences generally adopt some top-down
approaches, whereas SciPy more often than not uses more bottom-up systems. The
essential messages perceived were also often bipolar: software getting harder in
supercomputing communities with exascale computing and optimization issues in
mind, but software becoming better in the SciPy group with emerging tools such
as Jupiter and productivity packages for research workflow. Admitting such
comparisons are somewhat unfair generalizations, Turk reminded the audience that
the different approaches bring different types of ideas to the table, but he
welcomed WSSSPE3 being conducted outside existing preconceptions.

Returning to the topic of his talk, Turk invited the audience to picture
scientific software as a flower on a landscape under the Sun, which may
represent a number of measurable factors such as number of citations; growth of
a community and number of contributors; amount of funding; prestigious prizes
awarded; stability of the community in terms of leadership transitions, serving
community needs, not breaking test suites, and performance on new architectures.
But all these metrics are strictly speaking \emph{proxies} for the values and
the impact scientific software bears. What we can measure does not give us
direct insight---it just gives us proxies of insight.
  
Turk then moved onto various different definitions of sustainability. His
favorite one was keeping up with bug reports, in which case even if no new
features were added, the software remains sustainable. Another definition of
sustainability Turk mentioned was adding of new features, or maintaining the
software for a long period of time such as the cases of TeX or LaTeX with
community help. A notion Turk heard often at supercomputing conferences was that
sustainable software continued to work on new architectures. Yet another metric
was people continuing to be able to learn how to use and apply the software. A
funder Turk heard talked about sustainability as continuing to get funded. Turk
also recalled that Greg Wilson, among others, said in WSSSPE1.1 that his view of
sustainable software was software that continued to give the same results over
time. A last measure of sustainability Turk presented was the ability to
transition between different people developing and using a piece of software.

At WSSSPE1, several models were presented for ensuring sustainability. Turk
considered that a familiar one was a funded piece of software where an external
agency provided funds to a group that was not necessarily exclusively working on
and developing a piece of software, keeps it going, and provides that out to the
scientific community. The model of productized software, in which a piece of
software has grown into something that individuals rely on to the point that
research groups or people are willing to support it at some pay rate, for
instance, subscription to use cloud services that deploy a piece of software, or
purchase of a piece of software. A final model Turk felt conflicted about is a
volunteer model that is traditional old-school---not modern-day open
source---developement.

Turk discussed whether productizing scientific software was synonoymous with
being sustsainable and self-sufficient. He thought it was not necessarily the
case and furthermore, it could lead to a divergence of interest between users
and developers. 

The volunteer model meant unpaid labor. On this note, Turk recommended Ashe
Dryden's blog
post\footnote{http://www.ashedryden.com/blog/the-ethics-of-unpaid-labor-and-the-oss-community}
on the ethics of unpaid labor and the open source software community, which
mentions GitHub as one's resume, for example. Often times, a person funded to
work on some scientific project full time has a small amount of time for working
on a piece of software necessary for the project. However, researchers's ability
to participate in that volunteer community is not always the same and it is not
always aligned with their research projects. From Turk's experience, we cannot
always rely on unpaid labor and volunteer time to sustain a piece of software---
that came down to the notions of the top-down and the bottom-up approaches, the
funded versus the grassroot.

%%%%%%%%%%%%%%%%%%%%%%%%%%%%%%%%%%%%%%%%%%%%%%%%%%%%%%%%%%%%
\section{Lightning Talks} \label{sec:lightning}
%%%%%%%%%%%%%%%%%%%%%%%%%%%%%%%%%%%%%%%%%%%%%%%%%%%%%%%%%%%%
\todo{need to update for WSSSPE3}
\begin{comment}
\note{
\href{http://wssspe.researchcomputing.org.uk/wssspe3/agenda/}{Slides.}}
\end{comment}

\begin{enumerate}
\item \textbf{Benjamin Tovar and Douglas Thain: \textit{Freedom vs. Stability:
Facilitating Research Training While Supporting Scientific Research}}

\item \textbf{Birgit Penzenstadler, Colin Venters, Christoph Becker, Stefanie
Betz, Ruzanna Chitchyan, Letícia Duboc, Steve Easterbrook, Guillermo
Rodriguez-Navas and Norbert Seyff: \textit{Manifesting the Ghost of the Future:
Sustainability}}

\item \textbf{Abani Patra, Hossein Aghakhani, Nikolay Simakov, Matthew D. Jones
and Tevfik Kosar: \textit{Integrating New Functionality Using Smart Interfaces to
Improve Productivity of Legacy Tools}}

\item \textbf{Abigail Cabunoc Mayes, Bill Mills, Arliss Collins and Kaitlin
Thaney: \textit{Collaborative Software Development as Sustainable Software: Lessons
from Open Source}}

\item \textbf{Louise Kellogg and Lorraine Hwang: \textit{Advancing Earth Science
through Best Practices in Open Source Software: Computational Infrastructure
for Geodynamics}}

\item \textbf{Lorraine Hwang, Joe Dumit, Alison Fish, Louise Kellogg, Mackenzie
Smith and Laura Soito: \textit{Software Attribution for Geoscience Applications in the
Computational Infrastructure for Geodynamics}}

\item \textbf{Mike Hildreth, Jarek Nabrzyski, Da Huo, Peter Ivie, Haiyan Meng,
Douglas Thain and Charles Vardeman: \textit{Data And Software Preservation for Open
Science (DASPOS)}}

\item \textbf{James Hetherington, Jonathan Cooper, Robert Haines, Simon
Hettrick, James Spencer, Mark Stillwell, Mike Croucher, Christopher Woods and
Susheel Varma: \textit{Research Software Engineering Groups in Universities: The Story
from the UK}}

\item \textbf{Dan Gunter, Sarah Poon and Lavanya Ramakrishnan: \textit{Bringing the
User into Building Sustainable Software for Science}}

\item \textbf{Dan Gunter, Adam Arkin, Rick Stevens, Robert Cottingham and
Sergei Maslov: \textit{Challenges of a Sustainable Software Platform for Predictive
Biology: Lessons Learned on the KBase Project}}

\item \textbf{Yolanda Gil, Chris Duffy, Chris Mattmann, Erin Robinson and Karan
Venayagamoorthy: \textit{The Geoscience Paper of the Future Initiative: Training
Scientists in Best Practices of Software Sharing}}

\item \textbf{Neil Chue Hong: \textit{Building a Scientific Software Accreditation
Framework}}

\item \textbf{Jeffrey Carver: \textit{On the Need for Software Engineering Support for
Sustainable Scientic Software}}

\item \textbf{Matthias Bussonnier: \textit{User Data Collection in Open Source}}

\item \textbf{Alice Allen: \textit{We’re giving away the store! (Merchandise not
included)}}

\item \textbf{Stan Ahalt, Bruce Berriman, Maxine Brown, Jeffrey Carver, Neil
Chue Hong, Allison Fish, Ray Idaszak, Greg Newman, Dhabaleswar Panda, Abani
Patra, Elbridge Gerry Puckett, Chris Roland, Douglas Thain, Selcuk Uluagac, and
Bo Zhang: \textit{Scientific Software Success: Developing Metrics While Developing
Community}}

\end{enumerate}

%%%%%%%%%%%%%%%%%%%%%%%%%%%%%%%%%%%%%%%%%%%%%%%%%%%%%%%%%%%%
\section{Working Groups} \label{sec:WGs}
%%%%%%%%%%%%%%%%%%%%%%%%%%%%%%%%%%%%%%%%%%%%%%%%%%%%%%%%%%%%

\subsection{White paper/journal paper about best practices in developing sustainable software}
\label{sec:best-practices}

%\subsubsection{Why it is important}
Reviewing multiple past articles and talks at different meetings like WSSSPEx \todo{add cites to WSSSPE papers, heroux paper, ...} 
and 
analyzing and promoting sustainable scientific software makes it clear that there 
are several common and recurring ideas that underpin success in developing sustainable software. However, outside of a small community, this knowledge is not widely
shared. This is  especially true for the large community of scientists who generate most of the software used by scientists but are not primarily software developers. In this
scenario, a clear and precise exposition of these best practices collected from many sources and open collaboration among all in the community
in a single source (e.g., journal paper, tutorial)  that can be widely disseminated is necessary and likely to be very valuable.

\subsubsection{Fit with related activities}

The creation of such a ``best practices'' document will build upon the range of activities and topics discussed at WSSSPE3 and associated prior meetings. We will attempt to
distill the emerging body of knowledge into this document. The large number of  articles from the NSF funded SI2 projects (SSE and SSI), ``lightning talks'', ``white papers,'' and reports from different workshops have created a large if somewhat diffuse source for this report.

\subsubsection{Discussion}

Core questions that will need to be explored are in reliability (reproducibility), usability, extensibility, knowledge management and continuity (transitions between people). Answers to these will guide us on how a software tool becomes part of the core workflow of well identified users (stakeholders) relating to tool success and hence sustainability. 
%\katznote{prev sentence is complex and awkward} % see fix
Ideas  that may need to be explored include:
\begin{itemize}
\item Requirements engineering to create tools with immediate uptake;
\item When should software ``die''?
\item Catering to disruptive developments in environment (e.g., new hardware, new methodology);
\item Dimensions of sustainability -- economic, technical, environmental and
obsolescence. 
%\katznote{not sure what the prev. comment goes with} %social.
\end{itemize}

Sustainability requires community participation in code development and/or a wide adoption of software.
The larger the community base is using a piece of software, the better are the funding possibilities and thus also the sustainability options.
Additionally developer commitment to an application is essential and experience shows that software packages with an evangelist imposing strong inspiration and discipline are more likely to achieve sustainability.
While a single person can push sustainability to a certain level, open source software also needs sustained commitment from the developer community.
Such sustained commitments include diverse tasks and roles, which can be fulfilled by diverse developers with different knowledge levels.
Besides developing software and appropriate software management with measures for extensibility and scalability of the software, active (expertise) support for users via a user forum with a quick turnaround is crucial.
The barrier to entry for the community as users as well as developers has to be as low as possible.

For additional information about the discussion, see Appendix~\ref{sec:appendix_best_practices}.

\subsubsection{Plans}
%\todo{short text here - not bullets}
The creation of a best-practices document needs a large and diverse community involved. We have enlisted over ten contributors from the attendees at the WSSSPE3 and 
those on the mailing list.
The primary mechanism for developing this document will be to examine and analyze the success of several well known community scientific
software and organizations supporting scientific software.
We will attempt then to abstract general principles and best practices.
Some of the tools identified for such analysis are
the general purpose PeTSC toolkit for linear system solution, NWChem for computational chemistry and the CIG (Computational Infrastructure for Geodynamics) organization dedicated to supporting an ensemble of related tools for the geodynamics community. 
We also established a timeline and a rough outline (see Appendix~\ref{sec:appendix_best_practices}) for the report.

\noindent{\bf Timeline:}
\begin{itemize}
%
\item 15 Nov: Introduction and scope finished
\item 15 Nov: Sections assigned
\item 31 Jan: Analyzing funding possibilities for survey
\item 31 Jan: First versions of section
\item 15 Feb: Distribution to WSSSPE community
\item 31 Mar: Final version of white paper
\item 30 Apr: Submission of peer-reviewed paper?
\end{itemize}


\subsubsection{Landing Page}
The landing page with instructions, timeline and the white paper is here: \url{https://drive.google.com/drive/folders/0B7KZv1TRi06fbnFkZjQ0ZEJKckk}
Discussions can be also continued in \url{https://github.com/WSSSPE/meetings/issues/42}

\subsection{Funding Research Programmer Expertise}
\label{RSE}

%\katznote{I don't think this section title is exactly right - are there some other options?  Also, the title of Appendix~\ref{sec:appendix_funding_spec_expert} should match, whatever is chosen.  Maybe `Funding Research Programmer Expertise'?}

%\subsubsection{Why it is important}

Research Software Engineers (RSEs) -- those who contribute to science and
scholarship through software development -- are an important part of the team
needed to deliver 21st century research. However, existing academic structures
and systems of funding do not effectively fund and sustain these skills. The
resulting high levels of turnover and inappropriate incentives are significant
contributing factors to low levels of reliability and readability observed in
scientific software. Moreover, the absence of skilled and experienced developers
retards progress in key projects, and at times causes important projects to fail
completely.

Effective development of software for advanced research requires that
researchers work closely with scientific software developers who understand the
research domain sufficiently to build meaningful software at a reasonable pace.
This requires a collaborative approach -- where developers who are fully engaged
or invested in the research context are co-developing software with domain
academics.

\subsubsection{Fit with related activities}

The solution we envision entails creating an environment where software
developers are a stable part of a research team. Such an environment mitigates
the risk of losing a key developer at a critical moment in a projects lifetime,
and provides the benefits of building a store of institutional knowledge about
specific projects as well as about software development for today's research.
Our vision is to find a way to promote a university/research institute
environment where software developers are stable components of research project
teams.

One strategy to promote stability is implementing a mechanism for developers to
obtain academic credit for software development work. With such a mechanism in
place, traditional academic funding models and career tracks could properly
sustain individuals for whom software development is their primary contribution
to research. A contributing factor to the problem with the current academic
reward system is the devastating effect on an academic publication record
resulting from time in industry; such postings often develop exactly the skills
that research software engineers need, yet returns to university positions
following an industry role are penalized by the current structures. Retention of
senior developers is hard, because these people are high in demand by the
economy. However, people who have a PhD in science and enter industry, may
desire to return for diverse reasons, and should be welcomed back.

While development of new mechanisms in the current academic reward system is a
worthy aspirational goal, such a dramatic change in this structure does not seem
likely in a time scale relevant to this working group. Accordingly, our working
party sought alternative solutions that may be achievable within the context of
existing academic structures. The group felt that developing dedicated research
software engineering roles within the university, and finding stable funding for
those individuals is the most promising mechanism for creating a stable software
development staff.

Measures of impact and success for research programming groups, as well as for
individual research software engineers, will be required in order to make the
case to the university for continued funding. Research software engineers will
not be measured by publications, we hope, but by other metrics. Middle-author
publications are common for RSEs. Most RSEs welcome co-authorship on papers
where the PIs think that the contribution deserves it.

\subsubsection{Discussion}

It is hard for an individual PI in a university or college to support dedicated
research software engineering resources, as the need and funding for these
activities are intermittent within a research cycle. To sustain this capacity,
therefore, it is necessary to aggregate this work across multiple research
groups.

One solution is to fund dedicated software engineering roles for major research
software projects at national laboratories or other non-educational
institutions. This solution is in place and working well for many well-used
scientific codebases. However, this strategy has limited application, as much of
the body of software is created and maintained in research universities.
Therefore, we argue that research institutions should develop hybrid
academic-technical tracks for this capacity, where employees in this track work
with more than one PI, rather than the traditional RA role within a single
group. This could be coordinated centrally, as a core facility, perhaps within
research computing organizations which have traditionally supported university
cyberinfrastructure, library organizations, or research offices. Alternatively,
these groups could be organizationally closer to research groups, sitting within
academic departments. The most effective model will vary from institution to
institution, but the mandate and ways of working should be similar.

Having convinced ourselves that this would be a positive innovation, we were
then faced with the specific question of how to fund the initiation of this
activity. A self-sustaining research software group will support itself through
collaborations with PIs in the normal grant process, with PIs choosing to fund
some amount of research software engineering effort through grants in the usual
way. However, to bootstrap such a function to a level where it has sufficient
reputation and client base to be self -sustaining will generally require seed
investment.

This might come from universities themselves (this was the model that led to the
creation of the group in University College London), but more likely, seed
funding needs to come from research councils (as with the Research Software
Engineering Fellowship provided by the UK Engineering and Physical Sciences
Research Council). We therefore recommend that funding organizations consider
how they might provide such seed funding.

Success, appropriately measured, will help make the case to such funding bodies
for further investment. One might expect that metrics such as improved
productivity, software adoption rates, and grant success rates would be
sufficient arguments in favor of such a model. However, useful measurement of
code cleanliness, and the resulting productivity gains, is an unsolved problem
in empirical software engineering. To measure ``what did not go wrong'' because
of an intervention is particularly hard.

We finally noted that the institutional case for such groups is made easier by
having successful examples to point to. In the UK, a collective effort to
identify the research software engineering community, with individuals clearly
stating ``I am a research software engineer,'' has been important to the
campaign. It will be useful to the global effort to similarly identify emerging
research software organizations, and also, importantly, to identify
longer-running research software groups, which have in some cases had a long
running \emph{sui-generis} existence, but which now can be identified as part of
a wider solution. There remains the problem of how to ``sell'' the value of this
investment to investigators within a university. This is an issue best addressed
by the individual organizations that embark on the plan.

For more details on the discussion, see
Appendix~\ref{sec:appendix_funding_spec_expert}.

\subsubsection{Plans}

The first step in moving this strategy forward is to gather a list of groups
that self-identify as research software engineering groups, and to reach out to
other organizations to see if there may be a widespread community of RSEs who do
not identify themselves as such at this time. We will collect information about
the organizational models under which these groups function, and how they are
funded. For example, how many research universities currently fund people in the
RSE track, whether they bear the RSE moniker or not. Are these developers paid
by the university or through a program supported by research grants/individual
PIs? How did they bootstrap the developer track to get this started? How
successful is the university in getting investigators to pay for fractional
RSEs? We will author a report describing our findings, should funding be
available to conduct the investigation.

\subsubsection{Landing Page}

To see our list of international research software engineering groups so far
visit \url{http://www.rse.ac.uk/international}. For the UK groups, see
\url{http://www.rse.ac.uk/groups}. Too add your group to the list, please make a pull request to
\url{https://github.com/UKRSE/UKRSE.github.io/blob/master/international.md}

%\subsection{Topic} \todo{change the title to your topic}

\subsubsection{Why it is important}
\todo{short text here}

\subsubsection{Fit with related activities}
\todo{short text here - can include links/cites}

\subsubsection{Discussion}
\todo{short-ish text here}

\subsubsection{Plans}
\todo{short text here - not bullets}

\subsubsection{Landing Page}
\todo{link to landing page}


%%%%%%%%%%%%%%%%%%%%%%%%%%%%%%%%%%%%%%%%%%%%%%%%%%%%%%%%%%%%
\subsection{Transition Pathways to Sustainable Software: Industry \& Academic Collaboration}
\label{sec:industry_interaction}
%%%%%%%%%%%%%%%%%%%%%%%%%%%%%%%%%%%%%%%%%%%%%%%%%%%%%%%%%%%%

%\subsubsection{Why it is important}

Most scientific software is produced as a part of grant-funded research projects typically 
sponsored by federal governments. If we are interested in the sustainability of
scientific software, then we need to understand what exactly happens when that
sponsorship ends. More than likely, the project and its resulting software will
need to undergo some kind of transition in funding and consequently governance.

At WSSSPE3 our group was interested in better understanding successful pathways
for scientific software to ``transition'' from grant-funded research
projects to industry sponsorship. (This may be an initially awkward
phrase---some software projects will begin their life being sponsored by
industry, or result in collaboration between industry and academia. In such
cases, there is still a need to understand how IP and how maintenance of the
software is sustained over time.)

\subsubsection{Fit with related activities}

Most previous research and discussion of industry and academic collaboration, sharing, and funding of research software has focused on the impact of such arrangements. Examples of these types of reports are as follows:

\begin{itemize}
\item REF Impact Case Studies: \url{http://impact.ref.ac.uk/CaseStudies/}
\item Background of projects funded in the UK: \url{http://gtr.rcuk.ac.uk/}
\item Dowling Review from the UK: addresses complexity of work between these two
communities: \url{http://www.raeng.org.uk/policy/dowling-review}
\item Pathway to Impact - UK report: two pages of grant proposals are asked to
forecast what impact they might have (including environmental, academic, economic).
\end{itemize}

\subsubsection{Discussion}

Although sustainability transitions are often studied under the broad umbrella of ``technology
transfer,'' we believe there are likely to be a number of different ways in which
a pathway from initial production to long-term maintenance and secure funding is
achieved. In short, industry sponsorship and/or direct participation is an important aspect of sustaining
scientific software, but our current understanding of these transitions focuses
narrowly on commercial successes or failures of those collaborations.

In looking at existing literature that addresses industry transitions, many
reports (such as those listed above) focus on benefits that accrue to the private sector, or to a government
that originally sponsored the research project. This literature does not address
the impact that these transitions have on the accessibility or usability of the
software, or the impact that these transitions have on the career of the
researchers involved.

For more detail on the group's discussion, see
Appendix~\ref{sec:appendix_industry_interaction}.

\subsubsection{Plans}

Plans for carrying forward are currently unclear---this project would require
sustained attention and effort from our team, and at least some amount of
funding in order for us to be involved for extended periods of time.

The broad goals that we would like to accomplish are: 

\begin{enumerate}
\item To complete a set of case studies which look at successful and unsuccessful transitions between academic researchers and industry
\item To create a generalizable framework, which might allow for a broader study of different transition pathways (other than between academia and industry)
\end{enumerate}

The main plan for the group going forward is the creation of a white paper on the topic of sustainability transitions. 

\subsubsection{Landing Page}
Transitions Pathways discussions can be posted at \url{https://github.com/WSSSPE/meetings/issues/46} or an email be sent to Nic Weber\footnote{email: \href{mailto:nmweber@uw.edu}{nmweber@uw.edu}} to find out more about the group's efforts and how to participate.
\subsection{Legacy Software} 

This group met only briefly, for one period on the first day. They discussed
that it is difficult to define legacy code because there is so much stigma
associated with the term. At some point there will be more difficulty and
resources wasted trying to keep legacy software supported, but it will
eventually be too expensive compared to how much it would be to just rebuild the
software or kill it. Most of the group members were not able to attend on the
second day, and those who were able to attend joined other groups.


%\subsubsection{Why it is important}
%\todo{short text here}
%
%\subsubsection{Fit with related activities}
%\todo{short text here - can include links/cites}
%
%\subsubsection{Discussion}
%\todo{short-ish text here}
%
%\subsubsection{Plans}
%\todo{short text here - not bullets}
%
%\subsubsection{Landing Page}
%\todo{link to landing page}




\subsection{Principles for Software Engineering Design for Sustainable Software} 

%\subsubsection{Why it is important}

Software engineering principles form the basis of methods, techniques, methodologies and tools~\cite{Vliet:2008}. However, there is often a mismatch between software engineering theory and practice particularly in the fields of computational science and engineering, which can lead to the development of unsustainable software~\cite{Merali:2010,hettrick:2014}. Understanding and applying software engineering principles is essential in order to create and maintain sustainable software~\cite{Becker:2016}.

\subsubsection{Fit with related activities}
The group discussion focused on identifying existing principles of software engineering design that could be adopted by the computational science and engineering communities.

\subsubsection{Discussion}

Software engineering principles form the basis of methods, techniques, methodologies and tools. This group, which included members from different backgrounds, including quantum chemistry, epidemiology, computer science, software engineering, and microscopy, discussed the principles of software engineering design for sustainable software (starting with principles from the Karlskrona Manifesto on Sustainability Design~\cite{Becker:2015}, Tate~\cite{tate:2005}, and the SoftWare Engineering Body of Knowledge (SWEBOK)~\cite{swebokv3}) and their application in various domains including quantum chemistry and epidemiology.  The group examined the principles and took a retrospective analysis of what the developers did in practice against how the principles could have made a difference, and asked, what do the principles mean for  computational scientific and engineering software, and how do the principles relate to non-functional requirements? It
appeared that the sustainable software engineering principles should be mapped to two core quality attributes that underpin technically sustainable software: extensibility, the software's ability to be extended and the level of effort required to implement the extension; and
maintainability: the effort required to locate and fix an error in operational software.

For more information about the discussion, see Appendix~\ref{sec:appendix_eng_design}.

\subsubsection{Plans}
The next steps in this endeavor are to (1) Systematically analyze a number of example systems from different scientific domains with regards to the identified principles, to (2) Identify the commonalities and gaps in applying those principles to different scientific systems, and to (3) Propose a set of guidelines on the principles and how they exemplary apply to scientific software system. Preliminary work will be carried out through undergraduate or post-graduate student projects.

\subsubsection{Landing Page}
In the absence of a landing page, the Principles for Software Engineering Design for Sustainable Software working group requests an email be sent to Birgit Penzenstadler\footnote{email: \href{mailto:birgit.penzenstadler@csulb.edu}{birgit.penzenstadler@csulb.edu}} and Colin C.\ Venters\footnote{email:\href{mailto:c.venters@hud.ac.uk}{c.venters@hud.ac.uk}} to find out more about the group's efforts and how to participate.

%%%%%%%%%%%%%%%%%%%%%%%%%%%%%%%%%%%%%%%%%%%%%%%%%%%%%%%%%%%%
\subsection{Useful Metrics for Scientific Software}
\label{sec:software-metrics}
%%%%%%%%%%%%%%%%%%%%%%%%%%%%%%%%%%%%%%%%%%%%%%%%%%%%%%%%%%%%

%\subsubsection{Why it is important}

Metrics for scientific software are important for tenure and promotion,
scientific impact, discovery, reducing duplication, serving as a basis for
potential industrial interest in adopting software, prioritizing development and
support towards strategic objectives, and making a case for new or continued
funding. However, there is no commonly-used standard for collecting or
presenting metrics, nor is it known if there is a common set of metrics for
scientific software. It is imperative that scientific software stakeholders
understand that it is useful to collect metrics.

\subsubsection{Fit with related activities}

The group discussion focused on identifying existing frameworks and activities
for scientific software metrics. The group identified the following related
activities:


\begin{itemize}

\item Computational Infrastructure for Geodynamics: Software Development Best
Practices\footnote{\url{https://geodynamics.org/cig/dev/best-practices/}}

\item WSSSPE3 Breakout Session: How can we measure the impact of a piece of code on
research, and its value to the
community?\footnote{\url{https://docs.google.com/document/d/1cgUDH3RxrfsLotWhKKOrXUnaYFhrtjcV1TDRkFtwQKI/edit}}

\item 2015 NSF SI2 PI Workshop Breakout Session on Framing Success
Metrics\footnote{\url{https://docs.google.com/document/d/10yj7MYEjvrg__t522XR41ogASYMp647-l-BpFTsqEV4/}}

\item 2015 NSF SI2 PI Workshop Breakout Session on Software
Metrics\footnote{\url{https://docs.google.com/document/d/1uDim5bw8rBuubmtaUrz5Eh35NxzDgivmmdXhVzDs3tc/edit}}

\item NSF Workshop on Software and Data Citation Breakout Group on Useful
Metrics\footnote{\url{https://docs.google.com/presentation/d/1PPLVL6uoOmisqnHTlwhsVKJBTFFK1IVzvr8FdEEIvAE/}}

\item U.K. Software Sustainability Institute Software Evaluation
Guide\footnote{\url{http://www.software.ac.uk/software-evaluation-guide}}

\item U.K. Software Sustainability Institute Blog post: The five stars of
research
software\footnote{\url{http://www.software.ac.uk/blog/2013-04-09-five-stars-research-software}}

\item Minimal information for reusable scientific
software\footnote{\url{http://figshare.com/articles/Minimal_information_for_reusable_scientific_software/1112528}}

\item EPSRC-funded Equipment Data Search
Site\footnote{\url{http://equipment.data.ac.uk/}}

\item Canarie Research Software: Software to accelerate
discovery\footnote{\url{http://www.canarie.ca/software/}}

\item Canarie Research Software: Research Software Platform
Registry\footnote{\url{https://science.canarie.ca/researchmiddleware/platforms/list/main.html}}

\item BlackDuck Open HUB\footnote{\url{https://www.openhub.net/}}
\item Innovation Policy
Platform\footnote{\url{https://www.innovationpolicyplatform.org/frontpage}}


\end{itemize}


\subsubsection{Discussion}

The group discussion began by agreeing on the common purpose of creating a set
of guidance giving examples of specific metrics for the success of scientific
software in use, why they were chosen, what they are useful to measure, and any
challenges and pitfalls; then publish this as a white paper. The group discussed
many questions related to useful metrics for scientific software including
addressing if there is a common set of metrics that can be filtered in some way,
can metrics be fit into a common template, which metrics would be the most
useful for each stakeholder, which metrics are the most helpful and how would we
assess this, how are metrics monitored, and many more. A more complete bulleted
list of these questions can be found in Appendix H. Next, a roadmap for how to
proceed was discussed including creating a set of milestones and tasks. The idea
was put forth for the group to interact with the organizing committee of the
2016 NSF Software Infrastructure for Sustained Innovation (SI2) PI workshop in
order to send a software metrics survey to all SI2 and related awardees as a
targeted and relevant set of stakeholders. The five solicitations for software
elements released under the NSF SI2 program all included metrics as a required
component with submitters requested to include {\it ``a list of tangible metrics,
with end user involvement, to be used to measure the success of the software
element developed, \dots''}. These metrics are then reported as part of annual
reports to NSF by the projects. Although neither the proposal text describing
the metrics nor the reported metric results are publicly available, there is
reason to believe that the community will be willing to provide this information
through a survey mechanism. This survey would be created by one of the student
group members. Similarly, it was suggested that a software metrics survey be
sent to the UK SFTF (Software For The Future, led by the Engineering and Physical Sciences Research Council) and TRDF (Tools and Resources Development Fund, led by the Biotechnology and Biological Sciences Research Council) software projects to ask them what metrics would be useful to report. The remainder of the discussion focused mainly on the creation
of a white paper on this topic. This resulted in a paper outline and writing
assignments with the goal of publishing in venues including WSSSPE4, IEEE CiSE (Institute of Electrical and Electronic Engineers Computing in Science and Engineering magazine), or JORS (Journal of Open Research Software). More information about the group discussion is available in Appendix~\ref{sec:appendix_metrics}

\subsubsection{Plans}

The main plan for the group going forward is the creation of a white paper on
the topic of useful metrics for scientific software. The authoring of this white
paper would happen in parallel with the creation of a survey by the group with
the survey results to be incorporated in the white paper. The timeline for
completion of the white paper is approximately one year targeting venues
discussed in the previous section.

\subsubsection{Landing Page}

In lieu of a landing page, the Useful Metrics for Scientific Software working
group requests an email be sent to Gabrielle Allen
\href{mailto:gdallen@illinois.edu}{(gdallen@illinois.edu)} to find out more
about the group's efforts and how to participate.

\subsection{Training}

%\subsubsection{Why it is important}

This group explored a rapidly growing array of training that is seen to
contribute to sustainable software. The offerings are diverse, providing
training that is more or less directly relevant to sustainable software. While
research institutions support professional development for research staff, the
skills taught which might impact on sustainable software are limited at best,
often lacking a clear and coherent development pathway. Bringing together those
involved in leading relevant initiatives on a regular basis could helpfully
coordinate this growing array of training opportunities.

\subsubsection{Fit with related activities} Three existing venues for discussion
of related activities are identified:

\begin{itemize}

\item Working towards Sustainable Software for Science: Practice and
Experiences (WSSSPE) workshops~\cite{WSSSPE}

\item International Workshop on Software Engineering for High
Performance Computing in Computational Science and
Engineering (SEHPCCSE)~\cite{SEHPCCSE}

\item Workshop on Software Engineering for Sustainable Systems~\cite{se4susy}

\end{itemize}

\subsubsection{Discussion}

Next steps have been identified to quickly test whether there is interest in
establishing a community committed to increasing the degree of coordination
across training projects. See Appendix~\ref{sec:appendix_training} for more details about the discussion.

\subsubsection{Plans}

The main plan for the group is to convene a discussion to explore bringing
together regular meetings of those involved in leading relevant training
projects.

\subsubsection{Landing Page}

The Training working group requests an email be sent to Nick Jones\footnote{email: 
\href{mailto:nick.jones@nesi.org.nz}{nick.jones@nesi.org.nz}} to find out more
about the group's efforts and how to participate.

%%%%%%%%%%%%%%%%%%%%%%%%%%%%%%%%%%%%%%%%%%%%%%%%%%%%%%%%%%%%
\subsection{Software Credit Working Group}
\label{sec:software-credit}
%%%%%%%%%%%%%%%%%%%%%%%%%%%%%%%%%%%%%%%%%%%%%%%%%%%%%%%%%%%%

%\subsubsection{Why it is important}

Modern scientific and engineering research often relies considerably on software, 
but currently no standard mechanism exists for citing software or receiving 
credit for developing software akin to receiving credit via citations for 
writing papers. Ensuring that developers of such scientific software receive 
credit for their efforts will encourage additional creation and maintenance. 
Standardizing software citations offers one route to establishing such a
citation and credit mechanism. Software is currently eligible for DOI
assignment, but DOI metadata fields are not well tuned for software compared to
publications. Some software providers apply for DOIs but it is still not widely
adopted. Also, there is no mechanism to cite software dependencies within
software in the same way papers cite supporting prior work.

\subsubsection{Fit with related activities}

Publishing Software Working Group (\S\ref{sec:publishing-software}): publishing
a software paper offers one existing mechanism for receiving credit, and further
developing new publishing concepts for software will strengthen our activities.

A number of groups external to WSSSPE (although with some overlapping members)
are also focused on aspects of software credit, including the FORCE11 Software
Citation Working Group (see plans for coordination below). In addition, a
Software Credit workshop\footnote{London Software Credit workshop:
\url{http://www.software.ac.uk/software-credit}} convened in London on October
19, following the conclusion of WSSSPE3. See
Appendix~\ref{sec:appendix_SW_credit} for more detailed discussion of related
activities.

\subsubsection{Discussion}

The group discussed a number of topics related to software credit, including a
contributorship taxonomy, software citation metadata, standards for citing
software in publications, and increasing the value of software in academic
promotion and tenure reviews. Although initial discussions both prior to and
during WSSSPE3 focused on contribution taxonomy and dividing credit, discussing
as an example the Entertainment Identifier Registry~\cite{EIDR} used in the
entertainment industry, the group decided to prioritize software citation. This
decision was motivated by the idea that standardizing citations for software
would introduce some initial credit for developers, and later the quantification
of credit could be refined based on concepts such as transitive
credit~\cite{wssspe2_katz,Katz:2014_tc}.

The majority of the remaining discussion focused on standardizing (1) the
metadata necessary for software to be cited and (2) the mechanism for citing
software in publications. Moreover, discussions also oriented around the
indexing of software citations necessary for establishing a software citation
network either integrated with the existing paper citation ecosystem or
complementary to it. See Appendix~\ref{sec:appendix_SW_credit} for a more
detailed summary of the working group's discussion on these topics.

\subsubsection{Plans}

The group already merged with the FORCE11 Software Citation Working Group
(SCWG), and their efforts will focus (over the next six to nine months) on
developing a document describing principles for software citation. Following the
publication of that document, the group will focus on outreach to key
groups (e.g., journals, publishers, indexers, professional societies).
Longer-term plans include working with indexers to ensure that software
citations are indexed and pursuing an open\slash community indexer; these
activities may be organized by future FORCE11 working groups.

\subsubsection{Landing Page}

Since near-term efforts will be shifting to the FORCE11-SCWG, 
interested readers should go to that group's existing landing page\footnote{FORCE11-SCWG
landing page,
\url{https://www.force11.org/group/software-citation-working-group}} and GitHub
repository\footnote{FORCE11-SCWG GitHub page,
\url{https://github.com/force11/force11-scwg}}.

\subsection{Publishing Software Working Group Discussion} \label{sec:publishing-software}

This working group explored the value of executable papers (papers whose content
includes the code needed to produce their own results), and other forms of
publishing which include dynamic electronic content.
%
%\subsubsection{Why it is important}
%
Transitioning to this type of publication offers possibilities of addressing, or
partially addressing, sustainability concerns such as reproducibility, software
credit, and best practices.

\subsubsection{Fit with related activities}

\begin{itemize}
\item \textbf{Reproducibility}: Part of the purpose of these executable paper venues
is to (at least partially) address the reproducibility issue by making papers
recompute their own results.

\item\textbf{Software Credit (\S\ref{sec:software-credit})}: Since these forms of
publishing must make their sources explicit in order to execute, they should be
easier to trace even if appropriately worded credit for software is not
provided. In addition, they
make it possible to provide or define
additional metadata to make the tracing of credit clearer. Finally, attributions
could be added to citations to identify whether a paper extends a result,
verifies it, contradicts it, etc.

\item \textbf{Best Practices (\S\ref{sec:best-practices})}: Because an executable
paper showcases the code, and the code itself is subject to the review process,
authors are more likely to pay attention to coding practices. In addition,
because the paper must explain what the code does, better documentation is
more likely to be achieved.
\end{itemize}

\subsubsection{Discussion}

The group felt that the best way to encourage the use of these new publishing
concepts would be to create and curate a list of publishing venues that
support them. The Software Sustainability Institute agreed to host this list.

See Appendix~\ref{sec:appendix_publishing_SW} for more details about the
discussion.

\subsubsection{Plans}

The plan is to create and curate a web page describing executable papers, their value, and
a list of what publishers support them. We expect the page to be available in
early January of 2016 on the Software Sustainability Institute's website.

\subsubsection{Landing Page}

The aforementioned page will be published on the Software Sustainability
Institute website: \url{http://www.software.ac.uk}.

\subsection{Building Sustainable User Communities for Scientific Software}

%\subsubsection{Why it is important}

User communities are the lifeblood of sustainable scientific software. The user
community includes the developers, both internal and external, of the software;
direct users of the software; other software projects that depend on the
software; and any other groups that create or consume data that is specific to
the software. Together these groups provide both the reason for sustaining the
software and, collectively, the requirements that drive its continued evolution
and improvement.

\subsubsection{Fit with related activities}

There are a number of activities already in progress that are targeted at improving
the user community for open-source software, including Mozilla Science's ``Working Open Project
Guide''~\cite{working-open-wssspe3} and
``UK Collaborative Computational Projects'' (CCP: http://www.ccp.ac.uk), or 
 books such as ``Art of Community'' by Jono Bacon~\cite{art-of-community}. 

\subsubsection{Discussion}

Discussion revolved around a few questions: what are the benefits of having a
``community'' for software sustainability; what practices and circumstances may lead
to having and maintaining a community; how can funding help or hinder this
process; and perhaps most importantly, how can best practices be described and
distilled into a document that can help new projects.

Everyone agreed on a few points: software must not only offer value, but there
must be some support for users; and funding can help pay for that support, in
addition to the usual funding for software development. Openness is generally 
a virtue. An evangelist, either in the form of a single person or some
domain-specific group of users, is often the key factor.

Additional details on the group's discussion can be found in
Appendix~\ref{sec:appendix_user_community}.

\subsubsection{Plans}

The most important next steps is a ``Best Practice'' document, which would
describe what successful projects with engaged communities look like, how to
replicate this type of project, and look at the end of life of a community project.
Another next step would be better training to increase recognition of need for science software
projects to focus on building and supporting their user communities.

\subsubsection{Landing Page}

This group does not have a landing page yet. Please send requests to join and
contribute by writing to both Dan Gunter\footnote{email:
\href{mailto:dkgunter@lbl.gov}{dkgunter@lbl.gov}} and Ethan
Davis\footnote{email: \href{mailto:edavis@ucar.edu}{edavis@ucar.edu}}.


%%%%%%%%%%%%%%%%%%%%%%%%%%%%%%%%%%%%%%%%%%%%%%%%%%%%%%%%%%%%
\section{Conclusions} \label{sec:conclusions}
%%%%%%%%%%%%%%%%%%%%%%%%%%%%%%%%%%%%%%%%%%%%%%%%%%%%%%%%%%%%

The WSSSPE3 workshop continued our experiment from WSSSPE1 and WSSSPE2 in how we can
collaboratively build a workshop agenda, and we began a new experiment in
how to build a series of workshops into an ongoing community activity.

\todo{need to update rest of this section for WSSSPE3, except the tweet table has already been updated}
The differences in workshop organization in WSSSPE2 from WSSSPE1
are in using an existing service (EasyChair) to handle submissions and reviews,
rather than an ad hoc process, and using an existing service (Well Sorted) to
allow collaborative grouping of papers into themes by all authors, reviewers,
and the community, rather than this being done in an ad hoc manner by the
organizers alone.

The fact remains that contributors also want to get credit for their
participation in the process. And the workshop organizers will want to make
sure that the workshop content and their efforts are recorded. Ideally, there
would be a service that would index the contributions to the
workshop, serving the authors, the organizers, and the larger community. 
Since there still isn't such a service today, the workshop organizers are
writing this initial report and making use of arXiv as a partial solution to
provide a record of the workshop.

\begin{table*}[t]
\centering
\caption{Top tweets tagged \#WSSSPE or \#WSSSPE3 on Sep.\ 28--29, 2015.}\label{tab:tweets}
  \begin{scriptsize}
  \begin{tabular}{@{}l l l l@{}}
 \toprule
    Author  &   Tweet  & Retweets &  Favorites
\\ \midrule
Neil P Chue Hong  &  Getting ready for the start of the \#WSSSPE workshop & 2 & 1
\\ & \url{http://wssspe.researchcomputing.org.uk/wssspe3/agenda/}  & &
\\ & Video stream: \url{http://ucarconnect.ucar.edu/live?room=cg1aud} & &
%
\\ August Muench & Astronomy \& Software for a totally distracting Monday (2/2) => & 1 & 3
\\ & Workshop on Sustainable SW for Science \#WSSSPE3 & & 
\\ & \url{http://wssspe.researchcomputing.org.uk/wssspe3/agenda/} & &
%
\\ Neil P Chue Hong & My \#WSSSPE lightning talk on Building a Scientific Software & 6 & 3
\\ &  Accreditation Framework: \url{http://dx.doi.org/10.6084/m9.figshare.1555925} & & 
%
\\ Mozilla Science Lab & At \#wssspe3 and want to learn more about our Working Open guide & 2 & 1 
\\ &  or Contributorship badges pilot? Keep an eye out for @abbycabs! & & 
%
\\ Daniel S.\ Katz & \#wssspe starting! & 1 & 3
%
\\ Daniel S.\ Katz & @powersoffour starting \#wssspe keynote & 1 & 2
%
\\ Neil P Chue Hong   &  The difference between software presented at SC \& SciPy & 1 & 4
\\  &  according to @powersoffour straw person \#wssspe & &
%
\\ Daniel S.\ Katz & sustainability may mean: keeps up with bug reports, adds new features, & 3 & 1
\\ &  works on new architectures (1/2)  -- @powersoffour at \#wssspe (1/2) & & 
%
\\ Daniel S.\ Katz & sustainability may mean: people use it, keeps getting funded, produces & 2 & 
\\ &  same results, transits over people (2/2)  -- @powersoffour at \#wssspe & &
%
\\ Nic Weber & Rec, by Matt Turk: Ash Dryden - Ethics of Unpaid Labor & 3 & 2 
\\ & \url{http://t.co/KfMwS4VG4N} & &
%
\\ Nic Weber & My .02 - many more models (than 3) for sustaining sci. software. & 3 & 2
\\ & See this pub on open-source:  & & 
\\ & \url{http://papers.ssrn.com/sol3/papers.cfm?abstract_id=2568185} & &
%
\\ Erin Robinson & \#WSSSPE - Software is like a vampire. It's only dead if you really kill it. & 2 & 2
%
\\ Neil P Chue Hong & Absolutely great \#wssspe keynote from @powersoffour to inspire & 2 & 1 
\\ & the workshop. Should be archived via & & 
\\ & \url{http://ucarconnect.ucar.edu/live?room=cg1aud#.VgmpvCdIrCQ} soon & & 
%
\\ Daniel S.\ Katz & \#wssspe lightning talks starting - streaming at  & 5  &  1
\\ &  \url{http://ucarconnect.ucar.edu/live?room=cg1aud} & & 
%
\\ Kyle Niemeyer & Participatory open source means better documentation  & 2 & 1 
\\ & (readme, roadmap, contributing) ? @abbycabs at \#wssspe & &
%
\\ Erin Robinson & Interesting resource: \url{http://mozillascience.github.io/leadership-training/} & 2 & 
\\ &  \#WSSSPE cc: @abbycabs & & 
%
\\ Nic Weber & @MozillaScience + @abbycabs giving look at cool resources & 2 & 3
\\ & for leading an open source community & & 
\\ & \url{http://mozillascience.github.io/leadership-training/} & & 
%
\\ Nic Weber & Some interesting sepeartion of best practices for Geosoftware dev & 2 & 
\\ & into categories \url{https://geodynamics.org/cig/dev/best-practices} & &
%
\\ Neil P Chue Hong & Some statistics of software citation in geodynamics & 2 & 
%
\\ Daniel S.\ Katz & @jamespjh goal: make research software less rubbish; solution is RSEs & 3 & 
%
\\ Nic Weber & Highly relevant to \#wssspe crowd - @Impactstory team is launching & 7 & 3
\\ & a impact of research software project \url{http://depsy.org/}  \#verycool & & 
%
\\ Neil P Chue Hong & @jamespjh talking about Research Software Engineer groups in the UK & 2 & 1
\\ &  \#wssspe @ResearchSoftEng @SoftwareSaved & & 
%
\\ Abby Cabunoc Mayes & Wonderful time @ \#WSSSPE 3! Encouraging 2 see a diverse community & 1 & 2 
\\ & working 2gether (\& making real plans!) \url{http://gif.co/s6v4.gif} & & 
%
\\ \bottomrule
    \end{tabular}
    \end{scriptsize}
\end{table*}

WSSSPE actively used the online social network Twitter, with hashtag
``\#WSSSPE''. There were substantially more tweets (messages) during the days of
the workshops WSSSPE2, WSSSPE1.1, and WSSSPE1. Out of about 670 tweets as of Apr
18, 2015, more than 225 were about WSSSPE2 and about 180 were posted during the
day of the workshop. Some of the main points and highlights in the meeting are
shown in Table~\ref{tab:tweets}, which summarizes the top \#WSSSPE tweets from
the day of workshop, selected by the metrics that number of retweets or
favorites larger than five and the sum of two measures greater than ten.

In terms of building community activities, we wanted to focus primarily on
working groups, which we were able to do, as discussed above, but we
also wanted to make sure that attendees felt they had a chance to get their
ideas across to the whole group, which was the purpose of the lightning talks.
Overall, this seemed to be successful at the time, in terms of both the lightning
talks and the breakout groups, and the discussion of sustainability also led
to interesting and useful results. However, the challenge that we have discovered
since WSSSPE2 is that it is very hard to continue the breakout groups'
activities.  The WSSSPE2 participants were willing to dedicate their time to
the groups while they were at the meeting, but afterwards, they have gone
back to their (paid) jobs.  We need to determine how to tie the WSSSPE
breakout activities to people's jobs, so that they feel that continuing them
is a higher priority than it is now, perhaps through funding the participants,
or through funding coordinators for each activity, or perhaps by getting
the workshop participants to agree to a specific schedule of activities during the
workshop.


%%%%%%%%%%%%%%%%%%%%%%%%%%%%%%%%%%%%%%%%%%%%%%%%%%%%%%%%%%%%
\section*{Acknowledgments} \label{sec:acks}
%%%%%%%%%%%%%%%%%%%%%%%%%%%%%%%%%%%%%%%%%%%%%%%%%%%%%%%%%%%%

Work by Katz was supported by the National Science Foundation while working at
the Foundation. Any opinion, finding, and conclusions or recommendations
expressed in this material are those of the author(s) and do not necessarily
reflect the views of the National Science Foundation.

\todo{feel free to add stuff here}


\appendix
%%%%%%%%%%%%%%%%%%%%%%%%%%%%%%%%%%%%%%%%%%%%%%%%%%%%%%%%%%%%
\section{Attendees}  \label{sec:attendees}
%%%%%%%%%%%%%%%%%%%%%%%%%%%%%%%%%%%%%%%%%%%%%%%%%%%%%%%%%%%%
%\todo{Do we want email addresses here?}
The following is list of participants registered for the WSSSPE3 workshop.

{\scriptsize
\begin{longtable}{lll}
Alice Allen & Astrophysics Source Code Library\\
Gabrielle Allen & NCSA\\
Janine Aquino & UCAR/NCAR Earth Observing Laboratory\\
Steven Brandt & Louisiana State University\\
Jed Brown & CU Boulder\\
Matthias Bussonnier & UC Berkeley\\
Jeffrey Carver & University of Alabama\\
Emily Chen & NCSA\\
Sou Cheng Choi & NORC at UChicago / IIT\\
Nancy Collins & NCAR\\
Ethan Davis & UCAR Unidata\\
Davide DelVento & NCAR/CISL\\
Yuhan Ding & Illinois Institute of Technology\\
Tim Dunne & KnowInnovation \\
Ward Fisher & UCAR/Unidata\\
Sandra Gesing & University of Notre Dame\\
Josh Greenberg & Sloan Foundation\\
Dan Gunter & LBNL\\
Ted Habermann & The HDF Group\\
James Hetherington & University College London\\
Neil Chue Hong & Software Sustainability Institute\\
Elisabeth Huffer & Lingua Logica/NASA \\
Lorraine Hwang & UC Davis - CIG\\
Raymond Idaszak & RENCI; University of North Carolina at Chapel Hill\\
Elizabeth Jessup & University of Colorado Boulder\\
Nick Jones & New Zealand eScience Infrastructure (NeSI)\\
Daniel Katz & U Chicago \& Argonne\\
Iain Larmour & EPSRC (UK)\\
Frank Löffler & Louisiana State University\\
Suresh Marru & Indiana University\\
Ryan May & UCAR/Unidata\\
Abigail Cabunoc Mayes & Mozilla Foundation\\
Jeff McWhirter & Geode Systems\\
Constantinos Michailidis & Knowinnovation\\
Don Middleton & NCAR\\
Mark Miller & SDSC\\
Pate Motter & University of Colorado\\
Jaroslaw Nabrzyski & University of Notre Dame\\
Patrick Nichols & National Center for Atmospheric Research\\
Kyle Niemeyer & Oregon State University\\
Laura Owen & NCSA\\
Abani Patra & Univ at Buffalo\\
Grace Peng & National Center for Atmospheric Research\\
Birgit Penzenstadler & California State University Long Beach\\
Lindsay Powers & The HDF Group\\
Bernie Randles & UCLA\\
Erin Robinson & Foundation for Earth Science\\
Daniel Sellars & CANARIE Inc\\
Nikolay Simakov & SUNY University at Buffalo\\
Ian Taylor & Cardiff University\\
Ilian Todorov & Science \& Technology Facilities Council, UK\\
Benjamin Tovar & University of Notre Dame\\
Gregory Tucker & University of Colorado at Boulder\\
Matthew Turk & NCSA\\
Colin Venters & University of Huddersfield\\
Alexander Vyushkov & University of Notre Dame\\
Fraser Watson & National Solar Observatory\\
Nic Weber & University of Washington\\
Daniel Ziskin & NCAR - ACOM\\

\end{longtable}
}

%%%%%%%%%%%%%%%%%%%%%%%%%%%%%%%%%%%%%%%%%%%%%%%%%%%%%%%%%%%%
\section{Best Practices Group Discussion}
\label{sec:appendix_best_practices}
%%%%%%%%%%%%%%%%%%%%%%%%%%%%%%%%%%%%%%%%%%%%%%%%%%%%%%%%%%%%

\todo{add POC here}

\subsection{Group Members}
\katznote{add affils for all, please}

\begin{itemize}
\item Abani Patra -- University at Buffalo
\item Sandra Gesing -- University of Notre Dame
\item Neil Chue Hong -- Software Sustainability Institute
\item Gregory Tucker -- University of Colorado at Boulder
\item Birgit Penzens -- California State University Long Beach
\item Abigail Cabunoc Mayes -- Mozilla Foundation
\item Jeff Carver -- University of Alabama
\item Frank L\"{o}ffler -- Louisiana State University 
\item Colin Venter --  University of Huddersfield
\item Lorraine Hwang -- UC Davis 
\item Sou-Cheng Choi -- NORC at the University of Chicago \&  Illinois Institute of Technology
\item Suresh Marru -- Indiana University
\item Don Middleton -- NCAR 
\item Daniel Katz --  University of Chicago \& Argonne National Laboratory
\item Kyle Niemeyer -- Oregon State University
\item Jeffrey Carver -- University of Alabama
\item Dan Gunter -- LBNL
\item Alexander Konovalov -- \choinote{TBD}
\item Tom Crick --  \choinote{TBD}

\end{itemize}

\subsection{Summary of Discussion}

Core questions that will need to be explored are in knowledge management, 
(transitions between people), reliability (reproducibility), usability, and how a software tool becomes part of the core workflow of well identified users (stakeholders)
relating to tool success and hence sustainability. \katznote{prev sentence is complex and awkward} Ideas 
that may need to be explored include:
\begin{itemize}

\item Requirements engineering to create tools with immediate uptake;

\item When should software ``die''?

\item Catering to disruptive developments in environment, e.g., new hardware,
new methodology;

\item Dimensions of sustainability -- economic, technical, environmental,
declining interest in primary application area), \katznote{not sure what the
prev. comment goes with} social.

\end{itemize}

Sustainability requires community participation in code development and/or a
wide adoption of software. The larger the community base is using a piece of
software, the better are the funding possibilities and thus also the
sustainability options. Additionally,  developers’ commitment to an application is
essential and experience shows that software packages with an evangelist
imposing strong inspiration and discipline are more likely to achieve
sustainability. While a single person can push sustainability to a certain
level, open source software also needs sustained commitment from the developer
community. Such sustained commitments include diverse tasks and roles, which can
be fulfilled by diverse developers with different knowledge levels. Besides
developing software and appropriate software management with measures for
extensibility and scalability of the software, active (expertise) support for
users via a user forum with a quick turnaround is crucial. The barrier to entry
for the community as users as well as developers has to be as low as possible.

\subsection{Description of Opportunity, Challenges, and Obstacles}

The opportunity lies in collaboration on a white paper, which will be revisited
regularly for further improvements, to enhance knowledge of the state of best
practices, resulting in a peer-reviewed paper. We would like to reach a wide
community by doing this. But these are also the challenges and obstacles -- to
get everyone to contribute to the paper and to reach the community.

\subsection{Key Next Steps}

The key next steps are to write an introduction, reach out to the co-authors,
and to agree on the scope of the white paper.

\subsection{Plan for Future Organization}

Sandra Gesing and Abani Patra are the main editors and will organize the overall
communication and the paper. Sections will be assigned to diverse co-authors.

\subsection{What Else is Needed?}

At the moment we do not see any further requirements.

\subsection{Key Milestones and Responsible Parties}
\begin{itemize}
\item 15 Nov: Introduction and scope finished (Abani Patra/Sandra Gesing)
\item 15 Nov: Sections assigned (Abani Patra/Sandra Gesing)
\item 31 Jan: Analyzing funding possibilities for survey
\item 31 Jan: First version of each section
\item 15 Feb: Distribution to the WSSSPE community
\item 31 Mar: Final version of the white paper
\item 30 Apr: Submission to a peer-reviewed journal?
\end{itemize}

\subsection{Description of Funding Needed}
We might need funding for a journal publication (open-access options).

%%%%%%%%%%%%%%%%%%%%%%%%%%%%%%%%%%%%%%%%%%%%%%%%%%%%%%%%%%%%
\section{Funding Specialist Expertise Group Discussion}
\label{sec:appendix_funding_spec_expert}
%%%%%%%%%%%%%%%%%%%%%%%%%%%%%%%%%%%%%%%%%%%%%%%%%%%%%%%%%%%%

James Hetherington\footnote{email: \href{mailto:j.hetherington@ucl.ac.uk}{j.hetherington@ucl.ac.uk}}
will serve as the point of contact for this working group, and be responsible for ensuring timely progress of the planned actions.

\subsection{Group Members}

The group at WSSPE:

\begin{itemize}
\item Don Middleton -- National Center for Atmospheric Research
\item Joshua Greenberg -- Alfred P. Sloan Foundation
\item James Hetherington -- University College London
\item Lindsay Powers -- The HDF Group
\item Mark A. Miller -- San Diego Supercomputer Center
\item Dan Sellars -- CANARIE
\end{itemize}

This was further enhanced by additional discussions at the following
GCE15 conference:

\begin{itemize}
\item Lorraine Hwang -- UC Davis
\item Simon Trigger
\item Nancy Wilkins-Diehr -- San Diego Supercomputer Center
\item Alexander Vyushkov -- Notre Dame
\item Sandra Gesing - Notre Dame
\item Ali Swanson -- University of Oxford

\end{itemize}

\subsection{Summary of Discussion}

In addition to the points noted in the main discussion~\ref{RSE}, we also
discussed the following:

``Are you an RSE or a RA?'' -- this is not properly a binary question. Most of
us sit at different points on that spectrum, and move along it during our
careers. (Ususally from RA to RSE -- examples of movement in the other direction
from readers would be welcomed.)
Either way, the label ``Research Software Engineer'' is now starting to
have some power. Many scientists do not want to be writing code; some do, to
varying degrees. These groups can usefully support each other.


he power of the label? Getting the word out about RSE support using the label.


Will research science developers be required, in the long run?
One issue that came up was whether the need for developers was a time bounded one; is it the case that the new generation of computer and software savvy scientists will be so comfortable in developing their own code that the professional developer will not be needed.  And this brings up the flip side question, “Do scientists really want to be writing code?”
Career Path.
We also had a little discussion about how to make a career path for research developers. It need not be solely an academic enterprise, but tenure is always problematic for people of this class.

Skills and resources may vary between teams. To help resolve this, maintaining
high levels of communication between groups will be valuable. In the UK, there
are plans to permit resource sharing between institutional RSE groups.

Collaborative funding can be important for RSE groups, to ensure that research
leadership remains with the domain scientists. At NCAR, University partnerships
are required for submission of proposals, so collaboration is an essential part
of grant submission, and this will tend to bring developers and
scientists together. The UCL group also follows this approach, with all bids
requiring an academic collaborator.
Domain scientists and developers are funded together in a single proposal.
Another example of a success is the development of semantics and linked data in
support of Ocean Sciences. An Earthcube funded project pairs domain scientists
with RSEs and has been successful as the semantics attached have increased data
use and discovery significantly.

An alternative approach has been the provision of programming expertise as
part of national compute services. The US XSEDE project's extended collaboration
 service (ECS) is a set of developers who are paid with XSEDE funding,
 and are on “permanent” staff.  When PIs request allocations on XSEDE resources,
 there is a finite pool of developer time that can be awarded,
typically for one year only, and at partial effort, typically 20 percent or so.
The finite time allowed provides motivation for the scientist and their group to
work closely the developer and to become educated in what the developer is doing,
so they can sustain the effort, once the ECS period is over. This funding
mechanism can be highly efficient for scientific problems, because the developer
pool assembled by the research providers are, by definition, expert in the
characteristics of their specific resource, and can very quickly assess the
scientists needs, and what it will take to implement software that meets the
user’s needs. However, it does not develop capacity within institutions,
and indeed can have a ``chilling effect''.

The UK allows this kind of collaboration to support the creation of scientific
software for the large supercomputing resource called ARCHER.
However, as well as resourcing this from the staff of the Edinburgh Parallel
Computing Centre, who host the computer, this ``embedded CSE'' resource also
allows the programming to come from local groups. This has been very helpful
in providing funding to establish local groups. These work best when they
develop good collaborations with national cyberinfrastucture pools.
When an organization assembles
a developer pool, diversity is developed and skills can be transferred.

We would like to see these models applied outside High Performance Computing;
most scientific software is not destined to run on national cyberinfrastructure,
but needs similar support. The argument regarding making better use of expensive
hardware through software improvements has been useful politically, (and many
RSE groups are cited in organisations which host clusters for this reason), but
the time has come to make the case that software itself is a critical
cyberinfrastructure, and, with a much longer shelf-life than hardware, is itself
a capital investment.

The Canarie group (Canada) accepts proposals
for providing services to broad communities, integrate people who are doing
things that are complementary, the goal is to make the available stack more
robust and richer for everyone. They offer short rounds of funding that can
have as a key metric creating some useful functionality that shows a diversity
of input and draws from across disciplines, then more funding could follow. This
could apply within or across institutions.

There can be problems communicating across cultural barriers, with domain
scientists seeing developers as “other”.
Collaboration, and tools to fund, encourage or motivate collaboration are
extremely important.

We think support from non-governmental organisations will be important if RSE
groups will become established.
The Sloan Foundation is currently funding data science engineers, who work in
the context of other SW developers at University of Washington.
These scientists work in the e-Science Studio/dataScience Studio, and they
help a group of graduate students in solving their problems in data
science and data management. During Fall and Spring, a 10-week incubator program
allows students to work two days a week to work on a data-intensive science
project. Some fraction of the developer time is dedicated to the developers'
personal interests as well as instruction.

The goal for Sloan is to provide success stories, to provide demonstrable value
in the presence of data scientists on university staff. These stories are the
basis for arguments to the host organization. This is an effort to create
awareness of the value of research scientist developers. Embedding with
scientists, and adding spare capacity is critical to making the innovation
possible. This model is essentially to argue for permanent budget lines to
support data scientists as part of university staff hires, just as with core
facilities. This could become a fee-for-service model requested by grant
funding, just as DNA sequencing is for core facilities, if it becomes apparent
that this gives competitive advantage to the University’s research effort.

One model that has been helpful in finding funding for RSE groups is the use of
funds left over on research grants when RAs have left prematurely -- PIs like
this arrangement as it is hard to find good staff for short-term positions, so
having a pool of research programming staff on hand resolves this problem.
We recommend that funders give explicit guidance to grant holders and
institutions that such arrangement are favourable. Framework agreements
permitting this to go ahead without
checking back every time with funders and/or grant panels would further
smooth this.
(This is also provides a more stable job for those who hold these skills, but
arguments about making life nicer for postdocs will not help persuade funders
or PIs!)

There is some question about the most effective
duration and percent of full time for a programmer's work on a project.
At least three months is necessary for the programmer to read into the science
(RSEs must not become so disengaged from research that they don't have time to
read a few papers -- this will result in code which doesn't meet scientific
needs), but too long could result in an RSE losing their flexibility, becoming
so engaged in one project that when that project ends, they find it hard to
transfer. For this reason, we also recommend that 40 percent is ideal;
two projects per developer, with some time for training and infrastructure work.
Two developers per project seems to be ideal, in the sense that software
development is enhanced by two pairs of eyes.

There is as yet no clear answer as to the scale of aggregation needed to
make such a program work. A university wide programme allows enough scale to
be robust to fluctuations of funding within one field.
But a specialization focus on developers to support, for example,
physical or biological
sciences may be preferable, if the customer base is large enough.
The desire to aggregate enough work to make it sustainable, and the need to
have domain-relevant research programming skils, are in tension.

In the UK, another source of funding for research software has been the
Collaborative computational projects (CCPs): domain specific communities put
forward proposals that are a priority of the community as a whole, for example,
biosimulation or plasma physics. These bodies act as custodians of community codes,
and a central team also provides software engineering support support.

However this area develops, the need for funding to secure software as a
cyberinfrastructure component is clear. Funding which permits code to be
refactored, tidied and optimised is rare; this is often done ``on the sly''
in a scientifically focused grant. The UK EPSRC's ``software for the future''
call, which really permits explicit investiment in software as an infrastructure,
is so oversubscribed as to have a FOUR PERCENT success rate; the demand is clear!

One opportunity is the idea of co-design; where infrastructural libraries are
developed alongside the scientific codes that will call them. However,
collaboration is hard to foster here; as incentive structures are still focused
on short-term papers. This can cause infrastructure developers to
focus more on publications
in their areas of mathematics and CS, the domain developers on the shorter-term
needs of their own fields. Genuine collaborative co-construction
is harder to foster.

It can be a more difficult
to help leading domain scientists see the value of engineering effort than those
in their teams who are forced to work with difficult-to-use or unreliable
software tools, as they do not see the pain. Perhaps a version of
``software carpentry'' targeted at those PIs
who are awarded or apply for software heavy grants could be valuable here.

RSEs provide a useful contribution to their Universities' *teaching* missions,
as well as for research, as they are well placed to deliver the research programming
training that many scientists now need. In the
longer term, with programming skills taught to all through their career,
we hope specialist
scientific developers will be less needed.

\subsection{Key Next Steps}

We will seek to identify and approach existing research programming organisations,
to get their permission to list them on a list of research software groups.
We will also look for examples of groups which have successfully become self-
sustaining following

\subsection{Plan for Future Organization and Future Needs}

The UKRSE community will provide initial facilities to host this list, and
continue to work to spread the initiative, but local leadership in the US
is needed if this campaign is to succeed. This will require an initial gathering
of identified research software organisations in the US to this end.

\subsection{Description of Funding Needed}

Financial support for an initial conference to bring together research software
groups to form an organisation and create a resource sharing structure would
help to further this campaign.

In the longer term, funding organisations, especially non-governmental organisations
with the capability to effect innovation through seed funding, could provide
support to nucleate the creation of research software groups.

%%%%%%%%%%%%%%%%%%%%%%%%%%%%%%%%%%%%%%%%%%%%%%%%%%%%%%%%%%%%%
\subsection{Catalogs Working Group Discussion}
\label{sec:appendix_catalogs}
%%%%%%%%%%%%%%%%%%%%%%%%%%%%%%%%%%%%%%%%%%%%%%%%%%%%%%%%%%%%

\todo{add POC here}

\subsubsection{Group Members}

\begin{itemize}
\item name -- affiliation
\item name2 -- affiliation2
\end{itemize}

\subsubsection{Summary of Discussion}

\subsubsection{Description of Opportunity, Challenges, and Obstacles}


\subsubsection{Key Next Steps}


\subsubsection{Plan for Future Organization}


\subsubsection{What Else is Needed?}


\subsubsection{Key Milestones and Responsible Parties}


\subsubsection{Description of Funding Needed}

%%%%%%%%%%%%%%%%%%%%%%%%%%%%%%%%%%%%%%%%%%%%%%%%%%%%%%%%%%%%
\section{Industry Interaction Working Group Discussion}
\label{sec:appendix_industry_interaction}
%%%%%%%%%%%%%%%%%%%%%%%%%%%%%%%%%%%%%%%%%%%%%%%%%%%%%%%%%%%%

\subsection{Group Members}
{\small
\begin{longtable}{ll}
   name            &  affiliation
\\ name2           &  affiliation2
\end{longtable}
}

\subsection{Summary of Discussion}

\subsection{Description of Opportunity, Challenges, and Obstacles}


\subsection{Key Next Steps}


\subsection{Plan for Future Organization}


\subsection{What Else is Needed?}


\subsection{Key Milestones and Responsible Parties}


\subsection{Description of Funding Needed}

%%%%%%%%%%%%%%%%%%%%%%%%%%%%%%%%%%%%%%%%%%%%%%%%%%%%%%%%%%%%
\section{Legacy Software Working Group Discussion}
\label{sec:appendix_legacy_SW}
%%%%%%%%%%%%%%%%%%%%%%%%%%%%%%%%%%%%%%%%%%%%%%%%%%%%%%%%%%%%

\subsection{Group Members}

\begin{itemize}
\item name -- affiliation
\item name2 -- affiliation2
\end{itemize}

\subsection{Summary of Discussion}

\subsection{Description of Opportunity, Challenges, and Obstacles}


\subsection{Key Next Steps}


\subsection{Plan for Future Organization}


\subsection{What Else is Needed?}


\subsection{Key Milestones and Responsible Parties}


\subsection{Description of Funding Needed}

%%%%%%%%%%%%%%%%%%%%%%%%%%%%%%%%%%%%%%%%%%%%%%%%%%%%%%%%%%%%
\section{Engineering Design Group Discussion}
\label{sec:appendix_eng_design}
%%%%%%%%%%%%%%%%%%%%%%%%%%%%%%%%%%%%%%%%%%%%%%%%%%%%%%%%%%%%

Birgit Penzenstadler\footnote{email: \href{mailto:birgit.penzenstadler@csulb.edu}{birgit.penzenstadler@csulb.edu}} and Colin C. Venters\footnote{email: \href{mailto:c.venters@hud.ac.uk}{c.venters@hud.ac.uk}} will serve as the points of contact for this working group, and be responsible for ensuring timely progress of the planned actions.

\subsection{Group Members}

\begin{itemize}
\item Birgit Penzenstadler -- California State University, CA, USA
\item Colin C. Venters -- University of Huddersfield, Huddersfield, UK
\item Matthias Bussonnier -- UC Berkeley, CA, USA
\item Jeff McWhirter -- Geode Systems 
\item Patrick Nichols -- National Center for Atmospheric Research, CO, USA
\item Ilian Todorov -- Science \& Technology Facilities Council, UK
\item Ian Taylor -- Cardiff University, UK
\item Alexander Vyushkov -- University of Notre Dame, IN, USA
\end{itemize}

\subsection{Summary of Discussion}
Software engineering principles form the basis of methods, techniques, methodologies and tools. This group discussed the principles of software engineering design for sustainable software and their application in various domains. The group included members from different backgrounds, including quantum chemistry, epidemiology, computer science, software engineering, and microscopy. Each participant was invited to give their perspective on the topic area and what they thought were the crucial points for discussion. There was a general consensus that there was a need for relating principles to practice for the computational science and engineering community. Furthermore, various members of the group expressed their interest in tools and best practices for facilitating the maintenance and evolution of scientific software systems. It was agreed to identify principles from software engineering and from sustainability design and, based on those lists, discuss what each of those would mean applied to specific example systems from the expert domains of some of the group members. The group identified a number of software engineering principles drawn from the SoftWare Engineering Body of Knowledge (SWEBOK)~\cite{swebokv3}:

Software design principles:
\begin{itemize}
\item Abstraction;
\item Coupling and cohesion;
\item Decomposition and modularization;
\item Encapsulation and information hiding;
\item Separation of interface and implementation;
\item Sufficiency completeness & primitiveness;
\item Separation of concerns.
\end{itemize}

User interface design principles:
\begin{itemize}
\item Learnability;
\item User familiarity;
\item Consistency;
\item Minimal surprise;
\item Recoverability;
\item User guidance;
\item User diversity
\end{itemize}

The sustainability design principles were drawn from the Karlskrona Manifesto on Sustainability Design~\cite{karlskrona,becker2015}:
\begin{itemize}
\item Sustainability is systemic;
\item ...is multidimensional
\item ...is interdisciplinary;
\item ...transcends the system’s purpose;
\item ...applies to both a system and its wider contexts;
\item ...requires action on multiple levels;
\item ...requires multiple timescales;
\item Changing design to take into account long-term effects doesn’t automatically imply sacrifices;
\item System visibility is a precondition for and enabler of sustainability design.
\end{itemize}

This congregated list is an initial collection of principles that could be extended by adding from further related work from separate disciplines within the field of software engineering, including requirements engineering, software architecture, and testing. The group identified two example systems to discuss the application of the principles. The first one was a quantum chemistry system that allows the analysis of the characteristics and capabilities of molecules and solids. The second one was a modeling system for malaria that permitted biologists to analyze a range of datasets across geography, biology, and epidemiology, and add their own datasets. The group then examined the principles and took a retrospective analysis of what the developers did in practice against how the principles could have made a difference. 

\subsection{Description of Opportunity, Challenges, and Obstacles}
The opportunity was identified to distill existing software engineering and sustainability design knowledge into “bite sized” chunks for the Computational Science and Engineering Community. In addition, two challenges were pointed out: 
\begin{itemize}
\item Mapping of the principles to best practice.
\item Demonstrating the return on investment of those best practices.
\end{itemize}

\subsection{Key Next Steps}
The next steps in this endeavor are to (1) Systematically analyze a number of example systems from different scientific domains with regards to the identified principles, to (2) Identify the commonalities and gaps in applying those principles to different scientific systems, and to (3) Propose a guideline on the principles and how they exemplary apply to scientific software system.

\subsection{Plan for Future Organization}


\subsection{What Else is Needed?}


\subsection{Key Milestones and Responsible Parties}


\subsection{Description of Funding Needed}

%%%%%%%%%%%%%%%%%%%%%%%%%%%%%%%%%%%%%%%%%%%%%%%%%%%%%%%%%%%%
\section{Metrics Working Group Discussion}
\label{sec:appendix_metrics}
%%%%%%%%%%%%%%%%%%%%%%%%%%%%%%%%%%%%%%%%%%%%%%%%%%%%%%%%%%%%

Gabrielle Allen\footnote{email: \href{mailto:gdallen@illinois.edu}{gdallen@illinois.edu}} will serve as the point of contact for this working group.


%%%%%%%%%%%%%%%%%%%%%%%%%%%%%%%%%%%%%%%%%%%%%%%%%%%%%%%%%%%%
\subsection{Group Members}
%%%%%%%%%%%%%%%%%%%%%%%%%%%%%%%%%%%%%%%%%%%%%%%%%%%%%%%%%%%%

\begin{itemize}
\item Gabrielle Allen -- University of Illinois at Urbana-Champaign
\item Emily Chen -- University of Illinois at Urbana-Champaign
\item Neil Chue Hong -- U.K. Software Sustainability Institute
\item Ray Idaszak -- RENCI, University of North Carolina at Chapel Hill
\item Iain Larmou -- Engineering and Physical Sciences Research Council
\item Bernie Randles -- University of California, Los Angeles
\item Dan Sellars -- Canarie
\item Fraser Watson -- National Solar Observatory
\end{itemize}

\subsection{Summary of Discussion}

The following summary of the group's discussion represents the Useful Metrics for Scientific Software working group's discussion during 
the WSSSPE3 workshop. The group discussion began by agreeing on the common purpose of creating a set of guidance giving examples 
of specific metrics for the success of scientific software in use, why they were chosen, what they are useful to measure, and 
any challenges and pitfalls; then publish this as a white paper.  The group discussed many questions related to useful metrics for scientific software as follows: 

\begin{itemize}

\item
Is there a common set of metrics, that can be filtered in some way

\end{itemize}

\begin{itemize}
\item
        Does this create a large cost
\end{itemize}

\item
Can we fit metrics into a common template (i.e. for collection, for description)

\item
Which would be the most useful ones

\begin{itemize}
\item
        Which ones would be most useful for each stakeholder
\end{itemize}

\item
Which ones are the most helpful, and how would we assess this

\item
How do you monitor

\begin{itemize}
\item
        Self-checking - if monitoring is done in the open, then people will call out cheats
\end{itemize}

\item
Should this be published with the software metadata

\begin{itemize}
\item
        This would make it easier for public to see the metadata

\item
        However, there is no commonly used standard (DOAP is a good standard but not widely adopted) 

\item
        The Open Directory Project (ODP) metadata is available for UK infrastructure
\end{itemize}

\item
Intersection of most useful and easiest to collect should be explored

\item
How can students/curricula be used as part of a solution

\item
Number of users could be affected by other metrics e.g. by accessibility

\item
Assume metrics are collected properly, but guidance should be provided none-the-less

\item
Continuum for each metric

\begin{itemize}
\item
        Ideal situation is the absolute minimum, so that people can decide on their own what the cost versus usefulness tipping point is
\end{itemize}

\item
Maturity plays a part

\begin{itemize}
\item
        Consider different metrics brackets for different maturity levels
\end{itemize}

\item
What are we using metrics for

\begin{itemize}
\item
        What software should I use if I have a choice

\item
        Where should funders place funding for best impact (e.g. funding two-star software versus three-star) and where there are gaps

\item
        How to promote reduction of code proliferation

\item
        Metrics used for software panels to provide information

\item
        Metrics used for finding problems in their systems

\end{itemize}

\item
Can we use metrics to help people identify the best codes as part of a community effort

\end{itemize}

\smallskip
\noindent
Next, a roadmap for how to proceed was discussed including creating a set of milestones and tasks as follows:

\begin{itemize}
\item
Can we create a roadmap and milestones for this activity

\item
Need to come up with a set of tasks

\item
Go to NSF Software Infrastructure for Sustained Innovation (SI2) projects asking them what metrics they defined, and how useful they were

\begin{itemize}
\item
        Milestone: Create report which assesses the metrics that SI2 projects used

\begin{itemize}
\item
                Ask SI2 PIs to say what metrics they said they would use (copied from proposal)

\item
                Ask SI2 PIs what numbers they reported

\item
                Ask SI2 PIs what they would have changed

\item
                A UIUC student on the project will work on this
\end{itemize}

\item
        Tentatively aim for March 2016
\end{itemize}

\item
Do something similar for UK SFTF and TRDF software projects to ask them what would be useful metrics to report; also eCSE projects

\begin{itemize}
\item
        Compare these to understand if there were any implications for including metrics
\end{itemize}

\item
Collaboratively create plan and documentation for doing this

\begin{itemize}
\item
        Give some examples from group members projects, and aim to build out some of the measurement continuum

\item
        Road-test at the WSSSPE4 meeting
\end{itemize}

\item
Collect the various frameworks together and do a comparison summary

\end{itemize}

\smallskip
\noindent
The idea was put forth for the group to interact with the organizing committee of the 2016 NSF Software Infrastructure for Sustained Innovation (SI2) PI workshop in order to email out a software metrics survey to all SI2 and related awardees as a targeted and relevant set of stakeholders.  This survey would be created by one of the student group members.  Similarly, it was suggested that a software metrics survey be sent to the UK SFTF and TRDF software projects to ask them what metrics would be useful to report.  The remainder of the discussion focused mainly on the creation of a white paper on this topic.  This resulted in a paper outline and writing assignments with the goal of publishing in venues including WSSSPE4, IEEE CISE, or JORS.



\subsection{Description of Opportunity, Challenges, and Obstacles}

The following opportunities, challenges, and obstacles were discussed:

\begin{itemize}
\item
Metrics are important for:

\begin{itemize}
\item
        Tenure and promotion

\item
        Scientific impact

\item
        Discovery

\item
        Reducing duplication

\item
        Basis for potential industrial interest in adopting software

\item
        Make case for funding
\end{itemize}

\item
No commonly-used standard for collecting or presenting metrics

\item
We don't know if there is a common set of metrics

\item
We have to persuade projects that it is useful to collect metrics

\end{itemize}



\subsection{Key Next Steps}

The following next steps were discussed:

\begin{itemize}
\item
Skype phone call to coordinate shortly after the conclusion of the WSSSPE3 workshop

\item
Get started on IRB at University of Illinois Urbana-Champaign in anticipation of SI2 project survey (may need more thought into survey)

\item
Get started on white paper and associated survey

\end{itemize}



\subsection{Plan for Future Organization}

The following plan for future organization was discussed:

\begin{itemize}
\item
Our group has created a white paper outline with sections assigned to the above individuals, plus see Section 2 response above for timeline

\item
Organizing coordinating phone calls

\end{itemize}




\subsection{What Else is Needed?}

The following list of what else is needed was discussed:

\begin{itemize}
\item
IRB approval/exemption needed for surveys, collecting data

\item
Coordination with 2016 NSF SI2 PI workshop organizing committee to possibly piggyback on this event to offer survey to attendees in advance

\item
Coordination (mail communication, info page etc), via WSSSPE github or?

\end{itemize}



\subsection{Key Milestones and Responsible Parties}

The following items were discussed as a roadmap for the production of a white paper:

\begin{enumerate}
\item
October -- November 2015: IRB paperwork as appropriate completed (Gabrielle Allen and Emily Chen)

\item
October -- December 2015: Draft white paper sections 1-3 (the paper outline has initial writing assignments)

\item
October -- December 2015: Run surveys and collect information

\begin{enumerate}
\item
        Piggyback on planning for 2016 NSF SI2 PIs meeting to be held Feb 16-17, 2016
\end{enumerate}

\item
January -- February 2016: Analyze results of data collection from projects

\item
March -- April 2016: Draft sections 4-7 of the white paper

\item
May 2016: Draft section 8-9 of the white paper

\item
May -- June 2016: Get initial feedback from members of the community and revise

\item
Est. July 2016: By time of next CFP for WSSSPE have complete draft of white paper

\item
Est. Sept -- Oct 2016: Responses to white paper submitted to WSSSPE4

\end{enumerate}


\subsection{Description of Funding Needed}

Funding needs were not discussed in this working group and it was thought that this could potentially be revisited down the road.




%%%%%%%%%%%%%%%%%%%%%%%%%%%%%%%%%%%%%%%%%%%%%%%%%%%%%%%%%%%%
\subsection{Training Working Group Discussion}
\label{sec:appendix_training}
%%%%%%%%%%%%%%%%%%%%%%%%%%%%%%%%%%%%%%%%%%%%%%%%%%%%%%%%%%%%

Nick Jones\footnote{email:
\href{mailto:nick.jones@nesi.org.nz}{nick.jones@nesi.org.nz}} will serve as the
point of contact for this working group, and be responsible for ensuring timely
progress of the planned actions.

\subsubsection{Group Members}
\begin{itemize}
\item Nick Jones -- New Zealand eScience Infrastructure
\item Iain Larmour -- Engineering \& Physical Sciences Research Council, UK
\item Erin Robinson -- Foundation for Earth Science
\end{itemize}

\subsubsection{Summary of Discussion}


While little training focuses specifically on sustainable
software, a variety of training activities could increase researcher awareness of
and engagement with software professionals and software engineering practices.
Research Software Engineers are being recognized as critical contributors to
high quality research; the pathway to acquire and master the relevant skills
is not yet clear; equally those skills required by researchers in general are
also not commonly understood nor routinely developed.

The group's discussion explored a rapidly growing array of training that is seen
to contribute to sustainable software. The offerings are diverse, including:
self-paced online modules focused around specific tools; single and multiple day
training workshops that raise awareness of a tool chain to support collaborative
and shared software development within a research workflow; block courses
specializing on particular methods, technologies, and applications; academic
programs at undergraduate and masters levels; doctoral training programs that in
part contain requisite skills training activities.

While some of this training focuses on applying software engineering practices
within the context of research, meeting the values and goals of research are
less often incorporated as explicit learning outcomes. With software (and
similarly, data) often being the only tangible artifact of a research method or
protocol, the dependency between software applications and the quality of
research adds complexity to the learner's journey. In recognition of the longer
term investment required by researchers to integrate such skills into their
research practices, many activities are focusing on emotionally engaging
researchers and cohorts, to build a sense of shared purpose beyond the obvious
goal of technical skill acquisition.

In reviewing current training activities, the group identified a variety of
perspectives seen as useful in positioning activities in ways to better
communicate why and when best to apply each activity. Training can be
categorized on a variety of spectra, with content and delivery ranging across them, for example:
programming to research; basic to advanced; technical to emotional; informal to
formal; and self-paced to participative. A few attempts have been made to situate a
cross section of training activities within such dimensions, creating easier
means of communicating the value of any specific opportunity and the pathways
across opportunities over time.

Evaluation of training delivery and outcomes is seen as a weakness common to
most non-academic training activities. Opportunities for measuring success in
delivering training start simply with collecting a Net Promoter Score, which
lets those delivering training know whether attendees are likely to recommend
the training to others. In looking at the longer term outcomes for the learner,
frameworks such as Bloom's taxonomy and Kirkpatrick's evaluation model offer possible
approaches.

In this latter case of formal evaluation, ownership of evaluation as a component
of career development for any researcher appears mostly absent. While academic
research institutions have professional development centers to support research
staff, the skills taught which might impact on sustainable software are limited
at best, and lack a clear and coherent development pathway.

Coordination of these training projects will depend on buy-in from a broad range
of training program and activity leaders, suggesting a key opportunity lies in
identifying and bringing together these people on a regular basis.

\subsubsection{Description of Opportunity, Challenges, and Obstacles}

Software skills are needed by an increasing array of researchers and fields. The
training arc is not well-defined, with a sometimes baffling array of training
opportunities responding to various facets of skill deficit and need. Given this
current complexity, coordination across training projects would create common
frames of reference, communicating and integrating activities to better serve
the needs of researchers.

Building this community could lift the maturity of training projects and
capabilities, enabling more advanced approaches to address key gaps in
evaluation, career development, and a lift in the standard of research
practices.

In aiming at these opportunities, it will be necessary to find the means to
support those involved in leading training activities to allocate time to
coordination activities, which will often sit beyond their current scope of
responsibility.

These activities are also distributed globally, with no single country or region
offering a comprehensive set of capabilities and initiatives. Any coordination
activity will therefore need to raise the profile of the opportunity gap with
relevant research funders and policy makers.

\subsubsection{Key Next Steps}

The goal of the following next steps is to quickly test whether there is
interest in establishing a community committed to increasing the degree of
coordination across training projects.

\begin{enumerate}

\item Hold a virtual meeting by December 2015, to bring together a broader group
of interest in this topic, with specific goals to:

	\begin{enumerate}
	    
	\item Identify programs with existing activities aimed at integrating across
	training projects.
	        
	\item Identify training projects with an interest in participating in
	coordination efforts.
	        
	\item Identify funding opportunities to bring together training program and
	project leaders to identify shared goals for future coordination of activities.
	        
	\item Agree on a communications plan to qualify whether programs, projects, and
	funders are interested in engaging and committing to ongoing activities.
	        
	\end{enumerate}
    
\item Review progress within 3 months, to establish next steps, if any.

\end{enumerate}

\subsubsection{Plan for Future Organization}

Continue to track progress by posting comments to WSSSPE3 issue.

\subsubsection{What Else is Needed?}

If the group moves from early-stage formation into working towards shared goals,
expertise will likely be required in pedagogy and training evaluation.

\subsubsection{Key Milestones and Responsible Parties}
\begin{enumerate}

\item October through December, Nick Jones and Erin Robinson to draft WSSSPE3 report back.

\item Before February 2016, Nick Jones and Erin Robinson to call a meeting of
the broader group, to review key next steps.
    
\item Second quarter 2016 -- if willing parties are identified, draft workshop proposal
and identify a relevant forum, including future WSSSPE events.
    
\end{enumerate}

\subsubsection{Description of Funding Needed}

Workshop/RCN travel funding to bring together key program, project, and funder
representatives from across North America, EU, UK, Australasia. In addition,
funding to support work on better defining the landscape of training activities,
the useful perspectives in communicating the value of the varied training
projects, and the possible pathways through training activities over time.

%%%%%%%%%%%%%%%%%%%%%%%%%%%%%%%%%%%%%%%%%%%%%%%%%%%%%%%%%%%%
\section{Software Credit Working Group Discussion}
\label{sec:appendix_SW_credit}
%%%%%%%%%%%%%%%%%%%%%%%%%%%%%%%%%%%%%%%%%%%%%%%%%%%%%%%%%%%%

\subsection{Group Members}
{\small
\begin{longtable}{ll}
   Alice Allen            &  Astrophysics Source Code Library (ASCL)
\\ Sou-Cheng Choi         &  
\\ James Hetherington     &  
\\ Lorraine Hwang         &  
\\ Daniel S.\ Katz        &  University of Chicago \& Argonne National Laboratory
\\ Frank Löffler          &  
\\ Abby Cabunoc Mayes     & 
\\ Kyle E.\ Niemeyer      &  Oregon State University
\\ Grace Peng             &  
\\ Ilian Todorov          &  
\end{longtable}
}

\subsection{Summary of Discussion}

\subsection{Description of Opportunity, Challenges, and Obstacles}


\subsection{Key Next Steps}


\subsection{Plan for Future Organization}


\subsection{What Else is Needed?}


\subsection{Key Milestones and Responsible Parties}


\subsection{Description of Funding Needed}

%%%%%%%%%%%%%%%%%%%%%%%%%%%%%%%%%%%%%%%%%%%%%%%%%%%%%%%%%%%%
\section{Publishing Software Working Group Discussion}
\label{sec:appendix_publishing_SW}
%%%%%%%%%%%%%%%%%%%%%%%%%%%%%%%%%%%%%%%%%%%%%%%%%%%%%%%%%%%%

\todo{add POC here}

\subsection{Group Members}

\begin{itemize}
\item Steven R. Brandt -- Louisiana State University
\item Daniel Gunter -- LBNL
\item Yuhan Ding -- Illinois Institute of Technology
\item Neil Chue Hong -- Software Sustainability Institute
\end{itemize}

\subsection{Summary of Discussion}

A tentative first cut at the list containing executable papers identified the following:

\begin{itemize}
\item ACM Transactions on Mathematical Software (TOMS) - which provides the extra step
 of having reviewers validate the code which was submitted with the publication.
\item The Mathematica Journal - which publishes Mathematica notebooks (with equations,
figures, etc.) directly.
\item O'Reily Media has announced that it plans to make IPython Notebooks a first-class
 authoring environment for their publishing program alongside their existing mechanisms.
\item Nature is offering a list of notebooks published alongside more traditional articles,
 and is looking at ways to make these documents more official. There are, in fact, a
 number of journals that offer "electronic supplements" to the more traditionally published
 static articles.
\item There is also a list of "reproducible academic publications" maintained here:
  \url{https://github.com/ipython/ipython/wiki/A-gallery-of-interesting-IPython-Notebooks#reproducible-academic-publications}
\item KBase narratives build on IPython/Jupyter notebooks to build publications that are
  reproducible, and can be commented or annotated.
\end{itemize}

The group also discussed future possibilities for this new publication format which might
provide advantages.

\begin{itemize}
\item Journals could be built around an existing, widely-used framework thereby reducing
  the burden of studying code on the part of reviewers (common bits of infrastructure
  which aren't relevant to the science would be automatically excluded).
\item Journals might be encouraged to use more metadata, making them easier to mine
  for various analytical purposes.
\item RIO is an effort to publish fragmentary results that can subsequently be combined
  into a single content item.
\item Papers could be made more understandable. Each equation or technical term could
  offer a link to a document/tutorial explaining its origin and/or derivation.
\item So many options for publication currently exist that good science may be getting
  lost in the noise. Would some sort of "upvote" mechanism be of value?
\item Some sort of Replicated Computation Results badge could be made available to
  publications that have undergone greater scrutiny (this is already done by TOMS).
\end{itemize}

\subsection{Description of Opportunity, Challenges, and Obstacles}

The opportunity is to collect a list of current executable papers and
shine a light on the experiments and development efforts currently underway.

The only obstacle to this is the difficulty in finding and identifying such
publications. The Software Sustainability Institute was able to do something similar
for publications about software by making a public page containing a catalog
of these publications and enlisting the help of the community to grow the list.

\subsection{Key Next Steps}

Create the first version of the web page to be displayed on the Software Sustainability
Institute's website.

\subsection{Plan for Future Organization}

None at this time.

\subsection{What Else is Needed?}

Nothing else at this time.

\subsection{Key Milestones and Responsible Parties}

Steven R. Brandt will create a first version of the page within a week or so of the WSSSPE3 conference.

Neil Chue Hong will take responsibility for the page once it is up.

\subsection{Description of Funding Needed}

None.

%%%%%%%%%%%%%%%%%%%%%%%%%%%%%%%%%%%%%%%%%%%%%%%%%%%%%%%%%%%%
\section{User Community Working Group Discussion}
\label{sec:appendix_user_community}
%%%%%%%%%%%%%%%%%%%%%%%%%%%%%%%%%%%%%%%%%%%%%%%%%%%%%%%%%%%%

\subsection{Group Members}

\begin{itemize}
\item name -- affiliation
\item name2 -- affiliation2
\end{itemize}

\subsection{Summary of Discussion}

\subsection{Description of Opportunity, Challenges, and Obstacles}


\subsection{Key Next Steps}


\subsection{Plan for Future Organization}


\subsection{What Else is Needed?}


\subsection{Key Milestones and Responsible Parties}


\subsection{Description of Funding Needed}


\bibliographystyle{vancouver}

\bibliography{wssspe}
\end{document}

