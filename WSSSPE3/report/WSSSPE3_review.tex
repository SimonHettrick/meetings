\documentclass[11pt, oneside]{amsart}
\pdfoutput=1

\usepackage{amsmath}
\usepackage{amssymb}

\usepackage{color}
\usepackage{dcolumn}
\usepackage{float}
\usepackage{graphicx}
\usepackage[utf8]{inputenc}
\usepackage[T1]{fontenc}
\usepackage{lmodern}
\usepackage{multirow}
\usepackage{rotating}
\usepackage{subfigure}
\usepackage{psfrag}
\usepackage{tabularx}
\usepackage[hyphens]{url}
\usepackage{wrapfig}
\usepackage{longtable}
\usepackage{verbatim}

% The following three lines are used for displaying footnote in tables.
\usepackage{footnote}
\makesavenoteenv{tabular}
\makesavenoteenv{table}


\usepackage{enumitem}
\setlist{leftmargin=7mm}

%\setcounter{secnumdepth}{3}
%\setcounter{tocdepth}{3}


\usepackage[bookmarks, bookmarksopen, bookmarksnumbered]{hyperref}
\usepackage[all]{hypcap}
\urlstyle{rm}

\definecolor{orange}{rgb}{1.0,0.3,0.0}
\definecolor{violet}{rgb}{0.75,0,1}
\definecolor{darkgreen}{rgb}{0,0.6,0}
\definecolor{cyan}{rgb}{0.2,0.7,0.7}
\definecolor{blueish}{rgb}{0.2,0.2,0.8}

\newcommand{\todo}[1]{{\color{blue}$\blacksquare$~\textsf{[TODO: #1]}}}
\newcommand{\note}[1]{ {\textcolor{blueish}    { ***Note:      #1 }}}
\newcommand{\katznote}[1]{ {\textcolor{magenta}    { ***Dan:      #1 }}}
\newcommand{\clunenote}[1]{ {\textcolor{orange}    { ***Tom:      #1 }}}
\newcommand{\gabnote}[1]{ {\textcolor{cyan}    { ***Gabrielle:     #1 }}}
\newcommand{\nchnote}[1]{  {\textcolor{orange}      { ***Neil: #1 }}}
\newcommand{\manishnote}[1]{  {\textcolor{violet}     { ***Manish: #1 }}}
\newcommand{\davidnote}[1]{  {\textcolor{darkgreen}      { ***David: #1 }}}
\newcommand{\colinnote}[1]{ {\textcolor{red}    {***Colin: #1 }}}
\newcommand{\choinote}[1]{ {\textcolor{orange}    {***Choi: #1 }}}

% Don't use tt font for urls
\urlstyle{rm}

% 15 characters / 2.5 cm => 100 characters / line
% Using 11 pt => 94 characters / line
\setlength{\paperwidth}{216 mm}
% 6 lines / 2.5 cm => 55 lines / page
% Using 11pt => 48 lines / pages
\setlength{\paperheight}{279 mm}
\usepackage[top=2.5cm, bottom=2.5cm, left=2.5cm, right=2.5cm]{geometry}
% You can use a baselinestretch of down to 0.9
\renewcommand{\baselinestretch}{0.96}

\sloppypar

\begin{document}

\title[]{Report on the Third Workshop on Sustainable Software for Science: Practice and Experiences (WSSSPE3)}

\author{tbd by writing and organizing}

%\author{Daniel S. Katz$^{(1)}$, Sou-Cheng T. Choi$^{(2)}$, Nancy Wilkins-Diehr$^{(3)}$, Neil Chue Hong$^{(4)}$,
%\\Colin C. Venters$^{(5)}$, James Howison$^{(6)}$, Frank Seinstra$^{(7)}$, Matthew Jones$^{(8)}$,
%\\Karen Cranston$^{(9)}$, Thomas L. Clune$^{(10)}$, Miguel de Val-Borro$^{(11)}$, Richard Littauer$^{(12)}$}
%%
%\thanks{{}$^{(1)}$ Computation Institute, 
%University of Chicago \& Argonne National Laboratory, Chicago, IL, USA; \url{dsk@uchicago.edu}}
%%
%\thanks{{}$^{(2)}$ NORC at the University of Chicago and Illinois Institute of Technology, Chicago, IL, USA; \url{sctchoi@uchicago.edu}}
%%
%\thanks{{}$^{(3)}$ University of California-San Diego, San Diego, CA, USA; \url{wilkinsn@sdsc.edu}}
%%
%\thanks{{}$^{(4)}$ Software Sustainability Institute, 
%University of Edinburgh, Edinburgh, UK; \url{N.ChueHong@software.ac.uk}}
%%
%\thanks{{}$^{(5)}$ University of Huddersfield, School of Computing and Engineering, Huddersfield, UK; \url{C.Venters@hud.ac.uk}}
%%
%\thanks{{}$^{(6)}$ University of Texas at Austin, Austin, TX, USA; \url{jhowison@ischool.utexas.edu}}
%%
%\thanks{{}$^{(7)}$ Netherlands eScience Centre, Amsterdam, Netherlands; \url{F.Seinstra@esciencecenter.nl}}
%%
%\thanks{{}$^{(8)}$ National Center for Ecological Analysis and Synthesis, Santa Barbara, CA, USA; \url{jones@nceas.ucsb.edu}}
%%
%\thanks{{}$^{(9)}$ National Evolutionary Synthesis Center, Durham, NC, USA; \url{karen.cranston@nescent.org}}
%%
%\thanks{{}$^{(10)}$ NASA Goddard Space Flight Center, Greenbelt, MD, USA; \url{Thomas.L.Clune@nasa.gov}}
%%
%\thanks{{}$^{(11)}$ Department of Astrophysical Sciences, 
%Princeton University, Princeton, NJ, USA; \url{mdevalbo@astro.princeton.edu}}
%%
%\thanks{{}$^{(12)}$ University of Saarland, Germany; \url{richard.littauer@gmail.com}}
%%
 

\begin{abstract}
\todo{need to update for WSSSPE3}
This technical report records and discusses the Second Workshop on Sustainable
Software for Science: Practice and Experiences (WSSSPE2). 
%The workshop used an
%alternative submission and peer-review process, which led to a set of papers
%divided across five topic areas: 
The report includes a description of the alternative, experimental submission
and review process, two workshop keynote presentations, a series of lightning
talks, a discussion on sustainability, and five discussions from the topic areas
of exploring sustainability; software development experiences; credit \&
incentives; reproducibility \& reuse \& sharing; and code testing \& code
review. For each topic, the report includes a list of tangible actions that were
proposed and that would lead to potential change.
%
The workshop recognized that reliance on scientific software
is pervasive in all areas of world-leading research today. The workshop
participants then proceeded to explore different perspectives on the concept of
sustainability. Key enablers and barriers of sustainable scientific software
were identified from their experiences. In addition,
recommendations with new requirements such as software credit files and software
prize frameworks were outlined for improving practices in sustainable software
engineering.
%
There was also broad consensus that formal
training in software development or engineering was rare among the
practitioners. Significant strides need to be made in building a sense of
community via training in software and technical practices, on increasing their
size and scope, and on better integrating them directly into graduate education
programs.
%
Finally, journals can define and publish policies to improve reproducibility, whereas
reviewers can insist that authors provide sufficient information and access to
data and software to allow them reproduce the results in the paper. Hence a list of
criteria is compiled for  journals to provide to reviewers so as to make it easier to
review software submitted for publication as a ``Software Paper.''

\end{abstract}


\maketitle
%\newpage

%%%%%%%%%%%%%%%%%%%%%%%%%%%%%%%%%%%%%%%%%%%%%%%%%%%%%%%%%%%%
\section{Introduction} \label{sec:intro}
%%%%%%%%%%%%%%%%%%%%%%%%%%%%%%%%%%%%%%%%%%%%%%%%%%%%%%%%%%%%

%\katznote{example comment by Dan}
%
%\gabnote{example comment by Gabrielle}
%
%\nchnote{example comment by Neil}
%
%\manishnote{example comment by Manish}
%
%\davidnote{example comment by David}

%\note{google doc of notes for reference: \url{http://tinyurl.com/q6ew45v}}
%https://docs.google.com/document/d/1-BxkYWDQ6nNNBXBStUL0xcKF9qCTlEALwf928J_MemI/edit?usp=sharing

\todo{need to update for WSSSPE3}
The Second Workshop on Sustainable Software for Science: Practice and
Experiences
(WSSSPE2)\footnote{\url{http://wssspe.researchcomputing.org.uk/wssspe2/}} was
held on 16 November, 2014 in the city of New Orleans, Louisiana, USA, in
conjunction with the International Conference for High Performance Computing,
Networking, Storage and Analysis
(SC14)\footnote{\url{http://sc14.supercomputing.org}}. WSSSPE2 followed the
model of a general initial workshop,
WSSSPE1\footnote{\url{http://wssspe.researchcomputing.org.uk/wssspe1/}}~\cite{WSSSPE1-pre-report,WSSSPE1},
which co-occurred with SC13, and a focused workshop,
WSSSPE1.1\footnote{\url{http://wssspe.researchcomputing.org.uk/wssspe1-1/}},
which was organized in July 2014 jointly with the SciPy
conference\footnote{\url{https://conference.scipy.org/scipy2014/participate/wssspe/}}.

Progress in scientific research is dependent on the quality and accessibility of
software at all levels. Hence it is critical to address challenges related to
the development, deployment, maintenance, and overall sustainability of reusable
software as well as education around software practices. These challenges can be
technological, policy based, organizational, and educational, and are of
interest to developers (the software community), users (science disciplines),
software-engineering researchers, and researchers studying the conduct of
science (science of team science, science of organizations, science of science
and innovation policy, and social science communities). The WSSSPE1 workshop
engaged the broad scientific community to identify challenges and best practices
in areas of interest to creating sustainable scientific software. WSSSPE2
invited the community to propose and discuss specific mechanisms to move towards
an imagined future practice for software development and usage in science and
engineering. The workshop included multiple mechanisms for participation,
encouraged team building around solutions, and identified risky solutions with
potentially transformative outcomes. It strongly encouraged participation of
early-career scientists, postdoctoral researchers, and graduate students, with
funds provided to the conference organizers by the Moore Foundation and the
National Science Foundation (NSF), to support the travel of potential
participants who would not otherwise be able to attend, and young participants
and those from underrepresented groups, respectively. These funds allowed 17
additional participants to attend, and each was offered the chance to present a
lightning talk.

This report extends a previous report that discussed the submission,
peer-review, and peer-grouping processes in detail~\cite{WSSSPE2-pre-report}. It
is also based on a collaborative set of notes taken with Google Docs during the
workshop~\cite{WSSSPE2-google-notes}. Overall, the report discusses the
organization work done before the workshop (\S\ref{sec:preworkshop}); the
keynotes (\S\ref{sec:keynotes}); a series of lightning talks
(\S\ref{sec:lightning}), intended to give an opportunity for attendees to
quickly highlight an important issue or a potential solution; a session on
defining sustainability (\S\ref{sec:defining}). The report also gives summaries
of action plans proposed by five breakout sessions, which explored in specific
areas including sustainability (\S\ref{sec:exploring}); software development
experiences (\S\ref{sec:devel}); credit \& incentives (\S\ref{sec:credit});
reproducibility, reuse, \& sharing (\S\ref{sec:reproduce}); code testing \& code
review (\S\ref{sec:code_testing}). Lastly, the report also includes some
conclusions (\S\ref{sec:conclusions}) and an incomplete list of attendees
(Appendix~\ref{sec:attendees}).



%%%%%%%%%%%%%%%%%%%%%%%%%%%%%%%%%%%%%%%%%%%%%%%%%%%%%%%%%%%%
\section{Submissions, Peer-Review, and Peer-Grouping} \label{sec:preworkshop}
%%%%%%%%%%%%%%%%%%%%%%%%%%%%%%%%%%%%%%%%%%%%%%%%%%%%%%%%%%%%

%\note{this section is taken from \cite{WSSSPE2-pre-report}. It could be shortened.}

\todo{need to update for WSSSPE3}
WSSSPE2 began with a call for papers~\cite{WSSSPE2-pre-report}. Based on the
goal of encouraging a wide range of submissions from those involved in software
practice, ranging from initial thoughts and partial studies to mature
deployments, but focusing on papers that are intended to lead to changes, the
organizers wanted to make submission as easy as possible. The call for papers
stated:

\begin{quote} We invite short (4-page) \textbf{actionable} papers that will lead
to improvements for sustainable software science. These papers could be a call
to action, or could provide position or experience reports on sustainable
software activities. The papers will be used by the organizing committee to
design sessions that will be highly interactive and targeted towards
facilitating action. Submitted papers should be archived by a third-party
service that provides DOIs. We encourage submitters to license their papers
under a Creative Commons license that encourages sharing and remixing, as we
will combine ideas (with attribution) into the outcomes of the workshop.
\end{quote}

The call included the following areas of interest:
\begin{quote}
\begin{itemize}
\renewcommand{\labelenumi}{\textbf{\theenumi}.}
\setlength{\rightmargin}{1em}
\item defining software sustainability in the context of science and engineering
software
\begin{itemize}
\item how to evaluate software sustainability
\end{itemize}

\item improving the development process that leads to new software
\begin{itemize}
\item methods to develop sustainable software from the outset
\item effective approaches to reusable software created as a by-product of
research
\item impact of computer science research on the development of scientific
software
\end{itemize}

\item recommendations for the support and maintenance of existing software
\begin{itemize}
\item software engineering best practices
\item governance, business, and sustainability models
\item the role, operation, and
sustainability of community software repositories
\item reproducibility, transparency needs that may be unique to science
\end{itemize}

\item successful open source software implementations
\begin{itemize}
\item incentives for using and contributing to open source software
\item transitioning users into contributing developers
\end{itemize}

\item building large and engaged user communities
\begin{itemize}
\item developing strong advocates
\item measurement of usage and impact
\end{itemize}

\item encouraging industry's role in sustainability
\begin{itemize}
\item engagement of industry with volunteer communities
\item incentives for industry
\item incentives for community to contribute to industry-driven projects
\end{itemize}

\item recommending policy changes
\begin{itemize}
\item software credit, attribution, incentive, and reward
\item issues related to multiple organizations and multiple countries, such as
intellectual property, licensing, etc.
\item mechanisms and venues for publishing software, and the role of publishers
\end{itemize}

\item improving education and training
\begin{itemize}
\item best practices for providing graduate students and postdoctoral
researchers in domain communities with sufficient training in software
development
\item novel uses of sustainable software in education (K-20)
\item case studies from students on issues around software development in the
undergraduate or graduate curricula
\end{itemize}

\item careers and profession
\begin{itemize}
\item successful examples of career paths for developers
\item institutional changes to support sustainable software such as promotion
and tenure metrics, job categories, etc.
\end{itemize}

\end{itemize}

\end{quote}


31 submissions were received; all but one used arXiv\footnote{\url{http://arxiv.org}}
or figshare\footnote{\url{http://figshare.com}} to self-publish their papers.

The review process was fairly standard. First, reviewers bid for papers. Then an
automated system matched the bids to determine assignments. After the reviewers
completed their assigned reviews (with an average of 4.9 reviews per paper and
4.1 reviews per reviewer), they used EasyChair\footnote{\url{http://easychair.org/}} 
to record scores on relevance
and comments. Finally, the organizers accessed the information to decide which
papers to associate with the workshop and provided authors with the comments to
help them improve their papers.

The organizers decided to list 28 of the papers as significantly contributing to
the workshop, a very high acceptance rate, but one that is reasonable, given the
goal of broad participation and the fact that the reports were already
self-published.

The organizers wanted very interactive sessions, with the process of creating
the sessions open to the full program committee, the paper authors, and others
who might attend the workshop. In order to do this, the organizers used 
Well Sorted\footnote{\url{http://www.well-sorted.org}} with the following steps:
\begin{enumerate}
%
\item Authors were asked to create Well Sorted ``cards'' for the papers. These
cards have a title (50 characters maximum) and a body (255 characters maximum).
%
\item Authors, members of the WSSSPE program committee, and mailing list subscribers
were asked to sort the cards. Each person dragged the cards, one by one, into
groups. A group could have as many cards as the person wanted it to have, and it
could have any meaning that made sense to that person.
%
\item Well Sorted produced a set of averages of all the sorts, with
various numbers of card clusters.
%
\end{enumerate}

The organizers then chose a sort that contained five groups that felt most
meaningful. After that, they decided on names for the five groups:
\begin{itemize}
\item Exploring Sustainability
\item Software Development Experiences
\item Credit \& Incentives
\item Reproducibility \& Reuse \& Sharing
\item Code Testing \& Code Review.
\end{itemize}

Finally, since some of the papers were not represented by cards in the process,
they were not placed in groups by the peer-grouping system. The authors of
these papers were asked which groups seemed the best for their papers; these
papers were then placed in those groups. Sections~\ref{sec:exploring}-\ref{sec:code_testing}
discuss the breakout groups, including a list of the papers associated with each
group.


%%%%%%%%%%%%%%%%%%%%%%%%%%%%%%%%%%%%%%%%%%%%%%%%%%%%%%%%%%%%
\section{Keynote} \label{sec:keynote}
%%%%%%%%%%%%%%%%%%%%%%%%%%%%%%%%%%%%%%%%%%%%%%%%%%%%%%%%%%%%
\todo{need to update for WSSSPE3}
The workshop featured two keynote addresses. In the opening keynote
presentation, Kaitlin Thaney of the Mozilla Science Lab talked about her
organization's work and policy to enable and support sustainable and
reproducible scientific research through the open web. The second keynote
speaker was Neil Chue Hong of Software Sustainability Institute. He shed
light on how scientific software is prevalently driving advances in many science
and engineering fields. Both keynote speeches spawned further discussion among
workshop participants on the crucial notion of \emph{software sustainability}
in the theme of our workshop.


Kaitlin Thaney is the Director of the Mozilla Science Lab (hereafter Mozilla), which
is a non-profit organization interested in openness, news, website
creation, and Science, all taking advantage of the open web.

Thaney started noting the unfortunate fact that many current systems suffer the
unintended consequence of creating friction that hinders users, despite
designers' original purposes to do good. An example is the National Cancer
Institute's caBIG. A total of \$350 million was spent, including more than \$60
million for management. More than $70$ tools were created, but caBIG is still
seen as a failure\footnote{Report Blasts Problem-Plagued Cancer Research Grid,
\url{http://tinyurl.com/maf6dz2}}. Those that had the least investment were the
most used; the most invested software were the least utilized.

Thaney emphasized that for efficient reproducible open research, we would need
research tools (e.g., software repositories), social capital (e.g., incentives),
and capacity (e.g., training and mentorship). Our systems would need to
communicate with each other. A point was made by a member of the audience that
as systems become less monolithic, it often becomes harder to sustain the links
between them\footnote{See, for example, \url{http://tinyurl.com/l76tba2}.}.
%http://www.slideshare.net/jameshowison/scientific-software-sustainability-and-ecosystem-complexity
%(but does Anon Grizzly have other cites/info for that?).
%In contrast, works licensed by Creative Commons typically do not
%have dependencies (a book, a photo, an artwork).

Thaney spoke about Mozilla's work around code citation, through a collaboration
and prototype crafted between Mozilla, GitHub, figshare, and Zenodo. This work
was presented at a closed meeting in May 2014 at the National Institutes of
Health (NIH) around these issues, sparking a conversation from that meeting
around what a \emph{Software Discovery
Index}\footnote{Software Discovery Index, \url{http://softwarediscoveryindex.org/report/}} might look
like. The meeting included a number of publishers, researchers, and those behind
major scientific software efforts such as
Bioconductor\footnote{Bioconductor, \url{http://www.bioconductor.org}},
Galaxy\footnote{Galaxy, \url{http://galaxyproject.org}}, and
nanoHUB\footnote{nanoHUB, \url{https://nanohub.org}}.
%facilitates more efficient scientific research. SDI identifies scientific
%software by archiving and standardizing metadata for software and hence help
%connect both developers and users.
Ted Habermann in the audience commented that if the metadata is minimal, it
would be less onerous for data providers, but more burdensome for users---it
could be challenging to keep a balance between what have to be captured and what
would be ideal if we do not want to lose user engagement as in the case of the
old Harvard Dataverse, finding often only the first four fields of three pages
of metadata were filled out.

The speaker concluded her talk urging the audience to design scientific software
with the general community, not an individual, in mind; and to design to unlock
latent potential of our systems. In addition, she encouraged everyone to rethink
how we reward researchers and support roles.
%Lastly, she cautioned the community to be mindful of jargon or semantics traps.



%%%%%%%%%%%%%%%%%%%%%%%%%%%%%%%%%%%%%%%%%%%%%%%%%%%%%%%%%%%%
\section{Lightning Talks} \label{sec:lightning}
%%%%%%%%%%%%%%%%%%%%%%%%%%%%%%%%%%%%%%%%%%%%%%%%%%%%%%%%%%%%
\todo{need to update for WSSSPE3}
\begin{comment}
\note{
\href{http://wssspe.researchcomputing.org.uk/wssspe3/agenda/}{Slides.}}
\end{comment}

\begin{enumerate}
\item \textbf{Benjamin Tovar and Douglas Thain: \textit{Freedom vs. Stability:
Facilitating Research Training While Supporting Scientific Research}}

\item \textbf{Birgit Penzenstadler, Colin Venters, Christoph Becker, Stefanie
Betz, Ruzanna Chitchyan, Letícia Duboc, Steve Easterbrook, Guillermo
Rodriguez-Navas and Norbert Seyff: \textit{Manifesting the Ghost of the Future:
Sustainability}}

\item \textbf{Abani Patra, Hossein Aghakhani, Nikolay Simakov, Matthew D. Jones
and Tevfik Kosar: \textit{Integrating New Functionality Using Smart Interfaces to
Improve Productivity of Legacy Tools}}

\item \textbf{Abigail Cabunoc Mayes, Bill Mills, Arliss Collins and Kaitlin
Thaney: \textit{Collaborative Software Development as Sustainable Software: Lessons
from Open Source}}

\item \textbf{Louise Kellogg and Lorraine Hwang: \textit{Advancing Earth Science
through Best Practices in Open Source Software: Computational Infrastructure
for Geodynamics}}

\item \textbf{Lorraine Hwang, Joe Dumit, Alison Fish, Louise Kellogg, Mackenzie
Smith and Laura Soito: \textit{Software Attribution for Geoscience Applications in the
Computational Infrastructure for Geodynamics}}

\item \textbf{Mike Hildreth, Jarek Nabrzyski, Da Huo, Peter Ivie, Haiyan Meng,
Douglas Thain and Charles Vardeman: \textit{Data And Software Preservation for Open
Science (DASPOS)}}

\item \textbf{James Hetherington, Jonathan Cooper, Robert Haines, Simon
Hettrick, James Spencer, Mark Stillwell, Mike Croucher, Christopher Woods and
Susheel Varma: \textit{Research Software Engineering Groups in Universities: The Story
from the UK}}

\item \textbf{Dan Gunter, Sarah Poon and Lavanya Ramakrishnan: \textit{Bringing the
User into Building Sustainable Software for Science}}

\item \textbf{Dan Gunter, Adam Arkin, Rick Stevens, Robert Cottingham and
Sergei Maslov: \textit{Challenges of a Sustainable Software Platform for Predictive
Biology: Lessons Learned on the KBase Project}}

\item \textbf{Yolanda Gil, Chris Duffy, Chris Mattmann, Erin Robinson and Karan
Venayagamoorthy: \textit{The Geoscience Paper of the Future Initiative: Training
Scientists in Best Practices of Software Sharing}}

\item \textbf{Neil Chue Hong: \textit{Building a Scientific Software Accreditation
Framework}}

\item \textbf{Jeffrey Carver: \textit{On the Need for Software Engineering Support for
Sustainable Scientic Software}}

\item \textbf{Matthias Bussonnier: \textit{User Data Collection in Open Source}}

\item \textbf{Alice Allen: \textit{We’re giving away the store! (Merchandise not
included)}}

\item \textbf{Stan Ahalt, Bruce Berriman, Maxine Brown, Jeffrey Carver, Neil
Chue Hong, Allison Fish, Ray Idaszak, Greg Newman, Dhabaleswar Panda, Abani
Patra, Elbridge Gerry Puckett, Chris Roland, Douglas Thain, Selcuk Uluagac, and
Bo Zhang: \textit{Scientific Software Success: Developing Metrics While Developing
Community}}

\end{enumerate}

%%%%%%%%%%%%%%%%%%%%%%%%%%%%%%%%%%%%%%%%%%%%%%%%%%%%%%%%%%%%
\section{Working Groups} \label{sec:WGs}
%%%%%%%%%%%%%%%%%%%%%%%%%%%%%%%%%%%%%%%%%%%%%%%%%%%%%%%%%%%%


%%%%%%%%%%%%%%%%%%%%%%%%%%%%%%%%%%%%%%%%%%%%%%%%%%%%%%%%%%%%
\section{Conclusions} \label{sec:conclusions}
%%%%%%%%%%%%%%%%%%%%%%%%%%%%%%%%%%%%%%%%%%%%%%%%%%%%%%%%%%%%

\todo{need to update for WSSSPE3}
The WSSSPE2 workshop continued our experiment from WSSSPE1 in how we can
collaboratively build a workshop agenda, and we began a new experiment in
how to build a series of workshops into an ongoing community activity.

The differences in workshop organization in WSSSPE2 from WSSSPE1
are in using an existing service (EasyChair) to handle submissions and reviews,
rather than an ad hoc process, and using an existing service (Well Sorted) to
allow collaborative grouping of papers into themes by all authors, reviewers,
and the community, rather than this being done in an ad hoc manner by the
organizers alone.

The fact remains that contributors also want to get credit for their
participation in the process. And the workshop organizers will want to make
sure that the workshop content and their efforts are recorded. Ideally, there
would be a service that would index the contributions to the
workshop, serving the authors, the organizers, and the larger community. 
Since there still isn't such a service today, the workshop organizers are
writing this initial report and making use of arXiv as a partial solution to
provide a record of the workshop.

\begin{table*}[t]
\centering
\caption{Top tweets tagged \#WSSSPE on Nov 16, 2014. \todo{update this for WSSSPE3?}}\label{tab:tweets}
  \begin{scriptsize}
  \begin{tabular}{l|l|r|r}
 \hline
    Author  &   Tweet  & Retweets &  Favorites
\\ \hline
% Software Carpentry & Nov 17 & You call it ``project planning'' if it hasn't started yet,          & 8 & 1
%\\                          & &  and``software sustainability'' if it has and wasn't planned.     &    &
%
Neil P Chue Hong   &  Here's @SoftwareSaved guidance on Writing and using a software & 7 & 4
\\     &  management plan used by EPSRC software grants    &    &
\\     &  \url{http://www.software.ac.uk/resources/guides/software-management-plans}  &    &
%
\\  Neil P Chue Hong  &  @jameshowison as well as software plans   & 4 & 7
\\   &   \url{http://www.software.ac.uk/resources/guides/software-management-plans} &    &
\\   &   we provide a software evaluation tool:   &    &
\\   &   \url{http://www.software.ac.uk/online-sustainability-evaluation}  &    &
%
\\ Tom Crick  & $56\%$ of UK researchers develop their own software $\rightarrow  140,000$   &  14 & 8
\\ & UK researchers write research software w$/$out any formal training &    &
%
\\ Karthik Ram  &  OH: ``Institutionalize metadata before metadata institutionalizes you'' & 8 & 6
%
\\ Josh Greenberg  &  @jameshowison: ``1. retract any paper with bitrotten dependencies'' *mic drop* & 13 & 8
\\   &   ``2. add anyone who fixes bitrot as an author'' *mic drop*  &    &
%
\\ Ethan White &  ``@rOpenSci is all about community... our measures of success  & 9 & 3
\\ &   [include] how many faces are up on our community page''   &    &
%
\\ Ethan White & Daniel Katz talking about implementing transitive credit for  & 9 & 7
\\ &  software \url{http://arxiv.org/abs/1407.5117}  Work with @arfon  &    &
%
\\ Kaitlin Thaney  & Great point by @tracykteal about planning for ``end of life'' with scientific & 4 & 8
\\ &  software projects and sustainability.  & &
%
%\\ Aleksandra Pawlik  &  ``Tell us how you test your scientific software'' @gvwilson @swcarpentry  & 7 & 2
%
\\ Aleksandra Pawlik  & Lack of training as one of the main barriers for sustainable software & 10 & 4
\\ &    at @Supercomputing. @swcarpentry @datacarpentry can fix that!  & &
%
\\ Kaitlin Thaney  & My slides from this morning's keynote at & 11 & 12
\\ &  WSSSPE on Designing for Truth, Scale $+$ Sustainability:  & &
\\ &  \href{http://www.slideshare.net/kaythaney/designing-for-truth-scale-and-sustainability-wssspe2-keynote}{http://www.slideshare.net/kaythaney/}     & &
\\ &  \href{http://www.slideshare.net/kaythaney/designing-for-truth-scale-and-sustainability-wssspe2-keynote}{designing-for-truth-scale-and-sustainability-wssspe2-keynote} & &
%
\\ Neil P Chue Hong & @kaythaney shout out for @swcarpentry @datacarpentry & 9 & 4
\\ &  @rOpenSci @stilettofiend around open training activities for sustainability  & &
%
%\\ Richard Littauer &  Sweet! @swcarpentry has had 4000+ learners in the past year. & 6 &
%
\\ Neil P Chue Hong & For those interested in Github - Figshare/Zenodo integration, & 5 & 12
\\ & but want SWORD/DSpace/Fedora/ePrints see:  & &
\\ & \url{http://blog.stuartlewis.com/2014/09/09/github-to-repository-deposit/}  & &
%
\\ Hilmar Lapp & Re: adopting the unix philosophy, consider signing the Small Tools in & 7 & 6
\\ & Bioinformatics Manifesto: \url{https://github.com/pjotrp/bioinformatics}   &
%
\\Andre Luckow &  ``Traditions last not because they are excellent, & 12 & 3
\\ & but because influential people are averse to change...''  C. Sunstein     & &
%
\\ Tom Crick &  ``Can I Implement Your Algorithm?'':  A Model for  Reproducible & 9 & 8
\\   &  Research Software \url{http://arxiv.org/abs/1407.5981}  & &
%
\\Mozilla Science Lab & At a loose end this Sunday? Care about reproducibility, software  $+$ & 10 & 5
\\ &  \#openscience? Follow the    \#WSSSPE hashtag for more, live from New Orleans.  &  &
%
\\Kaitlin Thaney & I'm in New Orleans at \#WSSSPE , speaking at 9:50 ET on  scientific software & 9 & 9
\\ &   $+$ sustainability. Tune in! Live stream: \url{http://ustre.am/17ddh} & &
%
\\   Daniel S. Katz & \#WSSSPE Agenda (Sunday):  & 10 & 1
\\ & \url{http://wssspe.researchcomputing.org.uk/wssspe2/agenda/}   &  &
\\ & URL for live stream of keynotes \& lightning talks: \url{http://ustre.am/17ddh}   &  &
\\ \hline
    \end{tabular}
    \end{scriptsize}
\end{table*}

WSSSPE actively used the online social network Twitter, with hashtag
``\#WSSSPE''. There were substantially more tweets (messages) during the days of
the workshops WSSSPE2, WSSSPE1.1, and WSSSPE1. Out of about 670 tweets as of Apr
18, 2015, more than 225 were about WSSSPE2 and about 180 were posted during the
day of the workshop. Some of the main points and highlights in the meeting are
shown in Table~\ref{tab:tweets}, which summarizes the top \#WSSSPE tweets from
the day of workshop, selected by the metrics that number of retweets or
favorites larger than five and the sum of two measures greater than ten.

In terms of building community activities, we wanted to focus primarily on
working groups, which we were able to do, as discussed above, but we
also wanted to make sure that attendees felt they had a chance to get their
ideas across to the whole group, which was the purpose of the lightning talks.
Overall, this seemed to be successful at the time, in terms of both the lightning
talks and the breakout groups, and the discussion of sustainability also led
to interesting and useful results. However, the challenge that we have discovered
since WSSSPE2 is that it is very hard to continue the breakout groups'
activities.  The WSSSPE2 participants were willing to dedicate their time to
the groups while they were at the meeting, but afterwards, they have gone
back to their (paid) jobs.  We need to determine how to tie the WSSSPE
breakout activities to people's jobs, so that they feel that continuing them
is a higher priority than it is now, perhaps through funding the participants,
or through funding coordinators for each activity, or perhaps by getting
the workshop participants to agree to a specific schedule of activities during the
workshop.


%%%%%%%%%%%%%%%%%%%%%%%%%%%%%%%%%%%%%%%%%%%%%%%%%%%%%%%%%%%%
\section*{Acknowledgments} \label{sec:acks}
%%%%%%%%%%%%%%%%%%%%%%%%%%%%%%%%%%%%%%%%%%%%%%%%%%%%%%%%%%%%

Work by Katz was supported by the National Science Foundation while working at
the Foundation. Any opinion, finding, and conclusions or recommendations
expressed in this material are those of the author(s) and do not necessarily
reflect the views of the National Science Foundation.

\todo{feel free to add stuff here}


\appendix
%%%%%%%%%%%%%%%%%%%%%%%%%%%%%%%%%%%%%%%%%%%%%%%%%%%%%%%%%%%%
\section{Attendees}  \label{sec:attendees}
%%%%%%%%%%%%%%%%%%%%%%%%%%%%%%%%%%%%%%%%%%%%%%%%%%%%%%%%%%%%
\todo{need to update for WSSSPE3}
The following is a partial list of workshop attendees who registered on the
collaborative notes document~\cite{WSSSPE2-google-notes} that was used
for shared note-taking at the meeting, or who participated in a breakout groups
and were noted in that group's notes.


{\small
\begin{longtable}{ll}
   Jordan Adams          &  Tulane University
\\ Alice Allen           &  Astrophysics Source Code Library (ASCL)
\\ Gabrielle Allen       & University of Illinois Urbana-Champaign
\\ Pierre-Yves Aquilanti &  TOTAL E\&P R\&T USA
\\ Wolfgang Bangerth & Texas A\&M University
\\ David Bernholdt       &  Oak Ridge National Laboratory
\\ Jakob Blomer
\\ Carl Boettiger        &  University of California Santa Cruz \& rOpenSci
\\ Chris Bogart          &  ISR/CMU
\\ Steven R. Brandt      &  Louisiana State University
\\ Neil Chue Hong        &  Software Sustainability Institute \& University of Edinburgh
\\ Tom Clune             &  NASA GSFC
\\ John W. Cobb
\\ Dirk Colbry           &  Michigan State University
\\ Karen Cranston        &  NESCent
\\ Tom Crick             &  Cardiff Metropolitan University, UK
\\ Ethan Davis           &  UCAR Unidata
\\ Robert R Downs        &  CIESIN, Columbia University
\\ Anshu Dubey           &  Lawrence Berkeley National Laboratory
\\ Nicole Gasparini      &  Tulane University, New Orleans
\\ Yolanda Gil           &  Information Sciences Institute, University of Southern California
\\ Kurt Glaesemann       &  Pacific northwest national lab
\\ Sol Greenspan         &  National Science Foundation
\\ Ted Habermann         &  The HDF Group
\\ Marcus D. Hanwell     &  Kitware
\\ Sarah Harris          &  University of Leeds
\\ David Henty           &  EPCC, The University of Edinburgh
\\ James Howison         &  University of Texas
\\ Maxime Hughes
\\ Eric Hutton           &  University of Colorado
\\ Ray Idaszak           &  RENCI/UNC
\\ Samin Ishtiaq         &  Microsoft Research Cambridge, UK
\\ Matt Jones            &  University of California Santa Barbara
\\ Nick Jones            &  New Zealand eScience Infrastructure, University of Auckland
\\ Daniel S. Katz        &  University of Chicago \& Argonne National Laboratory
\\ Ian Kelley
\\ Hilmar Lapp           &  National Evolutionary Synthesis Center (NESCent) \& Duke University
\\ Christopher Lenhardt
\\ Richard Littauer      &  University of Saarland
\\ Frank L\"{o}ffler     &  Louisiana State University
\\ Andre Luckow          &  Rutgers
\\ Berkin Malkoc         &  Istanbul Technical University
\\ Kyle Marcus           &  University at Buffalo
\\ Bryan Marker          &  The University of Texas at Austin
\\ Suresh Marru          &  Indiana University
\\ Robert H. McDonald    &  Data to Insight Center/Libraries, Indiana University
\\ Rupert Nash
\\ Andy Nutter-Upham     &  Whitehead Institute
\\ Abani Patra           &  University at Buffalo
\\ Aleksandra Pawlik     &  Software Sustainability Institute
\\ Cody J. Permann       &  Idaho National Laboratory
\\ John W. Peterson      &  Idaho National Laboratory
\\ Benjamin Pharr        &  University of Mississippi
\\ Stephen Piccolo       &  Brigham Young University, Utah
\\ Marlon Pierce         &  Indiana University
\\ Ray Plante            &  NCSA, University of Illinois Urbana-Champaign
\\ Sushil Prasad         &  Georgia State University, Atlanta
\\ Karthik Ram           &  Berkeley Institute for Data Science, University of California Berkeley \& rOpenSci
\\ Mike Rilee            &  NASA/GSFC \& Rilee Systems Technologies
\\ Erin Robinson         &  Foundation for Earth Science
\\ Mark Schildhauer      &  NCEAS, Univ. California, Santa Barbara
\\ Jory Schossau         &  Michigan State University
\\ Frank Seinstra        &  Netherlands eScience Center
\\ James Shepherd        &  Rice University
\\ Justin Shi
\\ Ardita Shkurti        &  University of Nottingham
\\ Alan Simpson          &  EPCC, The University of Edinburgh
\\ Carol Song            &  Purdue University
\\ James Spencer         &  Imperial College London
\\ Tracy Teal            &  Data Carpentry
\\ Kaitlin Thaney        &  Mozilla Science Lab
\\ Matt Turk             &  NCSA, University of Illinois Urbana-Champaign
\\ Colin C. Venters      &  University of Huddersfield
\\ Nathan Weeks
\\ Ethan White           &  University of Florida/Utah State University
\\ Nancy Wilkins-Diehr   &  San Diego Supercomputer Center, University of California San Diego
\\ Greg Wilson           &  Software Carpentry
\end{longtable}
}

\bibliographystyle{vancouver}

\bibliography{wssspe}
\end{document}

