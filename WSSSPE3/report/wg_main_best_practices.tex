\subsection{White paper/journal paper about best practices in developing sustainable software}
\label{sec:best-practices}

%\subsubsection{Why it is important}
Reviewing multiple past articles and talks at different meetings like WSSSPEx
%\cite{WSSSPE papers, heroux paper, ...} 
and 
analyzing and promoting sustainable scientific software makes it clear that there 
are several common and recurring ideas that underpin success in developing sustainable software. However, outside of a small community, this knowledge is not widely
shared. This is  especially true for the large community of scientists who generate most of the software used by scientists but are not primarily software developers. In this
scenario, a clear and precise exposition of these best practices collected from many sources and open collaboration among all in the community
in a single source (e.g., journal paper, tutorial)  that can be widely disseminated is necessary and likely to be very valuable.

\subsubsection{Fit with related activities}

The creation of such a ``best practices'' document will build upon the range of activities and topics discussed at WSSSPE3 and associated prior meetings. We will attempt to
distill the emerging body of knowledge into this document. The large number of  articles from the NSF funded SI2 projects (SSE and SSI), ``lightning talks'', ``white papers,'' and reports from different workshops have created a large if somewhat diffuse source for this report.

\subsubsection{Discussion}

Core questions that will need to be explored are in knowledge management, 
(transitions between people), reliability (reproducibility), usability, and how a software tool becomes part of the core workflow of well identified users (stakeholders)
relating to tool success and hence sustainability. \katznote{prev sentence is complex and awkward} Ideas 
that may need to be explored include:
\begin{itemize}
\item Requirements engineering to create tools with immediate uptake;
\item When should software ``die''?
\item Catering to disruptive developments in environment e.g.(new hardware, new methodology) ;
\item Dimensions of sustainability -- economic, technical, environmental, 
declining interest in primary application area), \katznote{not sure what the prev. comment goes with} social.
\end{itemize}

Sustainability requires community participation in code development and/or a wide adoption of software.
The larger the community base is using a piece of software, the better are the funding possibilities and thus also the sustainability options.
Additionally developer commitment to an application is essential and experience shows that software packages with an evangelist imposing strong inspiration and discipline are more likely to achieve sustainability.
While a single person can push sustainability to a certain level, open source software also needs sustained commitment from the developer community.
Such sustained commitments include diverse tasks and roles, which can be fulfilled by diverse developers with different knowledge levels.
Besides developing software and appropriate software management with measures for extensibility and scalability of the software, active (expertise) support for users via a user forum with a quick turnaround is crucial.
The barrier to entry for the community as users as well as developers has to be as low as possible.

For additional information about the discussion, see Appendix~\ref{sec:appendix_best_practices}.

\subsubsection{Plans}
%\todo{short text here - not bullets}
The creation of a best-practices document needs a large and diverse community involved. We have enlisted over ten contributors from the attendees at the WSSSPE3 and 
those on the mailing list.
The primary mechanism for developing this document will be to examine and analyze the success of several well known community scientific
software and organizations supporting scientific software.
We will attempt then to abstract general principles and best practices.
Some of the tools identified for such analysis are
the general purpose PeTSC toolkit for linear system solution, NWChem for computational chemistry and the CIG (Computational Infrastructure for Geodynamics) organization dedicated to supporting an ensemble of related tools for the geodynamics community. 
We also established a timeline and a rough outline for the report.

\noindent{\bf Timeline:}
\begin{itemize}
%
\item 15 Nov: Introduction and scope finished
\item 15 Nov: Sections assigned
\item 31 Jan: Analyzing funding possibilities for survey
\item 31 Jan: First versions of section
\item 15 Feb: Distribution to WSSSPE community
\item 31 Mar: Final version of white paper
\item 30 Apr: Submission of peer-reviewed paper?
\end{itemize}


\noindent{\bf Outline:}
\begin{enumerate}
\item Introduction and Scope of White Paper 
\item Related Work
\item Case Studies
\begin{enumerate} 
\item PeTSC
\item NWChem
\item CIG
\end{enumerate}
\item Community Related Practices
\begin{enumerate} 
\item Findings
\item Recommendations
\end{enumerate}
\item Governance and management
\begin{enumerate} 
\item Findings
\item Recommendations
\end{enumerate}
\item Funding Related
\begin{enumerate} 
\item Findings
\item Recommendations
\end{enumerate}
\item Metrics for sustainability
\item Tools
\item Conclusions
\end{enumerate}

\subsubsection{Landing Page}
The landing page with instructions, timeline and the white paper is here: \url{https://drive.google.com/drive/folders/0B7KZv1TRi06fbnFkZjQ0ZEJKckk}
Discussions can be also continued in \url{https://github.com/WSSSPE/meetings/issues/42}
