\subsection{Principles for Software Engineering Design for Sustainable Software} 

%\subsubsection{Why it is important}

\todo{fill in cites in this paragraph}
Software engineering principles form the basis of methods, techniques, methodologies and tools~\cite{Vliet:2008}. However, there is often a mismatch between software engineering theory and practice particulalry in the fields of compuational science and engineering, which can lead to the development of unsustainable software~\cite{Merali:2010,hettrick:2014}. Understanding and applying software engineering principles is essential in order to create and maintain sustainable software~\cite{Becker:2016}.

\subsubsection{Fit with related activities}
The group discussion focused on identifying existing principles of software engineering design that could be adopted by the computational science and engineering communities.

\subsubsection{Discussion}

Software engineering principles form the basis of methods, techniques, methodologies and tools. This group, which included members from different backgrounds, including quantum chemistry, epidemiology, computer science, software engineering, and microscopy, discussed the principles of software engineering design for sustainable software (starting with principles from the Karlskrona Manifesto on Sustainability Design~\cite{Becker:2015}, Tate~\cite{tate:2005}, and the SoftWare Engineering Body of Knowledge (SWEBOK)~\cite{swebokv3}) and their application in various domains including quantum chemistry and epidemiology.  The group examined the principles and took a retrospective analysis of what the developers did in practice against how the principles could have made a difference, and asked, what do the principles mean for  computational scientific and engineering software, and how do the principles relate to non-functional requirements? It
appeared that the sustainable software engineering principles should be mapped to two core quality attributes that underpin technically sustainable software: extensibility, the software's ability to be extended and the level of effort required to implement the extension; and
maintainability: the effort required to locate and fix an error in operational software.

For more information about the discussion, see Appendix~\ref{sec:appendix_eng_design}.

\subsubsection{Plans}
The next steps in this endeavor are to (1) Systematically analyze a number of example systems from different scientific domains with regards to the identified principles, to (2) Identify the commonalities and gaps in applying those principles to different scientific systems, and to (3) Propose a set of guidelines on the principles and how they exemplary apply to scientific software system. Preliminary work will be carried out through undergraduate or post-graduate student projects.

\subsubsection{Landing Page}
In the absence of a landing page, the Principles for Software Engineering Design for Sustainable Software working group requests an email be sent to Birgit Penzenstadler\footnote{email: \href{mailto:birgit.penzenstadler@csulb.edu}{birgit.penzenstadler@csulb.edu}} and Colin C. Venters\ to find out more about the group's efforts and how to participate.
