\subsection{Principles for Engineering Design for Sustainable Software} 

\subsubsection{Why it is important}
\todo{short text here}

\subsubsection{Fit with related activities}
\todo{short text here - can include links/cites}

\subsubsection{Discussion}
\todo{short-ish text here}
This group discussed the principles of engineering design for sustainable software and their application in various domains. The group included members from different backgrounds, including quantum chemistry, epidemiology, computer science, software engineering, and microscopy. Each participant was invited to give their perspective on the topic area and what they thought were the crucial points for discussion. There was a general consensus that there was a need for relating principles to practice for the computational science and engineering community. Furthermore, various members of the group expressed their interest in tools and best practices for facilitating the maintenance and evolution of scientific software systems.
It was agreed to identify principles from software engineering and from sustainability design and, based on those lists, discuss what each of those would mean applied to specific example systems from the expert domains of some of the group members. The group identified a number of software engineering principles drawn from the SoftWare Engineering Body of Knowledge (SWEBOK):

Software design principles:
\begin{itemize}
\item Abstraction;
\item Coupling and cohesion;
\item Decomposition and modularization;
\item Encapsulation and information hiding;
\item Separation of interface and implementation;
\item Sufficiency completeness \& primitiveness;
\item Separation of concerns.
\end{itemize}

User interface design principles:
\begin{itemize}
\item Learnability;
\item User familiarity;
\item Consistency;
\item Minimal surprise;
\item Recoverability;
\item User guidance;
\item User diversity.
\end{itemize}

The sustainability design principles were drawn from the Karlskrona Manifesto on Sustainability Design~\cite{Becker:2014}:
\begin{itemize}
\item Sustainability is systemic;
\item ...is multidimensional
\item ...is interdisciplinary;
\item ...transcends the system's purpose;
\item ...applies to both a system and its wider contexts;
\item ...requires action on multiple levels;
\item ...requires multiple timescales;
\item Changing design to take into account long-term effects doesn't automatically imply sacrifices;
\item System visibility is a precondition for and enabler of sustainability design.
\end{itemize}

This congregated list is an initial collection of principles that could be extended by adding from further 
related work form separate disciplines within the field of software engineering, including requirements 
engineering, software architecture, and testing. The group identified two example systems to discuss 
the application of the principles. The first one was a quantum chemistry system that allows the analysis 
of the characteristics and capabilities of molecules and solids. The second one was a modeling system 
for malaria that permitted biologists to analyze a range of datasets across geography, biology, and 
epidemiology, and add their own datasets. They then examined the principles and took a retrospective 
analysis of what the developers did in practice against how the principles could have made a 
difference. The opportunity was identified to distill existing software engineering and sustainability 
design knowledge into ``bite sized'' chunks for the Computational Science and Engineering 
Community. In addition, two challenges were pointed out: mapping of the principles to best practice, 
and  demonstrating the return on investment of those best practices.

\subsubsection{Plans}
The next steps in this endeavor are to (1) Systematically analyze a number of example systems from different scientific domains with regards to the identified principles, to (2) Identify the commonalities and gaps in applying those principles to different scientific systems, and to (3) Propose a guideline on the principles and how they exemplary apply to scientific software system.

\subsubsection{Landing Page}
\todo{link to landing page}
