%%%%%%%%%%%%%%%%%%%%%%%%%%%%%%%%%%%%%%%%%%%%%%%%%%%%%%%%%%%%
\section{Publishing Software Working Group Discussion}
\label{sec:appendix_publishing_SW}
%%%%%%%%%%%%%%%%%%%%%%%%%%%%%%%%%%%%%%%%%%%%%%%%%%%%%%%%%%%%

\subsection{Group Members}
{\small
\begin{longtable}{ll}
   Steven R. Brandt & Louisiana State University
\\ Daniel Gunter    & 
\\ Yuhan Ding       & 
\\ Neil Chue Hong   & Software Sustainability Institute
\end{longtable}
}

\subsection{Summary of Discussion}

This group explored the value of executable papers (papers whose content includes
the code needed to produce their own results), and other forms of publishing which
include dynamic electronic content. Transitioning to this type of publication offers
possibilities of addressing, or partially addressing sustainability concerns 
such as reproducibility (the paper contains all the artifacts needed to verify its
results), transitive credit (modules an executable paper depends on must be explicitly
loaded, making it more feasible to identify them), and improving documentation (an executable
paper must explain what its code does).

While the group did not feel it had a way to influence what is currently happening
in these experimental publishing venues, it felt that creating and curating a list of
such efforts would have value by attracting interest to these activities.
The Software Sustainability Institute agreed to host this list.

A tentative first cut at the list contains the following:
\begin{enumeration}
\item ACM Transactions on Mathematical Software (TOMS) - which provides the extra step
 of having reviewers validate the code which was submitted with the publication.
\item The Mathematica Journal - which publishes Mathematica notebooks (with equations,
figures, etc.) directly.
\item O'Reily Media has announced that it plans to make IPython Notebooks a first-class
 authoring environment for their publishing program alongside their existing mechanisms.
\item Nature is offering a list of notebooks published alongside more traditional articles,
 and is looking at ways to make these documents more official. There are, in fact, a
 number of journals that offer "electronic supplements" to the more traditionally published
 static articles.
\item There is also a list of "reproducible academic publications" maintained here:
  \url{https://github.com/ipython/ipython/wiki/A-gallery-of-interesting-IPython-Notebooks#reproducible-academic-publications}
\end{enumeration}

\subsection{Description of Opportunity, Challenges, and Obstacles}

The opportunity is to collect a list of current executable papers and
shine a light on the experiments and development efforts currently underway.

The only obstacle to this is the difficulty in finding and identifying such
publications. The Software Sustainability Institute was able to do something similar
for publications about software by making a public page containing a catalog
of these publications and enlisting the help of the community to grow the list.

\subsection{Key Next Steps}

Create the first version of the web page to be displayed on the Software Sustainability
Institute's website.

\subsection{Plan for Future Organization}

None at this time.

\subsection{What Else is Needed?}

Nothing else at this time.

\subsection{Key Milestones and Responsible Parties}

Steven R. Brandt will create a first version of the page within a week or so of the WSSSPE3 conference.

Neil Chue Hong will take responsibility for the page once it is up.

\subsection{Description of Funding Needed}

None.
