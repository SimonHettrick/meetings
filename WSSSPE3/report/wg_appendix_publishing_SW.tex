%%%%%%%%%%%%%%%%%%%%%%%%%%%%%%%%%%%%%%%%%%%%%%%%%%%%%%%%%%%%
\section{Publishing Software Working Group Discussion}
\label{sec:appendix_publishing_SW}
%%%%%%%%%%%%%%%%%%%%%%%%%%%%%%%%%%%%%%%%%%%%%%%%%%%%%%%%%%%%

Steven R.\ Brandt\footnote{email:
\href{mailto:sbrandt@cct.lsu.edu}{sbrandt@cct.lsu.edu}} will serve as the point
of contact for this working group.

\subsection{Group Members}

\begin{itemize}
\item Steven R.\ Brandt -- Louisiana State University
\item Daniel Gunter -- LBNL
\item Yuhan Ding -- Illinois Institute of Technology
\item Neil Chue Hong -- Software Sustainability Institute
\end{itemize}

\subsection{Summary of Discussion}

A tentative first cut at the list containing executable papers identified the following:
\begin{itemize}

\item ACM Transactions on Mathematical Software (TOMS): provides the additional step
of having reviewers validate the code which was submitted with the publication.
 
\item The Mathematica Journal: publishes Mathematica notebooks (with equations,
figures, etc.) directly.

\item O'Reilly Media: announced that it plans to make IPython Notebooks a
first-class authoring environment for their publishing program alongside their
existing mechanisms.

\item Nature: offers a list of notebooks published alongside more traditional
articles, and is looking at ways to make these documents more official. There
are, in fact, a number of journals that offer ``electronic supplements'' to the
more traditionally published static articles.

\item IPython: maintained a list of ``reproducible academic
publications''~\cite{ipython-pubs}.

\item KBase: offers narratives built on IPython or Jupyter notebooks for assembling
publications that are reproducible, and can be commented or annotated.
  
\end{itemize}

The group also discussed future possibilities for a new publication format
that might provide advantages:
\begin{itemize}

\item Journals could be built around an existing, widely used framework thereby
reducing the burden of studying code on the part of reviewers (common bits of
infrastructure which are not relevant to the science would be automatically
excluded).

\item Journals might be encouraged to use more metadata, making them easier to
mine for various analytical purposes.

\item The Research Ideas and Outcomes (RIO) journal is an effort to publish
fragmentary results that can subsequently be combined into a single content
item.

\item Papers could be made more understandable. Each equation or technical term
could be linked to a document/tutorial explaining its origin and/or
derivation.

\item So many options for publication currently exist that good science may be
getting lost in the noise. Would some sort of ``upvote'' mechanism be of value?

\item Some sort of Replicated Computation Results badge could be made available
to publications that have undergone greater scrutiny (this is already done by
TOMS).
  
\end{itemize}

\subsection{Description of Opportunity, Challenges, and Obstacles}

The opportunity is to collect a list of executable papers and shine a
light on the experiments and development efforts currently underway.

The only obstacle to this is the difficulty in finding and identifying such
publications. The Software Sustainability Institute was able to do something
similar for publications about software by making a public page on the Software
Sustainability Institute's website (\url{http://www.software.ac.uk}) containing a
catalog of these publications and enlisting the help of the community to grow
the list. 

\subsection{Key Next Steps}

Create the first version of the web page to be displayed on the Software
Sustainability Institute's website: \url{http://www.software.ac.uk}.
% \choinote{already appeared above) /katznote{but important to have it here for poeple who are scanning/looking at this part of each section}
We expect the page to be live in early January of 2016.

An ongoing effort to update the page should follow.

\subsection{Plan for Future Organization}

None at this time.

\subsection{What Else is Needed?}

Nothing else at this time.

\subsection{Key Milestones and Responsible Parties}

Steven R.\ Brandt has created a first version of the page, and it is in the process
of being posted on the Software
Sustainability Institute's website: \url{http://www.software.ac.uk}. % \choinote{already appeared above)
Neil Chue Hong will take responsibility for the page once it is up.

\subsection{Description of Funding Needed}

None.
