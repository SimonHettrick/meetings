%%%%%%%%%%%%%%%%%%%%%%%%%%%%%%%%%%%%%%%%%%%%%%%%%%%%%%%%%%%%
\subsection{Useful Metrics for Scientific Software}
\label{sec:software-metrics}
%%%%%%%%%%%%%%%%%%%%%%%%%%%%%%%%%%%%%%%%%%%%%%%%%%%%%%%%%%%%

%\subsubsection{Why it is important}

Metrics for scientific software are important for tenure and promotion,
scientific impact, discovery, reducing duplication, serving as a basis for
potential industrial interest in adopting software, prioritizing development and
support towards strategic objectives, and making a case for new or continued
funding. However, there is no commonly-used standard for collecting or
presenting metrics, nor is it known if there is a common set of metrics for
scientific software. It is imperative that scientific software stakeholders
understand that it is useful to collect metrics.

\subsubsection{Fit with related activities}

The group discussion focused on identifying existing frameworks and activities
for scientific software metrics. The group identified the following related
activities:


\begin{itemize}

\item Computational Infrastructure for Geodynamics: Software Development Best
Practices\footnote{\url{https://geodynamics.org/cig/dev/best-practices/}}

\item WSSSPE3 Breakout Session: How can we measure the impact of a piece of code on
research, and its value to the
community?\footnote{\url{https://docs.google.com/document/d/1cgUDH3RxrfsLotWhKKOrXUnaYFhrtjcV1TDRkFtwQKI/edit}}

\item 2015 NSF SI2 PI Workshop Breakout Session on Framing Success
Metrics\footnote{\url{https://docs.google.com/document/d/10yj7MYEjvrg__t522XR41ogASYMp647-l-BpFTsqEV4/}}

\item 2015 NSF SI2 PI Workshop Breakout Session on Software
Metrics\footnote{\url{https://docs.google.com/document/d/1uDim5bw8rBuubmtaUrz5Eh35NxzDgivmmdXhVzDs3tc/edit}}

\item NSF Workshop on Software and Data Citation Breakout Group on Useful
Metrics\footnote{\url{https://docs.google.com/presentation/d/1PPLVL6uoOmisqnHTlwhsVKJBTFFK1IVzvr8FdEEIvAE/}}

\item U.K. Software Sustainability Institute Software Evaluation
Guide\footnote{\url{http://www.software.ac.uk/software-evaluation-guide}}

\item U.K. Software Sustainability Institute Blog post: The five stars of
research
software\footnote{\url{http://www.software.ac.uk/blog/2013-04-09-five-stars-research-software}}

\item Minimal information for reusable scientific
software\footnote{\url{http://figshare.com/articles/Minimal_information_for_reusable_scientific_software/1112528}}

\item EPSRC-funded Equipment Data Search
Site\footnote{\url{http://equipment.data.ac.uk/}}

\item Canarie Research Software: Software to accelerate
discovery\footnote{\url{http://www.canarie.ca/software/}}

\item Canarie Research Software: Research Software Platform
Registry\footnote{\url{https://science.canarie.ca/researchmiddleware/platforms/list/main.html}}

\item BlackDuck Open HUB\footnote{\url{https://www.openhub.net/}}
\item Innovation Policy
Platform\footnote{\url{https://www.innovationpolicyplatform.org/frontpage}}


\end{itemize}


\subsubsection{Discussion}

The group discussion began by agreeing on the common purpose of creating a set
of guidance giving examples of specific metrics for the success of scientific
software in use, why they were chosen, what they are useful to measure, and any
challenges and pitfalls; then publish this as a white paper. The group discussed
many questions related to useful metrics for scientific software including
addressing if there is a common set of metrics that can be filtered in some way,
can metrics be fit into a common template, which metrics would be the most
useful for each stakeholder, which metrics are the most helpful and how would we
assess this, how are metrics monitored, and many more. A more complete bulleted
list of these questions can be found in Appendix H. Next, a roadmap for how to
proceed was discussed including creating a set of milestones and tasks. The idea
was put forth for the group to interact with the organizing committee of the
2016 NSF Software Infrastructure for Sustained Innovation (SI2) PI workshop in
order to send a software metrics survey to all SI2 and related awardees as a
targeted and relevant set of stakeholders. The five solicitations for software
elements released under the NSF SI2 program all included metrics as a required
component with submitters requested to include {\it ``a list of tangible metrics,
with end user involvement, to be used to measure the success of the software
element developed, \dots''}. These metrics are then reported as part of annual
reports to NSF by the projects. Although neither the proposal text describing
the metrics nor the reported metric results are publicly available, there is
reason to believe that the community will be willing to provide this information
through a survey mechanism. This survey would be created by one of the student
group members. Similarly, it was suggested that a software metrics survey be
sent to the UK SFTF (Software For The Future, led by the Engineering and Physical Sciences Research Council) and TRDF (Tools and Resources Development Fund, led by the Biotechnology and Biological Sciences Research Council) software projects to ask them what metrics would be useful to report. The remainder of the discussion focused mainly on the creation
of a white paper on this topic. This resulted in a paper outline and writing
assignments with the goal of publishing in venues including WSSSPE4, IEEE CiSE (Institute of Electrical and Electronic Engineers Computing in Science and Engineering magazine), or JORS (Journal of Open Research Software). More information about the group discussion is available in Appendix~\ref{sec:appendix_metrics}

\subsubsection{Plans}

The main plan for the group going forward is the creation of a white paper on
the topic of useful metrics for scientific software. The authoring of this white
paper would happen in parallel with the creation of a survey by the group with
the survey results to be incorporated in the white paper. The timeline for
completion of the white paper is approximately one year targeting venues
discussed in the previous section.

\subsubsection{Landing Page}

In lieu of a landing page, the Useful Metrics for Scientific Software working
group requests an email be sent to Gabrielle Allen
\href{mailto:gdallen@illinois.edu}{(gdallen@illinois.edu)} to find out more
about the group's efforts and how to participate.
