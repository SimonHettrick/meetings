%%%%%%%%%%%%%%%%%%%%%%%%%%%%%%%%%%%%%%%%%%%%%%%%%%%%%%%%%%%%
\section{Best Practices Group Discussion}
\label{sec:appendix_best_practices}
%%%%%%%%%%%%%%%%%%%%%%%%%%%%%%%%%%%%%%%%%%%%%%%%%%%%%%%%%%%%

Sandra Gesing\footnote{email: \href{mailto:sandra.gesing@nd.edu}{sandra.gesing@nd.edu}} will serve as
the point of contact for this working group, and be responsible for ensuring timely progress of the planned actions.

\subsection{Group Members}

\begin{itemize}
\item Abani Patra -- University at Buffalo
\item Sandra Gesing -- University of Notre Dame
\item Neil Chue Hong -- Software Sustainability Institute
\item Gregory Tucker -- University of Colorado at Boulder
\item Birgit Penzens -- California State University Long Beach
\item Abigail Cabunoc Mayes -- Mozilla Foundation
\item Frank L\"{o}ffler -- Louisiana State University 
\item Colin Venter --  University of Huddersfield
\item Lorraine Hwang -- UC Davis 
\item Sou-Cheng Choi -- NORC at the University of Chicago \&  Illinois Institute of Technology
\item Suresh Marru -- Indiana University
\item Don Middleton -- NCAR 
\item Daniel S. Katz --  University of Chicago \& Argonne National Laboratory
\item Kyle Niemeyer -- Oregon State University
\item Jeffrey Carver -- University of Alabama
\item Dan Gunter -- LBNL
\item Alexander Konovalov -- University of St Andrews
\item Tom Crick --  Cardiff Metropolitan University

\end{itemize}

\subsection{Summary of Discussion}

Core questions that will need to be explored are in reliability (reproducibility), usability, extensibility, knowledge management and continuity (transitions between people). Answers to these will guide us on how a software tool becomes part of the core workflow of well identified users (stakeholders) relating to tool success and hence sustainability.
%\katznote{prev sentence is complex and awkward} % see fix
Ideas that may need to be explored include:
\begin{itemize}

\item Requirements engineering to create tools with immediate uptake;

\item When should software ``die''?

\item Catering to disruptive developments in environment, e.g., new hardware,
new methodology;

\item Dimensions of sustainability -- economic, technical, environmental,
and obsolescence.
%declining interest in primary application area), \katznote{not sure what the
%prev. comment goes with} social.

\end{itemize}

Sustainability requires community participation in code development and/or a
wide adoption of software. The larger the community base is using a piece of
software, the better are the funding possibilities and thus also the
sustainability options. Additionally, the developers’ commitment to an application is
essential and experience shows that software packages with an evangelist
imposing strong inspiration and discipline are more likely to achieve
sustainability. While a single person can push sustainability to a certain
level, open source software also needs sustained commitment from the developer
community. Such sustained commitments include diverse tasks and roles, which can
be fulfilled by diverse developers with different knowledge levels. Besides
developing software and appropriate software management with measures for
extensibility and scalability of the software, active (expertise) support for
users via a user forum with a quick turnaround is crucial. The barrier to entry
for the community as users as well as developers has to be as low as possible.

\subsection{Description of Opportunity, Challenges, and Obstacles}

The opportunity lies in collaboration on a white paper, which will be revisited
regularly for further improvements, to enhance knowledge of the state of best
practices, resulting in a peer-reviewed paper. We would like to reach a wide
community by doing this. But these are also the challenges and obstacles -- to
get everyone to contribute to the paper and to reach the community.

\subsection{Key Next Steps}

The key next steps are to write an introduction, reach out to the co-authors,
and to agree on the scope of the white paper.

\subsection{Plan for Future Organization}

Sandra Gesing and Abani Patra are the main editors and will organize the overall
communication and the paper. Sections will be assigned to diverse co-authors.

\subsection{What Else is Needed?}

At the moment we do not see any further requirements.

\subsection{Key Milestones and Responsible Parties}
\begin{itemize}
\item 15 Nov: Introduction and scope finished (Abani Patra/Sandra Gesing)
\item 15 Nov: Sections assigned (Abani Patra/Sandra Gesing)
\item 31 Jan: Analyzing funding possibilities for survey
\item 31 Jan: First version of each section
\item 15 Feb: Distribution to the WSSSPE community
\item 31 Mar: Final version of the white paper
\item 30 Apr: Submission to a peer-reviewed journal?
\end{itemize}

\subsection{Description of Funding Needed}
We might need funding for a journal publication (open-access options).
