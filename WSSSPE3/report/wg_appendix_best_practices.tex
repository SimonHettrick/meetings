%%%%%%%%%%%%%%%%%%%%%%%%%%%%%%%%%%%%%%%%%%%%%%%%%%%%%%%%%%%%
\section{Best Practices Group Discussion}
\label{sec:appendix_best_practices}
%%%%%%%%%%%%%%%%%%%%%%%%%%%%%%%%%%%%%%%%%%%%%%%%%%%%%%%%%%%%

\todo{add POC here}

\subsection{Group Members}

\begin{itemize}
\item Abani Patra 
\item Sandra Gesing -- University of Notre Dame
\item Neil Chue Hong 
\item Greg Tucker 
\item Birgit Penzens 
\item Abigail Cabunoc Mayes 
\item Jeff Carver 
\item Frank L\"{o}ffler 
\item Colin Venter 
\item Lorraine Hwang 
\item Sou-Cheng Choi
\item Suresh Marru 
\item Don Middleton 
\item Daniel Katz  
\item Kyle Niemeyer 
\item Jeff Carver 
\item Dan Gunter 
\item Alexander Konovalov 
\item Tom Crick 

\end{itemize}

\subsection{Summary of Discussion}
Core questions that will need to be explored are in knowledge management, 
(transitions between people), reliability (reproducibility), usability and how a software tool becomes part of the core workflow of well identified users (stakeholders)
relating to tool success and hence sustainability. Ideas 
that may need to be explored include:
\begin{itemize}
\item Requirements engineering to create tools with immediate uptake;
\item When should software ``die''?
\item Catering to disruptive developments in environment e.g.(new hardware, new methodology) ;
\item Dimensions of sustainability - economic, technical, environmental, 
declining interest in primary application area), social.
\end{itemize}

Sustainability requires community participation in code development and/or a wide adoption of software. The larger the community base is using a software, the better are the funding possibilities and thus also the sustainability options. Additionally developer commitment to an application is essential and experience shows that especially software packages with an evangelist imposing strong inspiration and discipline is required to achieve sustainability.
While a single person can push sustainability to a certain level, open source software also needs sustained commitment from the developer community. Such sustained commitments include diverse tasks and roles, which can be fulfilled by diverse developers with different knowledge levels. Besides developing software and appropriate software management with measures for extensibility and scalability of the software, active (expertise) support for users via a user forum with a quick turnaround is crucial. The barrier to entry the community as user as well as developer has to be as low as possible.

\subsection{Description of Opportunity, Challenges, and Obstacles}

The opportunity lays in the collaboration on a white paper, which will be revisited regularly for further improvements and enhances the state on best practices resulting in a peer-reviewed paper. This way we would like to reach a wide community. But these are also the challenges and obstacles - to get everyone to write on the paper and reach the community.

\subsection{Key Next Steps}

The key next steps are to write the introduction, reach out to the co-authors and agree on a scope of the white paper.

\subsection{Plan for Future Organization}

Sandra Gesing and Abani Patra are the main editors and will organize the overall communication and paper. Sections will be assigned to diverse co-authors.

\subsection{What Else is Needed?}

At the moment we don't see any further requirements.

\subsection{Key Milestones and Responsible Parties}
\begin{itemize}
\item 15 Nov: Introduction and scope finished (Abani Patra/Sandra Gesing)
\item 15 Nov: Sections assigned (Abani Patra/Sandra Gesing)
\item 31 Jan: Analyzing funding possibilities for survey
\item 31 Jan: First versions of section
\item 15 Feb: Distribution to WSSSPE community
\item 31 Mar: Final version of white paper
\item 30 Apr: Submission of peer-reviewed paper?
\end{itemize}

\subsection{Description of Funding Needed}
We might need funding for a journal publication (open-access options).
