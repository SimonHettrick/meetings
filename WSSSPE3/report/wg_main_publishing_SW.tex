\subsection{Publishing Software Working Group Discussion}

\subsubsection{Why it is important}

This group explored the value of executable papers (papers whose content includes
the code needed to produce their own results), and other forms of publishing which
include dynamic electronic content. Transitioning to this type of publication offers
possibilities of addressing, or partially addressing sustainability concerns 
such as reproducibility (the paper contains all the artifacts needed to verify its
results), transitive credit (modules an executable paper depends on must be explicitly
loaded, making it more feasible to identify them), and improving documentation (an executable
paper must explain what its code does).

Reproducibility: Part of the purpose of these venues is to (at least partially)
address the reproducibility issue by making the paper itself recompute its own
results.

Transitive Credit: Since these forms of publishing must make their sources explicit,
they should be easier to trace even if appropriately worded credit for software
is not provided. In addition, these notebooks make it possible to provide/define
additional metadata to make the tracing of credit clearer. In addition, attributions
could be added to citations to identify whether a paper extends a result, verifies it,
contradicts it, etc.

Best Practices: Because an executable paper showcases the code 

\subsubsection{Fit with related activities}
\todo{short text here - can include links/cites}

\subsubsection{Discussion}
\todo{short-ish text here}

\subsubsection{Plans}
\todo{short text here - not bullets}

\subsubsection{Landing Page}

A page will be made available on the Software Sustainability Institute website.
