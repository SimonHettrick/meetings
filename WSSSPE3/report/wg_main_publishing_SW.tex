\subsection{Publishing Software Working Group Discussion} \label{sec:publishing-software}

This group explored the value of executable papers (papers whose content
includes the code needed to produce their own results), and other forms of
publishing which include dynamic electronic content.
%
%\subsubsection{Why it is important}
%
Transitioning to this type of publication offers possibilities of addressing, or
partially addressing sustainability concerns such as reproducibility, software
credit, and best practices.

\subsubsection{Fit with related activities}

\textbf{Reproducibility}: Part of the purpose of these executable paper venues
is to (at least partially) address the reproducibility issue by making the paper
itself recompute its own results.

\textbf{Software Credit (\S\ref{sec:software-credit})}: Since these forms of
publishing must make their sources explicit in order to execute, they should be
easier to trace even if appropriately worded credit for software is not
provided. In addition, these notebooks make it possible to provide/define
additional metadata to make the tracing of credit clearer. Finally, attributions
could be added to citations to identify whether a paper extends a result,
verifies it, contradicts it, etc.

\textbf{Best Practices (\S\ref{sec:best-practices})}: Because an executable
paper showcases the code, and the code itself is subject to the review process,
authors are more likely to pay attention to coding practices. In addition,
because the paper must explain what the code does, better documentation is
likely to be achieved.

\subsubsection{Discussion}

The group felt that the best way to encourage the use of these new publishing
concepts would be to create and curate a list identifying publishing venues that
support them. The Software Sustainability Institute agreed to host this list.

See Appendix~\ref{sec:appendix_publishing_SW} for more details about the
discussion.

\subsubsection{Plans}

The plan is to create a web page describing executable papers, their value, and
a list of what publishers support them. \katznote{when?}

\subsubsection{Landing Page}

The aforementioned page will be published on the Software Sustainability
Institute website: \url{http://www.software.ac.uk}.
