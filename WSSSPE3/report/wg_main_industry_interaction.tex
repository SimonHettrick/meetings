
%%%%%%%%%%%%%%%%%%%%%%%%%%%%%%%%%%%%%%%%%%%%%%%%%%%%%%%%%%%%
\subsection{Transition Pathways to Sustainable Software: Industry \& Academic Collaboration}
\label{sec:industry_interaction}
%%%%%%%%%%%%%%%%%%%%%%%%%%%%%%%%%%%%%%%%%%%%%%%%%%%%%%%%%%%%

%\subsubsection{Why it is important}

Most scientific software is produced as a part of grant-funded research projects
typically sponsored by federal governments. If we are interested in the
sustainability of scientific software, then we need to understand what exactly
happens when that sponsorship ends. More than likely, the project and its
resulting software will need to undergo some kind of transition in funding and
consequently governance.

At WSSSPE3, this working group was interested in better understanding successful pathways
for scientific software to ``transition'' from grant-funded research projects to
industry sponsorship. (This may be an initially awkward phrase---some software
projects will begin their life being sponsored by industry, or result in
collaboration between industry and academia. In such cases, there is still a
need to understand how IP and how maintenance of the software is sustained over
time.)

\subsubsection{Fit with related activities}

Most previous research and discussion of industry and academic collaboration,
sharing, and funding of research software has focused on the impact of such
arrangements. Examples of these types of reports are:

\begin{itemize}
\item REF Impact Case Studies: \url{http://impact.ref.ac.uk/CaseStudies/}
\item Background of projects funded in the UK: \url{http://gtr.rcuk.ac.uk/}
\item Dowling Review from the UK: addresses complexity of work between these two
communities: \url{http://www.raeng.org.uk/policy/dowling-review}
\item Pathway to Impact -- UK report: two pages of grant proposals are asked to
forecast what impact they might have (including environmental, academic, economic).
\end{itemize}

\subsubsection{Discussion}

Although sustainability transitions are often studied under the broad umbrella
of ``technology transfer,'' the group believes there are likely to be a number of
different ways in which a pathway from initial production to long-term
maintenance and secure funding is achieved. In short, industry sponsorship
and/or direct participation is an important aspect of sustaining scientific
software, but our current understanding of these transitions focuses narrowly on
commercial successes or failures of those collaborations.

In looking at existing literature that addresses industry transitions, many
reports (such as those listed above) focus on benefits that accrue to the
private sector, or to a government that originally sponsored the research
project. This literature does not address the impact that these transitions have
on the accessibility or usability of the software, or the impact that these
transitions have on the career of the researchers involved.

For more detail on the group's discussion, see
Appendix~\ref{sec:appendix_industry_interaction}.

\subsubsection{Plans}

Plans for carrying forward are currently unclear---this project would require
sustained attention and effort from the group members, and at least some amount of
funding in order for those members to be involved for extended periods of time.

The broad goals that the group would like to accomplish are: 

\begin{enumerate}

\item To complete a set of case studies which look at successful and
unsuccessful transitions between academic researchers and industry

\item To create a generalizable framework, which might allow for a broader study
of different transition pathways (other than between academia and industry)

\end{enumerate}

The main plan for the group going forward is the creation of a white paper on
the topic of sustainability transitions.

\subsubsection{Landing Page}

Transitions Pathways discussions can be posted at
\url{https://github.com/WSSSPE/meetings/issues/46} or an email be sent to Nic
Weber\footnote{email: \href{mailto:nmweber@uw.edu}{nmweber@uw.edu}} to find out
more about the group's efforts and how to participate.
