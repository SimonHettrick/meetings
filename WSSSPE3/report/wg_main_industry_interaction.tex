\subsection{Topic} 

Transition Pathways to Sustainable Software: Industry \& Academic Collaboration

\subsubsection{Why it is important}

Most scientific software is produced as a part of grant-funded research projects sponsored by federal governments. If we are interested in the sustainability of scientific software then we need to understand what exactly happens when that sponsorship ends. More than likely, the project and its resulting software will need to undergo some kind of transition in funding and consequently management. 

There are a number of potential funding transitions that may occur:  

\begin{itemize}
\item A project could be \textbf{refunded}, and development or maintenance of the software continues as planned.
\item A project might located a \textbf{new source of funding} in which case the software may be further developed or simply maintained as before. 
\item The project could transition to a \textbf{community supported model} whereby ownership, maintenance and stewardship of the software becomes similar to peer-production models in open-source. (e.g. see Howison 2015)
\item The project could receive some form of industry sponsorship in which case ownership of the intellectual property, licensing, maintenance activities, hosting, etc. may change significantly. 
\end{itemize}

We characterize each of these potential changes in funding as ``transition pathways'' to sustainable software (see similar work by Geels and Schot, 2007). 

At WSSSPE 3 our group was interested in better understanding successful pathways for scientific software to ``transition'' from government-funded research projects to industry sponsorship. (This may be an initially awkward phrase - because some software projects will begin their life being sponsored by industry, or result in collaboration between industry and academia. In such cases, there is still a need to understand how IP and how maintenance of the software is sustained over time)

Although transitions are often studied under the broad umbrella of `technology transfer' we believe there are likely to be a number of different ways in which a pathway from initial production to long-term maintenance and secure funding is achieved. In short, industry sponsorship is an important aspect of sustaining scientific software, but our current understanding of these transitions focuses narrowly on commercial successes / failures.

\subsubsection{Fit with related activities}

In looking at existing literature that addresses industry transitions many reports focus on benefits that accrue to the private sector, or to a government that originally sponsored the research project. This literature does not address the impact that these transitions have on the accessibility, or usability of the software, or the impact that these transitions have on the career of the researchers involved. 

Examples of the former scenario (benefits accruing to private sector) are as follows:

\begin{itemize}
\item REF Impact Case Studies: \url{http://impact.ref.ac.uk/CaseStudies/}
\item Background of projects funded in the UK: \url{http://gtr.rcuk.ac.uk/}
\item Dowling Review from the UK - addresses complexity of work between these two communities: \url{http://www.raeng.org.uk/policy/dowling-review}
\item Pathway to Impact - UK report  - 2 pages of grant proposals are asked to forecast how they might impact (includes environmental, academic, economic )
\end{itemize}

\subsubsection{Discussion}

Our work at WSSSPE3 included the following activities (described in detail below): 1. Brainstorming goals for this type of research, 2. Imagining potential outcomes of completing a set of case studies on this topic, and 3. Generating a set of working definitions for some of the broad concepts we are describing. 

\textbf{GOAL}
\emph{What is the goal of doing research on transition pathways?} 

Can we identify collaborations that have occurred and try to understand which were successful, which were unsuccessful, and what factors contributed to these successes/failures? 

Can we determine what each partner wants to get out of such a collaboration? Why would industry be interested in collaborating with academia?  Why would academia be interested in collaborating with industry?

How could we design a study that focused on the impact of the software in undergoing this type of transition?   

\textbf{Potential outcomes}

Complete a set of case studies which look at successful and unsuccessful transitions between academic researchers and industry. This might address one of each of the transition types (described below). Successful transitions are described as those that lead to either weak or strong sustainability (also defined below). 

Create a generalizable framework that might allow for the study of different transition pathways (other than academia -> industry). 

\textbf{General Definitions}

We characterize transitions in the following way:

\begin{itemize}
\item Handoff model - i.e., academia initially writes the software, industry (for profit or nonprofit) then takes over project. 
\item Co-Production Model - i.e., industry and academia interact throughout development of the project.
\item Sponsorship Model  - i.e., academia writes and maintains the software, the industry contributes funding for the development/maintenance of software. In this example industry is also likely a user of the software.
\item Spinoff model- transition to for-profit company owned by or in collaboration original developers; or, non-profit formed by or in collaboration with developers. 
\end{itemize}

We characterize sustainability in the following ways:

\begin{itemize}
\item Weak Sustainability: Software continues to be accessible, useful, and usable. 
\item Strong Sustainability: Software meets criteria above, but is also able to be reused for further innovation (i.e., issued non-restrictive open-source license). (see Becker et al., 2014 for extended discussion of weak vs. strong sustainability.)
\end{itemize}

\subsubsection{Plans}

Plans for carrying forward are currently unclear - this project would require sustained attention and effort from our team, and at least some amount of funding in order for us to be involved for extended periods of time.

\subsubsection{Landing Page}

\todo{where should someone go who want to know more about this and perhaps wants to contribute?}

\subsection{Works Referenced}

\todo{please add these to the bib file and cite them from there.}

\begin{itemize}
\item Becker, C., Chitchyan, R., Duboc, L., Easterbrook, S., Mahaux, M., Penzenstadler, B., ... \& Betz, S. (2014). The Karlskrona manifesto for sustainability design. arXiv preprint arXiv:1410.6968.
\item Geels, F. W., \& Schot, J. (2007). Typology of sociotechnical transition pathways. Research policy, 36(3), 399-417.
\item Howison, J. (2015). Sustaining scientific infrastructures: transitioning from grants to peer production. Proceedings of the 2015 iConference. 
\end{itemize}