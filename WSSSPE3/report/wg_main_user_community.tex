\subsection{Building Sustainable User Communities for Scientific Software}
%\todo{change the title to your topic}

\subsubsection{Why it is important}
%\todo{short text here}

User communities are the lifeblood of sustainable scientific software. The user community includes the developers, 
both internal and external, of the software; direct users of the software; other software projects that depend on
the software; and any other groups that create or consume data that is specific to the software. Together these
groups provide both the reason for sustaining the software and, collectively, the requirements that drive its continued
evolution and improvement.

\subsubsection{Fit with related activities}
%\todo{short text here - can include links/cites}

Mozilla Science maintains a "'Working Open' Project Guide" (http://mozillascience.github.io/leadership-training). From the introduction:
\begin{quote}
Working openly with contributors enables your
    community to learn how to build and collaborate together. This
    document is a guideline on how to work openly and involve others
    in your projects with Mozilla. We want to help you engage your
    community in a way that encourages contributors and builds other
    leaders.
  \end{quote}

Several books have been written about software communities:
\begin{itemize}
\item "Art of Community" by Jono Bacon. We could consider distilling this for scientific software.

\subsubsection{Discussion}
\todo{short-ish text here}

\subsubsection{Plans}
\todo{short text here - not bullets}

\subsubsection{Landing Page}
\todo{link to landing page}
