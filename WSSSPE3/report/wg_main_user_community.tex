\subsection{Building Sustainable User Communities for Scientific Software}

%\subsubsection{Why it is important}

User communities are the lifeblood of sustainable scientific software. The user
community includes the developers, both internal and external, of the software;
direct users of the software; other software projects that depend on the
software; and any other groups that create or consume data that is specific to
the software. Together these groups provide both the reason for sustaining the
software and, collectively, the requirements that drive its continued evolution
and improvement.

\subsubsection{Fit with related activities}

Mozilla Science maintains a ``Working Open Project
Guide''~\cite{working-open-wssspe3}, the introduction of which states:
\begin{quote}
Working openly with contributors enables your community to learn how to build
and collaborate together. This document is a guideline on how to work openly and
involve others in your projects with Mozilla. We want to help you engage your
community in a way that encourages contributors and builds other leaders.
 \end{quote}

\katznote{the next little bit needs some work, as Choi points out}

Several books have been written about software communities: \choinote{only 1 book below}
\begin{itemize}

\item ``Art of Community'' by Jono Bacon~\cite{art-of-community}. We could
consider distilling this for scientific software.

\item Iain Larmour, from EPSRC in the UK (EPSRC: https://www.epsrc.ac.uk/) (not
sure who from mentioned UK Collaborative Computational Projects (CCP:
http://ccp.ac.uk) \choinote{Who is ``not sure who'' supposed to be?}

\end{itemize}

\subsubsection{Discussion}

Discussion revolved around a few questions: what is the benefit of having a
``community'' for software sustainability, what practices and circumstances lead
to having and maintaining a community, how can funding help or hinder this
process, and perhaps most importantly, how can best practices be described and
distilled into a document that can help new projects.

The benefits of having a community that were brought up were considered largely
obvious. In addition to having advocates for the software, and a possible source
of ``free'' contributions to the codebase, the community becomes a good source
for requirements, feedback, and metrics. The software community can also act as
``cheerleaders'' who convince funders or other potential users to fund/use the
software, and thus help sustain the software.

Practices and circumstances that lead to a community are first, that the
software offers value. But in addition to this, a community will be much more
likely to form if they receive (expert) support when they have questions.
Additional contributing factors are good usability (not always needed), and an
open development process such as IPython developer meetings on YouTube. It was
also pointed out that an evangelist for the project, not necessarily but often
one of the developers, can often make a big difference.

Funding can help the process by encouraging both value to the community and
high-quality user support. Only providing funding for the software development
may create good software, but with less likelihood to have a real community. It
was discussed that federal laboratories are a good incubator for software
communities, and that a general facility like EarthCube is too dispersed to
really make a community. Also, domain-specific groups within laboratories or
universities might provide as an incubator for software communities.

In describing best practices, the group discussed the different modes for
starting a scientific software project: building on an existing product that
needs improving, recognizing an unsatisfied need of an existing community, or
creating a new solution to a need not yet recognized by the community. The group
also thought that the existing books on software communities would need to be
evaluated in light of differences between Science Software projects and general
OSS projects in terms of scale, science, acknowledgement and credit, and funding
models.

Additional details on the group's discussion can be found in
Appendix~\ref{sec:appendix_user_community}.

\subsubsection{Plans}

The most important next steps is a ``Best Practice'' document, which would
describe what successful projects with engaged communities look like, how to
replicate this type of project, and look at end-of-life on a community project.
Inputs to this document would include a software community survey of highly
functioning communities such as R Open Science, Python SciPy, OPeNDAP, and
Unidata, with analysis of factors that feed into their success. Also references
lik the ``Art of Community'' could be adapted and summarized for the science
software community.

Another next step would be increasing recognition of need for science software
projects to focus on building and supporting their user communities. Good software
engineering practices are not enough, and popular training like Software
Carpentry does not currently address this issue head on.

\subsubsection{Landing Page}
\todo{link to landing page}
