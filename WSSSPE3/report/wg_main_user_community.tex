\subsection{Building Sustainable User Communities for Scientific Software}

%\subsubsection{Why it is important}

User communities are the lifeblood of sustainable scientific software. The user
community includes the developers, both internal and external, of the software;
direct users of the software; other software projects that depend on the
software; and any other groups that create or consume data that is specific to
the software. Together these groups provide both the reason for sustaining the
software and, collectively, the requirements that drive its continued evolution
and improvement.

\subsubsection{Fit with related activities}

There are a number of activities already in progress that are targeted at improving
the user community for open-source software, including Mozilla Science's ``Working Open Project
Guide''~\cite{working-open-wssspe3} and
``UK Collaborative Computational Projects'' (CCP: http://www.ccp.ac.uk), or 
 books such as ``Art of Community'' by Jono Bacon~\cite{art-of-community}. 

\subsubsection{Discussion}

Discussion revolved around a few questions: what is the benefit of having a
``community'' for software sustainability, what practices and circumstances lead
to having and maintaining a community, how can funding help or hinder this
process, and perhaps most importantly, how can best practices be described and
distilled into a document that can help new projects.

Everyone agreed on a few points: software must not only offer value, but there
must be some support for users; and funding can help pay for that support, in
addition to the usual funding for software development. Openness is generally 
a virtue. An evangelist, either in the form of a single person or some
domain-specific group of users, is often the key factor.

Additional details on the group's discussion can be found in
Appendix~\ref{sec:appendix_user_community}.

\subsubsection{Plans}

The most important next steps is a ``Best Practice'' document, which would
describe what successful projects with engaged communities look like, how to
replicate this type of project, and look at end-of-life on a community project.
Another next step would be better training to increase recognition of need for science software
projects to focus on building and supporting their user communities.

\subsubsection{Landing Page}

This group doesn't have a landing page yet. Please send requests to join and contribute to both
Dan Gunter <dkgunter@lbl.gov> and Ethan Davis <edavis@ucar.edu>.
