%%%%%%%%%%%%%%%%%%%%%%%%%%%%%%%%%%%%%%%%%%%%%%%%%%%%%%%%%%%%
\section{User Community Working Group Discussion}
\label{sec:appendix_user_community}
%%%%%%%%%%%%%%%%%%%%%%%%%%%%%%%%%%%%%%%%%%%%%%%%%%%%%%%%%%%%

POC:
Dan Gunter\footnote{email: \href{mailto:dkgunter@lbl.gov}{dkgunter@lbl.gov}} and 
Ethan Davis\footnote{email: \href{mailto:edavis@ucar.edu}{edavis@ucar.edu}}

\subsection{Group Members}

\begin{itemize}
\item Ethan Davis -- UCAR Unidata
\item Dan Gunter -- Lawrence Berkeley National Lab
\item Liz Jessup -- University of Colorado
\item Mark Miller -- University of California, San Diego
\item Lindsey Powers -- The HDF Group
\item Daniel Ziskin -- NCAR Atmospheric Chemistry Observations and Modeling (ACOM) Laboratory
\end{itemize}

\subsection{Summary of Discussion}

Discussion revolved around a few questions: what is the benefit of having a
``community'' for software sustainability, what practices and circumstances lead
to having and maintaining a community, how can funding help or hinder this
process, and perhaps most importantly, how can best practices be described and
distilled into a document that can help new projects.

The benefits of having a community that were brought up were considered largely
obvious. In addition to having advocates for the software, and a possible source
of ``free'' contributions to the codebase, the community becomes a good source
for requirements, feedback, and metrics. The software community can also act as
``cheerleaders'' who convince funders or other potential users to fund/use the
software, and thus help sustain the software.

Practices and circumstances that lead to a community are first, that the
software offers value. But in addition to this, a community will be much more
likely to form if they receive (expert) support when they have questions.
Additional contributing factors are good usability (not always needed), and an
open development process such as IPython developer meetings on YouTube. It was
also pointed out that an evangelist for the project, not necessarily but often
one of the developers, can often make a big difference.

Funding can help the process by encouraging both value to the community and
high-quality user support. Only providing funding for the software development
may create good software, but with less likelihood to have a real community. It
was discussed that federal laboratories are a good incubator for software
communities, and that a general facility like EarthCube is too dispersed to
really make a community. Also, domain-specific groups within laboratories or
universities might provide as an incubator for software communities.

In describing best practices, the group discussed the different modes for
starting a scientific software project: building on an existing product that
needs improving, recognizing an unsatisfied need of an existing community, or
creating a new solution to a need not yet recognized by the community. The group
also thought that the existing books on software communities would need to be
evaluated in light of differences between Science Software projects and general
OSS projects in terms of scale, science, acknowledgement and credit, and funding
models.


\subsection{Description of Opportunity, Challenges, and Obstacles}

The main opportunity is to increase awareness among scientific
software developers and project managers of the importance of
developing a community around their project.
While this message is fairly well understood in the open source
community, the scientific community can be more focused on the
science a software project is supporting rather than the software
project itself.

As with many of the issues relevant to the sustainability of science
software, the main challenge here will be changing the culture and
expectations around scientific software.

\subsection{Key Next Steps}

The most important next steps is a ``Best Practice'' document, which would
describe what successful projects with engaged communities look like, how to
replicate this type of project, and look at end-of-life on a community project.
Inputs to this document would include a software community survey of highly
functioning communities such as R Open Science, Python SciPy, OPeNDAP, and
Unidata, with analysis of factors that feed into their success. Also references
like the ``Art of Community'' could be adapted and summarized for the science
software community.

More specifically, the group would like to take the following steps:

\begin{itemize}
\item Survey successful science software projects
\item Survey community members from the surveyed projects
\item Distill the survey results and document best practices around community engagement
\item Look for ways to raise awareness
\end{itemize}

Another next step would be increasing recognition of need for science software
projects to focus on building and supporting their user communities. Good software
engineering practices are not enough, and popular training like Software
Carpentry does not currently address this issue head on.

\subsection{Plan for Future Organization}

No definite plans were agreed upon for future organization. The major ideas discussed
were coordinating with another group or adapting some existing text.

Collaboration within the framework of an existing organization seems a good initial
path. Mozilla Science maintains a ``Working Open Project
Guide''~\cite{working-open-wssspe3}, the introduction of which states:
\begin{quote}
Working openly with contributors enables your community to learn how to build
and collaborate together. This document is a guideline on how to work openly and
involve others in your projects with Mozilla. We want to help you engage your
community in a way that encourages contributors and builds other leaders.
 \end{quote}

 Another idea is to form a group that could adapt existing commercial-oriented 
 guidelines for the world of scientific software and top-down funding structures.
 For example, to distill the ``Art of Community'' by Jono Bacon~\cite{art-of-community}
for scientific software.


\subsection{What Else is Needed?}

The group had many points of agreement, but there is not currently a dedicated core group
of people who have committed to producing the key milestones. Coordination via phone or
online would be necessary to build this ``community'' of contributors.

\subsection{Key Milestones and Responsible Parties}

The key milestones for the group's activities align closely with the Key Next Steps above:

\begin{itemize}
\item Complete and write up a survey of project members, and community members, for successful science software projects
\item Distill the survey results and document best practices around community engagement
\end{itemize}


\subsection{Description of Funding Needed}

With a small amount of seed funding, it is possible that members of this group or other parties could
spend the time necessary to devise a survey of existing projects and deploy this, probably including travel to
meetings and workshops for the various software communities.
