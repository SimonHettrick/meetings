%%%%%%%%%%%%%%%%%%%%%%%%%%%%%%%%%%%%%%%%%%%%%%%%%%%%%%%%%%%%
\section{User Community Working Group Discussion}
\label{sec:appendix_user_community}
%%%%%%%%%%%%%%%%%%%%%%%%%%%%%%%%%%%%%%%%%%%%%%%%%%%%%%%%%%%%

\todo{add POC here}

\subsection{Group Members}

\begin{itemize}
\item Ethan Davis -- UCAR Unidata
\item Dan Gunter -- Lawrence Berkeley National Lab
\item Liz Jessup -- University of Colorado
\item Mark Miller -- University of California, San Diego
\item Lindsey Powers -- The HDF Group
\item Daniel Ziskin -- NCAR Atmospheric Chemistry Observations and Modeling (ACOM) Laboratory
\end{itemize}

\subsection{Summary of Discussion}

\subsection{Description of Opportunity, Challenges, and Obstacles}

The main opportunity is to increase awareness among scientific
software developers and project managers of the importance of
developing a community around their project.
While this message is fairly well understood in the open source
community, the scientific community can be more focused on the
science a software project is supporting rather than the software
project itself.

As with many of the issues relevant to the sustainability of science
software, the main challenge here will be changing the culture and
expectations around scientific software.

\subsection{Key Next Steps}

\begin{itemize}
\item Survey successful science software projects
\item Survey community members from the surveyed projects
\item Distill the survey results and document best practices around community engagement
\item Look for ways to raise awareness
\end{itemize}

\subsection{Plan for Future Organization}


\subsection{What Else is Needed?}


\subsection{Key Milestones and Responsible Parties}


\subsection{Description of Funding Needed}
