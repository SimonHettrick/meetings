%%%%%%%%%%%%%%%%%%%%%%%%%%%%%%%%%%%%%%%%%%%%%%%%%%%%%%%%%%%%
\section{Engineering Design Group Discussion}
\label{sec:appendix_eng_design}
%%%%%%%%%%%%%%%%%%%%%%%%%%%%%%%%%%%%%%%%%%%%%%%%%%%%%%%%%%%%

Birgit Penzenstadler\footnote{email: \href{mailto:birgit.penzenstadler@csulb.edu}{birgit.penzenstadler@csulb.edu}} and Colin C. Venters\footnote{email: \href{mailto:c.venters@hud.ac.uk}{c.venters@hud.ac.uk}} will serve as the points of contact for this working group, and be responsible for ensuring timely progress of the planned actions.

\subsection{Group Members}

\begin{itemize}
\item Birgit Penzenstadler -- California State University, CA, USA
\item Colin C. Venters -- University of Huddersfield, Huddersfield, UK
\item Matthias Bussonnier -- UC Berkeley, CA, USA
\item Jeff McWhirter -- Geode Systems 
\item Patrick Nichols -- National Center for Atmospheric Research, CO, USA
\item Ilian Todorov -- Science \& Technology Facilities Council, UK
\item Ian Taylor -- Cardiff University, UK
\item Alexander Vyushkov -- University of Notre Dame, IN, USA
\end{itemize}

\subsection{Summary of Discussion}

The group included members from different backgrounds, including quantum chemistry, epidemiology, computer science, software engineering, and microscopy. Each participant was invited to give their perspective on the topic area and what they thought were the crucial points for discussion. There was a general consensus that there was a need for relating principles to practice for the computational science and engineering community. Furthermore, various members of the group expressed their interest in tools and best practices for facilitating the maintenance and evolution of scientific software systems. It was agreed to identify principles from software engineering and from sustainability design and, based on those lists, discuss what each of those would mean applied to specific example systems from the expert domains of some of the group members. The group identified a number of software engineering principles drawn from the Software Engineering Body of Knowledge (SWEBOK)~\cite{swebokv3}. 

Software design principles included: abstraction; coupling and cohesion; decomposition and modularization; encapsulation and information hiding; separation of interface and implementation; sufficiency completeness \& primitiveness; and separation of concerns. Similarly, user interface design principles included: learnability; user familiarity; consistency; minimal surprise; recoverability; user guidance; and user diversity. The sustainability design principles were drawn from the Karlskrona Manifesto on Sustainability Design~\cite{Becker:2014}. The manifesto states that sustainability is systemic; multidimensional; interdisciplinary; transcends the system's purpose; applies to both a system and its wider contexts; requires action on multiple levels; requires multiple timescales; changing design to take into account long-term effects doesn't automatically imply sacrifices; system visibility is a precondition for and enabler of sustainability design.
A number of sustainable software engineering principles proposed by Tate~\cite{tate:2005} were also considered including: continual refinement of product and project practices; a working product at all times; continual emphasis on design; and value defect prevention over defect detection.

This congregated list is an initial collection of principles that could be extended by adding from further related work form separate disciplines within the field of software engineering, including requirements engineering, software architecture, and testing. The group identified two example systems to discuss the application of the principles. The first one was a quantum chemistry system that allows the analysis of the characteristics and capabilities of molecules and solids. The second one was a modeling system for malaria that permitted biologists to analyze a range of datasets across geography, biology, and epidemiology, and add their own datasets. The group then examined the principles and took a retrospective analysis of what the developers did in practice against how the principles could have made a difference. This raised the question, what do the principles mean for  computational scientific and engineering software? Similarly, how do the principles relate to non-functional requirements? It was suggested that at the very minimum, that sustainable software engineering principles should be mapped to two core quality attributes that underpin technically sustainable software:
\begin{itemize}
\item Extensibility: the software's ability to be extended and the level of effort required to implement the extension;
\item Maintainability: the effort required to locate and fix an error in operational software.
\end{itemize}
These fundamental building blocks could then be extended to include other quality attributes such as portability, reusability, scalability, usability, and energy efficiency etc. Nevertheless, this raises the question of what metrics and measures are suitable to demonstrate the sustainability of the software. In addition, what do the five dimensions of sustainability mean for scientific software, i.e., environmental, economic, social, technical and individual?


\subsection{Description of Opportunity, Challenges, and Obstacles}
The opportunity was identified to distill existing software engineering and sustainability design knowledge into ``bite sized'' chunks for the Computational Science and Engineering Community. In addition, two primary challenges were identified:
\begin{itemize}
\item Mapping of the principles to best practices.
\item Demonstrating the return on investment of those best practices.
\end{itemize}

\subsection{Key Next Steps}
In order to achieve (1) the systematic analysis a number of example systems from different scientific domains with regards to the identified principles, (2) the identification of the commonalities and gaps in applying principles to different scientific systems, and (3) the proposal of a set of guidelines on the principles, the following next steps were discussed:

\subsection{Plan for Future Organization}
The following plan for future organization was discussed:
\begin{itemize}
\item Identify suitable undergraduate or post-graduate students.
\item Design and pilot study.
\item Organizing coordinating online calls via Google Hangout.
\end{itemize}

\subsection{What Else is Needed?}
\begin{itemize}
\item Ethics committee review panel approval required for data collection.
\end{itemize}

\subsection{Key Milestones and Responsible Parties}
The following key milestones were discussed as a roadmap for the set of guidelines on software engineering principles:
\begin{itemize}
\item Oct/Nov 2015: Study design and interview guideline
\item Jan/Feb 2016: Interviews conducted and transcribed
\item Mar/Apr 2016: Analysis complete
\item May 2016: Report written
\end{itemize}

\subsection{Description of Funding Needed}
Specific funding was not discussed in this working group. However, this is a
open topic that can be discussed in relation to emerging funding calls from
National agencies or grant proposal initiatives.
