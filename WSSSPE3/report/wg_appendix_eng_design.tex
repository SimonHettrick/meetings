%%%%%%%%%%%%%%%%%%%%%%%%%%%%%%%%%%%%%%%%%%%%%%%%%%%%%%%%%%%%
\section{Engineering Design Group Discussion}
\label{sec:appendix_eng_design}
%%%%%%%%%%%%%%%%%%%%%%%%%%%%%%%%%%%%%%%%%%%%%%%%%%%%%%%%%%%%

Birgit Penzenstadler\footnote{email: \href{mailto:birgit.penzenstadler@csulb.edu}{birgit.penzenstadler@csulb.edu}} and Colin C. Venters\footnote{email: \href{mailto:c.venters@hud.ac.uk}{c.venters@hud.ac.uk}} will serve as
the points of contact for this working group, and be responsible for ensuring timely progress of the planned actions.

\subsection{Group Members}

\begin{itemize}
\item Birgit Penzenstadler -- California State University, CA, USA
\item Colin C. Venters -- University of Huddersfield, Huddersfield, UK
\item Matthias Bussonnier -- UC Berkeley, CA, USA
\item Jeff McWhirter -- Geode Systems 
\item Patrick Nichols -- National Center for Atmospheric Research, CO, USA
\item Ilian Todorov -- Science \& Technology Facilities Council, UK
\item Ian Taylor -- Cardiff University, UK
\item Alexander Vyushkov -- University of Notre Dame, IN, USA
\end{itemize}

\subsection{Summary of Discussion}

This group discussed the principles of engineering design for sustainable software and their application in various domains. The group included members from different backgrounds, including quantum chemistry, epidemiology, computer science, software engineering, and microscopy. Each participant was invited to give their perspective on the topic area and what they thought were the crucial points for discussion. There was a general consensus that there was a need for relating principles to practice for the computational science and engineering community. Furthermore, various members of the group expressed their interest in tools and best practices for facilitating the maintenance and evolution of scientific software systems. It was agreed to identify principles from software engineering and from sustainability design and, based on those lists, discuss what each of those would mean applied to specific example systems from the expert domains of some of the group members. The group identified a number of software engineering principles drawn from the SoftWare Engineering Body of Knowledge (SWEBOK):

\subsection{Description of Opportunity, Challenges, and Obstacles}

\subsection{Key Next Steps}


\subsection{Plan for Future Organization}


\subsection{What Else is Needed?}


\subsection{Key Milestones and Responsible Parties}


\subsection{Description of Funding Needed}
