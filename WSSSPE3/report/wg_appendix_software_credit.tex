%%%%%%%%%%%%%%%%%%%%%%%%%%%%%%%%%%%%%%%%%%%%%%%%%%%%%%%%%%%%
\section{Software Credit Working Group Discussion}
\label{sec:appendix_SW_credit}
%%%%%%%%%%%%%%%%%%%%%%%%%%%%%%%%%%%%%%%%%%%%%%%%%%%%%%%%%%%%

Kyle Niemeyer\footnote{email:
\href{mailto:kyle.niemeyer@oregonstate.edu}{kyle.niemeyer@oregonstate.edu}} will
serve as the point of contact for this working group, and be responsible for
ensuring timely progress of the planned actions.

%%%%%%%%%%%%%%%%%%%%%%%%%%%%%%%%%%%%%%%%%%%%%%%%%%%%%%%%%%%%
\subsection{Group Members}
%%%%%%%%%%%%%%%%%%%%%%%%%%%%%%%%%%%%%%%%%%%%%%%%%%%%%%%%%%%% 

\begin{itemize}
\item Alice Allen -- Astrophysics Source Code Library 
\item Sou-Cheng Choi -- NORC at University of Chicago, Illinois Institute of Technology
\item James Hetherington -- University College London
\item Lorraine Hwang -- University of California, Davis
\item Daniel S.\ Katz -- University of Chicago, Argonne National Laboratory
\item Frank L\"{o}ffler -- Louisiana State University
\item Abigail Cabunoc Mayes -- Mozilla Science Lab
\item Kyle E.\ Niemeyer -- Oregon State University
\item Grace Peng -- National Center for Atmospheric Research
\item Ilian Todorov -- Science \& Technology Facilities Council, UK
\end{itemize}


%%%%%%%%%%%%%%%%%%%%%%%%%%%%%%%%%%%%%%%%%%%%%%%%%%%%%%%%%%%%
\subsection{Summary of Discussion}
%%%%%%%%%%%%%%%%%%%%%%%%%%%%%%%%%%%%%%%%%%%%%%%%%%%%%%%%%%%%

The following section summarizes the working group's discussion based on
contributions prior to the meeting~\cite{WSSSPE3-SC-github-issues} and the
collaborative notes taken during the meeting~\cite{WSSSPE3-SC-google-notes}.
Please refer to the original sources for the unedited discussions if necessary.

Initial discussions focused on both various mechanisms for, and the
philosophical approach behind, crediting software in scientific papers. These
began with proposals for various ways to credit software (or other research
products including data) that contributed more significantly than a generic
citation, including:
\begin{itemize}

\item A hierarchy of citations, with a ``substantial'' citation category to
indicate software or data that played a more significant role in the research;

\item Transitive credit~\cite{wssspe2_katz,Katz:2014_tc}, which assigns
contriponents (contributors and components) various weights depending on their
level of importance;
    
\item Project CRediT~\cite{projectcredit}, which assigns roles to paper authors
based on their specific contributions; and
    
\item Mozilla Science Lab's recently introduced Contributorship Badges for
Science~\cite{Mozilla_badges}, which provide a badge---associated with an
ORCID~\cite{orcid}---that recognizes author contributions using the taxonomy
outlined in Project CRediT.
    
\end{itemize}
However, as of this writing, only Project CRediT
roles~\cite{McCall2015_credit,Lin2015_credit} and Contributorship
Badges~\cite{Mozilla_badges} have been implemented for published papers, and
both of these only provide a single ``Software'' or ``Computation'' category
associated with software. In addition, neither of these options allows for the
citation of software itself, but only provide an author contribution related to
software. The discussion quickly focused on transitive credit as a more
quantitative measure of allocating credit to both authors and software, although
there were some concerns about authors overestimating their own contributions
compared to prior work.

The discussion then evolved into philosophical questions about the importance or
reliance of a particular work on prior science, materials, or software---in
other words, whether there is a difference between depending on prior scientific
advances and depending on certain software (or experimental equipment).
Multiple contributors converged on the conclusion that unique capabilities
require some additional credit. The---albeit limited---consensus was that if a
particular study relied on the unique capabilities of software, data, or an
experimental apparatus, then the authors or developers that created this
capability should be credited somehow.

The group also agreed on the fact that additional data was required to support
the assertion that software was not being sufficiently cited in the literature.
In particular, this issue seemed to be field-dependent. For example, as shown by
a study of Howison and Bullard~\cite{Howison2015}, in the field of biology, the
most-cited papers appear to be those describing scientific software. However,
this may not---and likely is not---the case in other fields, nor is it clear
whether developers of scientific software, even in the case of the biology
field, are receiving sufficient credit for their efforts.

In the breakout sessions on the first day of WSSSPE3, the group discussed and deliberated over the
Entertainment Identifier Registry (EIDR)~\cite{EIDR} as a potential model for
scientific software. That system assigns unique Digital Object Identifiers
(DOIs)---the same system used for scientific publications---to all content
(e.g., movies, television shows) and contributors, along with relevant metadata.
One important use of the EIDR system is to track rights and credits for contributors
to entertainment works in order to distribute revenues---similar to the proposed
transitive credit concept.

The group also discussed separating quantitative measures (e.g., number of
citations) from the value of a work in order to give credit, moving towards
qualitative or anecdotal evidence of value. Other topics that were brought up included a
form of PageRank~\cite{Brin1998} for citations, based on number of mentions, and
using market penetration or adoption rate in a community as a metric, although
it was not clear how this would be measured. Finally, the concept a software
tool's uniqueness or indispensability to a community was mentioned, with value
being characterized by a particular piece of software either offering unique
capabilities or doing something better, faster, or with less computational
requirements than other offerings.

On the second day of WSSSPE3, the group decided to put aside the
taxonomy of contributions and focus on software citations to ensure developers
receive credit (regardless of contribution). Eventually, once 
software citations are standardized, the goal would be to return to establishing different
roles\slash contributions for this credit. Following this decision, the group
identified two necessary actions to move forward:
\begin{enumerate}

\item standardizing a citation file or some other form of metadata associated
with software, and

\item standardizing the way to cite software (used directly) in papers.
        
\end{enumerate}
For both of these actions, a number of ongoing efforts were identified and discussed.

\subsubsection{Software Citation Metadata}

At a minimum, the metadata required for software citation includes:
\begin{itemize}
    \item Name of software,
    \item DOI for software,
    \item Contributors, in the form of names and ORCIDs,
    \item Software dependencies, in the form of DOIs, and
    \item Other people and artifacts that would be cited or acknowledged in a paper.
\end{itemize}
This information would then be contained in a citation file, e.g., as part of
the GitHub repository. The group also discussed similar efforts such as
CodeMeta\footnote{CodeMeta: \url{https://github.com/codemeta/codemeta}}, an
attempt to codify minimal metadata schemes in JSON and XML for scientific
software and code, and implementing transitive credit via
JSON-LD~\cite{wssspe2_katz}. Some questions arose about how this information
would be stored for closed-source software.

As one mechanism for constructing accurate contributor lists from existing
project contributors, the group discussed associating GitHub accounts---as well
as accounts on Bitbucket, CodePlex, and other repositories for open-source
scientific software---with ORCID accounts. However, a (quick) response from
GitHub (via Arfon Smith) indicated that this might not be possible in the near
future: ``GitHub doesn't have any plans to allow ORCID accounts to be associated
with GitHub user accounts.''

\subsubsection{Citing Software in Publications}

Although far from a standard practice, examples of citing software in
publications can be found in various scientific communities---notably,
representative samples can be found in astronomy~\cite{astronomy_SW_examples}
and biology~\cite{Howison2015}. The group recommended collecting similar
examples from other communities, and then developing a software citation principles
document in concert with the FORCE11 Software Citation Working Group (see
\S\ref{SC:plan} for more details), following the model of the FORCE11 Data
Citation Principles document~\cite{DataCitation2014}.

The group further discussed briefly whether software used directly in a
publication---whether to perform simulation or analysis, or as a dependency for
newly developed software---should be distinguished from other references due to
the dependence of the study on these research artifacts. Suggestions included a
separate list of citations (with DOIs) for software and other research objects
that serve this sort of ``vital'' role. Similar recommendations were made by the
credit breakout group at WSSSPE2~\cite{WSSSPE2}.

Finally, although a discrete task from software citations,
significant discussion focused on ensuring software citations are indexed in the
same manner as publications, allowing the construction of a corresponding
software citation network. Currently, software releases can receive citable DOIs
via Zenodo~\cite{zenodo-web} and figshare~\cite{figshare-web}; however, these
citations are not processed by indexers such as Web of Science, Scopus, or
Google Scholar. Thus, either in parallel or following the primary task, the
group will need to reach out to these organizations. Initial conversations with
Elsevier\slash Scopus via Michael Taylor during WSSSPE3 clarified that Scopus
is not yet DataCite DOI aware, and also does not yet have an internal identifier
for software or data (but needs\slash plans to add this support). Taylor said
they prefer a ``software article'' with the usual article metadata (e.g.,
authors, citations), and mentioned Zenodo as an example -- this proposal seemed
to align with our group's discussions. Taylor also mentioned another benefit of the
software and associated DOI on GitHub: in addition to a citation, one could obtain
statistics on usage/downloads/forks, which happens to be what
Depsy\footnote{Depsy: \url{https://depsy.org}} is beginning to try to do.

%%%%%%%%%%%%%%%%%%%%%%%%%%%%%%%%%%%%%%%%%%%%%%%%%%%%%%%%%%%%
\subsection{Description of Opportunity, Challenges, and Obstacles}
%%%%%%%%%%%%%%%%%%%%%%%%%%%%%%%%%%%%%%%%%%%%%%%%%%%%%%%%%%%%

There currently is no standard mechanism for citing software or
receiving credit for software (akin to citations for publications). Software is
eligible for DOI assignment, but DOI metadata fields are not well tuned or
standardized for software (vs.\ publications). Some software providers apply for
DOIs, but this is not widely adopted. Also, there is no mechanism to cite
software dependencies within software.

Major obstacles include the fact that indexers (e.g., Scopus, Web of Science,
Google Scholar) do not currently support software/data document types or
DataCite DOIs. Therefore, even with universal association of scientific software
with DOIs and standardized practices for citing software in publications,
software citations will not be indexed in the same manner as traditional
publications.

Although this working group's discussions at WSSSPE3 did not focus much on the
topic of tenure and professional advancement, 
%we should recognized that this problem goes beyond faculty to recognizing the role of ``Research Software Engineers''
the group recognized that there is no standard policy---generally even within a
single university---for software products to be included in promotion and tenure
dossiers. Thus, it may be difficult to encourage valuing software contributions
across the United States or United Kingdom and globally; furthermore,
stakeholders are typically not tenured and thus may not be influential enough to
change the status quo. However, as discussed in Section~\ref{RSE}, this is
changing for Research Software Engineers, at least in the UK.

%%%%%%%%%%%%%%%%%%%%%%%%%%%%%%%%%%%%%%%%%%%%%%%%%%%%%%%%%%%%
\subsection{Key Next Steps}
\label{SC:next-steps}
%%%%%%%%%%%%%%%%%%%%%%%%%%%%%%%%%%%%%%%%%%%%%%%%%%%%%%%%%%%%

\begin{enumerate}

\item Hold virtual meeting to determine group members responsible\slash willing
to work on the following tasks, to be organized within one month of the workshop.

\item Compile best practices of software citation across multiple disciplines,
including journals and communities of interest\slash practice in the research
world, to begin by December 2015.

\item Compile examples of including other products in promotion and tenure
dossier, to begin by December 2015.

\item Draft the Software Citation Principles document (including citation
metadata file), by April 2016.

\item Publish\slash release the Software Citation Principles document, by August
2016.

\item Reach out to journals, publishers, teachers\slash educators, indexers, and
professional societies---likely through meetings with key groups, to begin by
September 2016.

\end{enumerate}

%%%%%%%%%%%%%%%%%%%%%%%%%%%%%%%%%%%%%%%%%%%%%%%%%%%%%%%%%%%%
\subsection{Plan for Future Organization}
\label{SC:plan}
%%%%%%%%%%%%%%%%%%%%%%%%%%%%%%%%%%%%%%%%%%%%%%%%%%%%%%%%%%%%

The WSSSPE breakout group plans to join efforts related to citing software with
the FORCE11 Software Citation Working Group (FORCE11-SCWG)\footnote{FORCE11
Software Citation Working Group,
\url{https://www.force11.org/group/software-citation-working-group}}; Kyle
Niemeyer formally requested the merging of these groups following the meeting.
However, some future plans of the WSSSPE group fall outside the scope of
FORCE11-SCWG, which covers software citation practices. These activities include
working with indexers such as Web of Science and Scopus to index software
citations archived on, e.g., Zenodo or figshare, and pursuing the development of
an open indexing service; such plans will be pursued either separately or
through the formation of follow-on FORCE11 working groups.

The group will primarily communicate electronically, with Kyle Niemeyer
responsible for ensuring regular progress.

%%%%%%%%%%%%%%%%%%%%%%%%%%%%%%%%%%%%%%%%%%%%%%%%%%%%%%%%%%%%
\subsection{What Else is Needed?}
%%%%%%%%%%%%%%%%%%%%%%%%%%%%%%%%%%%%%%%%%%%%%%%%%%%%%%%%%%%%

The near-term actions of the group, focused mainly on software citation, do not
require any additional resources. However, connections with publishers and
indexers will be needed to pursue related activities, although the FORCE11-SCWG
may satisfy this need; in addition, some members of the group already reached
out to relevant contacts. Funding may be needed to organize meetings or for
group members to attend relevant meetings, as discussed further below.

%%%%%%%%%%%%%%%%%%%%%%%%%%%%%%%%%%%%%%%%%%%%%%%%%%%%%%%%%%%%
\subsection{Key Milestones and Responsible Parties}
%%%%%%%%%%%%%%%%%%%%%%%%%%%%%%%%%%%%%%%%%%%%%%%%%%%%%%%%%%%%

Following the meeting, Kyle Niemeyer formally requested the merging of software
citation activities with FORCE11-SCWG. Within a month of the meeting, Niemeyer will
organize a virtual meeting of the group and manage the division of
responsibilities for compiling existing practices of software citation and
including software\slash products in promotion and tenure dossiers. Building off
of these efforts, the next major milestone is drafting the Software Citation
Principles document in collaboration with the SCWG, targeted for April 2016.
While the existing directors of the SCWG, Arfon Smith and Dan Katz, lead the
efforts of that group towards the Software Citation Principles document, Kyle
will help coordinate contributions from the WSSSPE group members.


%%%%%%%%%%%%%%%%%%%%%%%%%%%%%%%%%%%%%%%%%%%%%%%%%%%%%%%%%%%%
\subsection{Description of Funding Needed}
%%%%%%%%%%%%%%%%%%%%%%%%%%%%%%%%%%%%%%%%%%%%%%%%%%%%%%%%%%%%

Some funding would be useful to support primarily travel to conferences for
group meetings (e.g., FORCE2016\footnote{FORCE2016,
\url{https://www.force11.org/meetings/force2016}}), and to hold meetings to
bring together both group members and key stakeholders (e.g., journals,
publishers, professional societies, indexers). In addition, funding would be
desired to support group members' time to perform work towards the key steps
described previously.
