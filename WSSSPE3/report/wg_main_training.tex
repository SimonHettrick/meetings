\subsection{Training}

%\subsubsection{Why it is important}

This group explored a rapidly growing array of training that is seen to
contribute to sustainable software. The offerings are diverse, providing
training that is more or less directly relevant to sustainable software. While
research institutions support professional development for research staff, the
skills taught which might impact on sustainable software are limited at best,
often lacking a clear and coherent development pathway. This growing array of
training opportunities could usefully be coordinated by bringing together those
involved in leading relevant initiatives on a regular basis.

\subsubsection{Fit with related activities} Three existing venues for discussion
of related activities are identified:

\begin{itemize}

\item Working towards Sustainable Software for Science: Practice and
Experiences (WSSSPE) workshops~\cite{WSSSPE}

\item International Workshop on Software Engineering for High
Performance Computing in Computational Science and
Engineering (SEHPCCSE)~\cite{SEHPCCSE}

\item Workshop on Software Engineering for Sustainable Systems~\cite{se4susy}

\end{itemize}

\subsubsection{Discussion}

Next steps have been identified to quickly test whether there is interest in
establishing a community committed to increasing the degree of coordination
across training projects. See Appendix~\ref{sec:appendix_training} for more details about the discussion.

\subsubsection{Plans}

The main plan for the group is to convene a discussion to explore bringing
together regular meetings of those involved in leading relevant training
projects.

\subsubsection{Landing Page}

The Training working group requests an email be sent to Nick Jones
\href{mailto:nick.jones@nesi.org.nz}{(nick.jones@nesi.org.nz)} to find out more
about the group's efforts and how to participate.
