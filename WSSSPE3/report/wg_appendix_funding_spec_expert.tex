%%%%%%%%%%%%%%%%%%%%%%%%%%%%%%%%%%%%%%%%%%%%%%%%%%%%%%%%%%%%
\section{Funding Specialist Expertise Group Discussion}
\label{sec:appendix_funding_spec_expert}
%%%%%%%%%%%%%%%%%%%%%%%%%%%%%%%%%%%%%%%%%%%%%%%%%%%%%%%%%%%%

\todo{add POC here}

This is where the bit I need to write will go.

\subsection{Group Members}

Don Middleton
Josh Greenburg: SLOAN
James Hetherington: University College London j.hetherington@ucl.ac.uk
Lindsay Powers: The HDF Group, lpowers@hdfgroup.org
Mark Miller
Dan Sellars: CANARIE

\begin{itemize}
\item name -- affiliation
\item name2 -- affiliation2
\end{itemize}

\subsection{Temporary direct paste-in of notes from Workshop}

Our charge: document successful funding models as a means of creating a conversation that can be used in finding funding for new efforts;
Don Middleton
Josh Greenburg: SLOAN
James Hetherington: University College London j.hetherington@ucl.ac.uk
Lindsay Powers: The HDF Group, lpowers@hdfgroup.org
Mark Miller
Dan Sellars: CANARIE: Canada; like internet 2, they make some funding available to do projects.
Fund a PI /team to produce a platform to do the work.
Cultural barrier is steep for communicating across barriers, and scientists seeing the developers as “other”. SO collaboration, and tools to fund/encourage/motivate collaboration are extremely important.
Projects that engage communities on an “at will” basis
One model is to engage scientists in the development process as part of a community effort, and that may be largely unfunded. Models that are community driven can be very organic, so new contributions are possible from unexpected places.  
Funded engagement Models
It is usually more effective and efficient to do development where there is funding to support the collaboration. 
Co-funding on a single proposal.
One way to do this is by co-funding individuals from both camps, that is domain scientists and developers are funded together in a single proposal. At NCAR, University partnerships are required for submission of proposals, so collaboration is an essential part of grant submission, and this will tend to bring developers and scientists together. This is certainly an effective method for promoting project based/project scale work. We aren’t sure how this requirement could be imposed globally, but it should and probably often is required by any given RFP for SW development. One example of a success is the development of semantics and linked data in support of Ocean Sciences. An Earthcube funded project pairs domain scientists with RSEs and has been successful as the semantics attached have increased data use and discovery significantly.
Developer support from a resource provider to augment existing funding.
We discussed another fundamental strategy used by several organizations, where a group that wants to promote use of a new resource, be it internet (Canarie) or one or more compute engines (XSEDE (US) and Archer (UK)) fund a developer team that works directly with scientists from outside the organization to promote the use of the new resource.
The Canarie group (Canada) accepts proposals for providing services to broad communities, integrate people who are doing things that are complementary, the goal is to make the available stack more robust and richer for everyone. They offer short rounds of funding that can have as a key metric creating some useful functionality that shows a diversity of input and draws from across disciplines, then more funding could follow. This could apply within or across institutions.
The UK allows this kind of collaboration to support the creation of scientific software for the large supercomputing resource called Archer. The Archer project fund a group of developers, and domain scientists can apply to the group for developer time to help them launch a new tool on Archer.
The US XSEDE project does essentially the same thing. The extended collaboration service (ECS) is a set of developers who are paid with XSEDE funding, and are on “permanent” staff.  When PIs request allocations on XSEDE resources, there is a finite pool of developer time that can be awarded, typically for one year only, and at partial effort, typically 20% or so. The finite time allowed provides motivation for the scientist and their group to work closely the developer and to become educated in what the developer is doing, so they can sustain the effort, once the ECS period is over.
This funding mechanism is highly efficient for scientific problems, because the developer pool assembled by the research providers are, by definition, expert in the characteristics of their specific resource, and can very quickly assess the scientists needs, and what it will take to implement software that meets the user’s needs. Moreover, when an organization assembles a developer pool, there will be diversity in the talent pool, so issues that require multiple kinds of expertise can be addressed by distinct individuals, again with maximum efficiency.
Promoting institutional support for research science developers
The Sloan Foundation is currently funding data science engineers, who work in the context of other SW developers at University of Washington.  These scientists work in the e-Science Studio/dataScience Studio, and they help a group of graduate students in solving their problems in data science/data management. During Fall and Spring, a 10-week incubator program allows students to work two days a week to work on a data-intensive science project. Some fraction of the developer time is dedicated to the developers personal interests as well as instruction. Piloting and documenting what works and what doesn’t.
The goal for Sloan is to provide success stories, to provide demonstrable value in the presence of data scientists on university staff. These stories are the basis for arguments to the host organization. This is an effort to create awareness of the value of research scientist developers. Embedding with scientists, and adding spare capacity is critical to making the innovation possible. This model is essentially to argue for permanent budget lines to support data scientists as part of university staff hires, just as with core facilities. This could become a fee-for-service model requested by grant funding, just as DNA sequencing is for core facilities, if it becomes apparent that this gives competitive advantage to the University’s research effort.
Researchers wont be able to do this without seed money to get it going. James succeeded at his institution which was building cyberinsfrastructure. Three posts were funded by the university. This is budget line for professional developers. Currently developer support for individual projects is awarded by a panel. Researchers can contribute funds that are left over from other awards where people leave prematurely. This is happening currently.  Some people committed to using funds in an existing award. There is a UK fellowship for replicating this basic concept: the award is for salary for one position as team leader,  and a first team member; these awards are for five years duration (woohoo!). There are other RSEs in other UK institutions. These may engage in labor exchange via  informal work sharing agreements between institutions. Three month engagement preferred, but not everyone works that way some prefer quick in and done modes. 
To make such a program work, it seems it  must be university wide, to be broad enough to have the ftes needed funded. But there is not one mind on this, it maybe that the broadest approach is not idea. Perhaps a specialization focus on developers to support physical or biological sciences may be preferable, if the customer base is large enough. How to aggregate enough work to make it sustainable. How do you make sure you have the right staff skills? This will be an issue in a smaller group, and it will be exacerbated if the approach is general rather than focused on a domain emphasis where the number of possible skills required is less. The optimum use of resources seems to be 40% effort, on a project; two projects per developer. Two developers per project seems to be ideal, in the sense that software development is enhanced by two pairs of eyes..
Collaborative computational projects (CCP) to support community networking: domain specific communities put forward proposals that are a priority of the community as a whole: biosimulation plasma PHYSICS QUANTUM PHYSICS. % years of funding. Custodians of community codes. Software support. help these are incubators. Add new features. Software of the future, open to all. Ccp submits. 15 ftes for development across all CCPs. GET IAIN. 4% success rate for software of the future project, high end computing consortia, in domains of CCPs. Co-design; the art of bringing it into existence together. The mission is codesigning 5 projects: the collaboration did work, but not as intimately as it might have.  Incentive structures are still based on getting papers out, and the benefits to the postdocs are career focused.
RSE fellowships: 204 applications, 5 funded. Intention was to fund applicants that demonstrated broad benefit to community/ies. How to better incentivize the collaboration to serve a broader community? Challenge: Scientists do not appreciate the engineering issues, whereas the cs person does not necessarily have the understanding of the domain. 
Possibility of a training course for research software engineering, a course could be run to teach people to manage a grant that is focused on software development. Non-hands on software carpentry for PIs.
What are the next steps:
Have we actually been successful?
What are the metrics for success?
Publications are tracked in James organization.
How to evaluate the productivity gains that are made as a result of this engagement with James’ group?
How to evaluate what did not go wrong because of this intervention?
Notre Dame and LSU groups who are RSE groups.
NCEAS/SOCSYNC; some RSE support; NESCENT/NIMBIOS also.
The power of the label? Getting the word out about RSE support using the label.
 

 
 
 
 
 
 	
Will research science developers be required, in the long run?
One issue that came up was whether the need for developers was a time bounded one; is it the case that the new generation of computer and software savvy scientists will be so comfortable in developing their own code that the professional developer will not be needed.  And this brings up the flip side question, “Do scientists really want to be writing code?”
Career Path.
We also had a little discussion about how to make a career path for research developers. It need not be solely an academic enterprise, but tenure is always problematic for people of this class. There is a need for more senior people to have a track. Retention is hard, because these people are highly in demand by the economy, so retention is hard. People who have a PhD in science and enter industry, may return for diverse reasons.
Things we haven’t started talking about:
Other above the campus models for funding;
Industry:academic partnerships


Follow-on Discussion at GCE15
rse.ac.uk
People: Lorraine Hwang, Smon Trigger, Nancy Wilkins-Diehr,  Alex Vyshkov, Sandra, Keynote person from Zooniverse  

Good example: Notre Dame (14 research programmer), LSU, Princeton, 
SDSC -- actually even in national labs there are inefficiencies with per-project jobs rather than develop pool -- potential issue is degree of control. 
Matt Trunnel - CIO of Broad institute boston - Chris Dwan - they had individual RSEs, then aggregated. 
James Taylor’s team  - John’s Hopkins
Org models - OVPR, CS, Physics, IT, Library
IT Coalescence issues.
Maybe Yale?



\subsection{Summary of Discussion}

\subsection{Description of Opportunity, Challenges, and Obstacles}


\subsection{Key Next Steps}


\subsection{Plan for Future Organization}


\subsection{What Else is Needed?}


\subsection{Key Milestones and Responsible Parties}


\subsection{Description of Funding Needed}
