%%%%%%%%%%%%%%%%%%%%%%%%%%%%%%%%%%%%%%%%%%%%%%%%%%%%%%%%%%%%
\section{Funding Research Programmer Expertise Group Discussion}
\label{sec:appendix_funding_spec_expert}
%%%%%%%%%%%%%%%%%%%%%%%%%%%%%%%%%%%%%%%%%%%%%%%%%%%%%%%%%%%%

James Hetherington\footnote{email:
\href{mailto:j.hetherington@ucl.ac.uk}{j.hetherington@ucl.ac.uk}} will serve as
the point of contact for this working group, and be responsible for ensuring
timely progress of the planned actions.

\subsection{Group Members}

The group at WSSSPE:

\begin{itemize}
\item Don Middleton -- National Center for Atmospheric Research
\item Joshua Greenberg -- Alfred P. Sloan Foundation
\item James Hetherington -- University College London
\item Lindsay Powers -- The HDF Group
\item Mark A. Miller -- San Diego Supercomputer Center
\item Dan Sellars -- CANARIE
\end{itemize}

This was further enhanced by additional discussions at the following
GCE15 conference:
  
\begin{itemize}
\item Lorraine Hwang -- UC Davis
\item Simon Trigger -- BioTeam, Inc.
\item Nancy Wilkins-Diehr -- San Diego Supercomputer Center
\item Alexander Vyushkov -- University of Notre Dame
\item Sandra Gesing -- University of Notre Dame
\item Ali Swanson -- University of Oxford

\end{itemize}

\subsection{Summary of Discussion}

In addition to the points noted in the main discussion (\S\ref{RSE}), we also
discussed the following:

``Are you an RSE or a RA?'' is not properly a binary question. Most of
us sit at different points on that spectrum, and move along it during our
careers (usually from RA to RSE -- examples of movement in the other direction
from readers would be welcomed). Either way, the label ``Research Software
Engineer'' is now starting to have some power. Many scientists do not want to be
writing code; some do, to varying degrees. These groups can usefully support
each other.

What is the power of the label? How can we get the word out about RSE support using the label?

Will research science developers be required in the long run? One issue that
came up was whether the need for developers was a time bounded one; is it the
case that the new generation of computer and software savvy scientists will be
so comfortable in developing their own code that the professional developers will
not be needed? And this brings up the flip side question, ``Do scientists really
want to be writing code?''

We also had a little discussion about how to make a career path for
research developers. It need not be solely an academic enterprise, but in the
past tenure has often been problematic for people of this class.

Skills and resources may vary between teams. To help resolve this, maintaining
high levels of communication between groups will be valuable. In the United Kingdom (UK), there
are plans to permit resource sharing between institutional RSE groups. Perhaps
there are circumstances under which an RSE skill exchange could be arranged,
either formally or informally.

Collaborative funding can be important for RSE groups, to ensure that research
leadership remains with the domain scientists. At NCAR, university partnerships
are required for submission of proposals, so collaboration is an essential part
of grant submission, and this will tend to bring developers and scientists
together. The UCL group also follows this approach, with all bids requiring an
academic collaborator.

Domain scientists and developers are funded together in a single proposal.
Another example of a success is the development of semantics and linked data in
support of ocean sciences. An EarthCube-funded project pairs domain scientists
with RSEs and has been successful; the semantics attached have increased data
use and discovery significantly.

An alternative approach has been the provision of programming expertise as part
of national compute services. The US XSEDE project's Extended Collaboration Support 
Services (ECSS) is a set of developers who are paid with XSEDE funding, and are
on ``permanent'' staff. When PIs request allocations on XSEDE resources, there
is a finite pool of developer time that can be awarded, typically for one year
only, and at partial effort, typically 20 percent or so. The finite time allowed
provides motivation for the scientist and the scientist's group to work closely with the
developer and to become educated in what the developer is doing, so they can
sustain the effort once the ECSS period is over. This funding mechanism can be
highly efficient for scientific problems, because the developer pool assembled
by the research providers are, by definition, expert in the characteristics of
their specific resource, and can very quickly assess the scientist's needs, and
what it will take to implement software that meets the user's needs. However, it
does not develop capacity within institutions, and since XSEDE is a time-bounded
program, it should not be relied upon as a long-term solution to acquiring this
type of capacity.

The UK allows this kind of collaboration to support the creation of scientific
software for the large supercomputing resource (ARCHER). However, while the
support can come from the staff of the Edinburgh Parallel Computing Centre,
who hosts the computer, this ``embedded CSE'' resource also funds the
programming coming from local groups. This has been very helpful in providing
funding to establish local groups. These groups work best when they develop good
collaborations with national cyberinfrastructure pools. When an organization
assembles a developer pool, diversity is developed and skills can be
transferred.

We would like to see these models applied outside high performance computing.
Most scientific software is not destined to run on national cyberinfrastructure,
but needs similar support. The argument regarding making better use of expensive
hardware through software improvements has been useful politically, (and many
RSE groups are cited in organizations which host clusters for this reason), but
the time has come to make the case that software itself is a critical
cyberinfrastructure, and, with a much longer shelf-life than hardware, is itself
a capital investment.

The CANARIE group (Canada) accepts proposals for providing services to broad
communities, integrating people who are doing things that are complementary. The
goal is to make the available stack more robust and richer for everyone. They
offer short cycles of funding for creating some useful
functionality that shows a diversity of input and draws from across disciplines as a key metric,  If this metric is met successfully, 
then more funding may follow. This can apply within or across institutions.

There can be problems communicating across cultural barriers, with domain
scientists seeing developers as ``other''. Both collaboration and tools to fund,
encourage, or motivate collaboration are extremely important. 

We think support from non-governmental organizations will be important if RSE
groups will become established. The Sloan Foundation is currently funding data
science engineers, who work in the context of other software developers at the University
of Washington. These scientists work in the e-Science Studio/Data Science
Studio, and they help a group of graduate students in solving their problems in
data science and data management. During Fall and Spring, a 10-week incubator
program allows students to work two days a week on a data-intensive
science project. Some fraction of the developer time is dedicated to the
developers' personal interests as well as instruction.

The goal for Sloan is to obtain success stories and demonstrable value
of the presence of data scientists on university staff. These stories are the
basis for arguments to the host organization. This is an effort to create
awareness of the value of research scientist developers. Embedding with
scientists, and adding spare capacity is critical to making the innovation
possible. This model is essentially to argue for permanent budget lines to
support data scientists as part of university staff hires, just as with core
facilities. This could become a fee-for-service model requested by grant
funding, just as DNA sequencing is for core facilities, if it becomes apparent
that this gives competitive advantage to a university's research effort.

One model that has been helpful in finding funding for RSE groups is the use of
funds left over on research grants when RAs have left prematurely -- PIs like
this arrangement as it is hard to find good staff for short-term positions, so
having a pool of research programming staff on hand resolves this problem. We
recommend that funders give explicit guidance to grant holders and institutions
that such arrangement are favorable. Framework agreements permitting this to go
ahead without checking back every time with funders and/or grant panels would
further smooth this. (This also provides more stable jobs for those who hold
these skills, but arguments about making life nicer for postdocs will not help
persuade funders or PIs!)

There is some question about the most effective duration and percent of full
time for a programmer's work on a project. At least three months is necessary
for the programmer to read into the science (RSEs must not become so disengaged
from research that they do not have time to read a few papers -- this will result
in code which does not meet scientific needs), but too long could result in an
RSE losing their flexibility, becoming so engaged in one project that when that
project ends, they find it hard to transfer. For this reason, we also recommend
that 40 percent is ideal; two projects per developer, with some time for
training and infrastructure work. Two developers per project seems to be ideal,
in the sense that software development is enhanced by two pairs of eyes.

There is, as yet, no clear answer as to the scale of aggregation needed to make
such a program work. A university wide program allows enough scale to be robust
to fluctuations of funding within one field. But a specialization focus on
developers to support, for example, physical or biological sciences may be
preferable, if the customer base is large enough. The desire to aggregate enough
work to make it sustainable, and the need to have domain-relevant research
programming skills, are in tension.

In the UK, another source of funding for research software has been the
Collaborative Computational Projects (CCPs): domain specific communities put
forward proposals that are a priority of the community as a whole, for example,
biosimulation or plasma physics. These bodies act as custodians of community
codes, and a central team also provides software engineering support.

However this area develops, the need for funding for software as a
cyberinfrastructure component is clear.  Funding that permits code to be
refactored, tidied, and optimized is rare; this is often done ``on the sly'' in a
scientifically focused grant. The UK EPSRC's ``software for the future'' call,
which really permits explicit investment in software as an infrastructure, is so
oversubscribed as to have a 4\% success rate; the demand is clear!

One opportunity is the idea of co-design, where infrastructural libraries are
developed alongside the scientific codes that will call them. However,
collaboration is hard to foster here; as incentive structures are still focused
on short-term papers. This can cause infrastructure developers to focus more on
publications in their areas of mathematics and computer science, the domain developers on the
shorter-term needs of their own fields. Genuine collaborative co-construction is
harder to foster.

It can be a more difficult to help leading domain scientists see the value of
engineering effort than those in their teams who are forced to work with
difficult-to-use or unreliable software tools, as they do not see the pain.
Perhaps a version of ``software carpentry'' targeted at those PIs who are
awarded or apply for software-intensive grants could be valuable here.

RSEs provide a useful contribution to their universities' teaching
missions, as well as research, as they are well placed to deliver the
research programming training that many scientists now need. In the longer term,
with programming skills taught to all through their careers, we hope specialist
scientific developers will be less needed.

\subsection{Key Next Steps}

We will seek to identify and approach existing research programming organizations,
to get their permission to list them on a list of research software groups.
Casual conversation during the meeting made it clear that although the title is
not widely used in the US, this position is not rare. We spoke with several
individuals who, at distinct universities, had RSEs (in effect if not in name)
who were funded under differing models.

We will also look for examples of groups which have successfully become self-
sustaining following initial seed funding.

In this respect, information gathering via a survey and subsequent analysis could be
very useful. We would need to assemble a list of targeted individuals. (What
positions and ranks are likely to know and care enough to respond?) Perhaps the
Science Gateway Institute has already acquired information that could be helpful
to advance this issue, and/or craft a proper survey and suggest target individuals.

\subsection{Plan for Future Organization and Future Needs}

The UK RSE community will provide initial facilities to host this list, and
continue to work to spread the initiative, but local leadership in the US is
needed if this campaign is to succeed. This will require an initial gathering of
identified research software organizations in the US to this end.

\subsection{Description of Funding Needed}

Financial support for an initial conference to bring together research software
groups to form an organization and create a resource sharing structure would
help to further this campaign. Funding to conduct and analyze a survey could
also be quite useful as knowing where we stand today, and what models are in use
could fuel the ideas for further development of developers in this category.

In the longer term, funding organizations, especially non-governmental
organizations with the capability to effect innovation through seed funding,
could provide support to nucleate the creation of research software groups. As
noted above, Sloan has already initiated one such program, and collaboration
with Sloan or at least study of their methods and success or failure could be
extremely useful in approaching universities and other institutions in funding
this development track. It seems clear that if the value proposition can be made
to university administrators, this track could flourish with buy-in at the
administrative level.

