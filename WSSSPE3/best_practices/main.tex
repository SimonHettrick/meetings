\documentclass[11pt, oneside]{amsart}
\pdfoutput=1

\usepackage{amsmath}
\usepackage{amssymb}

\usepackage{color}
\usepackage{dcolumn}
\usepackage{float}
\usepackage{graphicx}
\usepackage[utf8]{inputenc}
\usepackage[T1]{fontenc}
\usepackage{lmodern}
\usepackage{multirow}
\usepackage{rotating}
\usepackage{subfigure}
\usepackage{psfrag}
\usepackage{tabularx}
\usepackage[hyphens]{url}
\usepackage{wrapfig}
\usepackage{longtable}
\usepackage{verbatim}
\usepackage{booktabs,multicol}

% The following three lines are used for displaying footnote in tables.
\usepackage{footnote}
\makesavenoteenv{tabular}
\makesavenoteenv{table}


\usepackage{enumitem}
\setlist{leftmargin=7mm}

%\setcounter{secnumdepth}{3}
%\setcounter{tocdepth}{3}


\usepackage[bookmarks, bookmarksopen, bookmarksnumbered]{hyperref}
\usepackage[all]{hypcap}
\urlstyle{rm}

\definecolor{orange}{rgb}{1.0,0.3,0.0}
\definecolor{violet}{rgb}{0.75,0,1}
\definecolor{darkgreen}{rgb}{0,0.6,0}
\definecolor{cyan}{rgb}{0.2,0.7,0.7}
\definecolor{blueish}{rgb}{0.2,0.2,0.8}
\definecolor{darkblue}{rgb}{0.1,0.1,0.9}

\newcommand{\todo}[1]{{\color{blue}$\blacksquare$~\textsf{[TODO: #1]}}}
\newcommand{\note}[1]{ {\textcolor{blueish}    { ***Note:      #1 }}}


% Don't use tt font for urls
\urlstyle{rm}

% 15 characters / 2.5 cm => 100 characters / line
% Using 11 pt => 94 characters / line
\setlength{\paperwidth}{216 mm}
% 6 lines / 2.5 cm => 55 lines / page
% Using 11pt => 48 lines / pages
\setlength{\paperheight}{279 mm}
\usepackage[top=2.5cm, bottom=2.5cm, left=2.5cm, right=2.5cm]{geometry}
% You can use a baselinestretch of down to 0.9
\renewcommand{\baselinestretch}{0.96}

\sloppypar

\begin{document}

\title[]{Best Practices for Developing Sustainable Scientific Software}

\author{Sandra Gesing$^{(1)}$,
Abani Patra$^{(2)}$,
Daniel S.\ Katz$^{(3)}$,
}

%
\thanks{{}$^{(1)}$ Center for Research Computing, University of Notre Dame, Notre Dame, IN, USA; sandra.gesing@nd.edu}
%
\thanks{{}$^{(2)}$ Mechanical and Aerospace Engineering, University at Buffalo, Buffalo, NY, USA; abani@buffalo.edu}
%
\thanks{{}$^{(3)}$ \hspace{-1ex}Computation Institute, 
University of Chicago \& Argonne National Laboratory, Chicago, IL, USA; d.katz@ieee.org}
%
 
\begin{abstract}


\end{abstract}

\maketitle
\newpage

%%%%%%%%%%%%%%%%%%%%%%%%%%%%%%%%%%%%%%%%%%%%%%%%%%%%%%%%%%%%
\section{Introduction} \label{sec:intro}
%%%%%%%%%%%%%%%%%%%%%%%%%%%%%%%%%%%%%%%%%%%%%%%%%%%%%%%%%%%%

\section{Related Work}
\label{sec:rel}


\section{Case Studies}
\label{sec:case-studies}


\section{Community-related Practices}
\label{sec:comm}

\section{Government and Management}
\label{sec:govern}

\section{Funding}
\label{sec:funding}

\section{Metrics}
\label{sec:metrics}

\section{Tools}
\label{sec:tools}


%%%%%%%%%%%%%%%%%%%%%%%%%%%%%%%%%%%%%%%%%%%%%%%%%%%%%%%%%%%%
\section{Conclusion} \label{sec:conclusion}
%%%%%%%%%%%%%%%%%%%%%%%%%%%%%%%%%%%%%%%%%%%%%%%%%%%%%%%%%%%%

%%%%%%%%%%%%%%%%%%%%%%%%%%%%%%%%%%%%%%%%%%%%%%%%%%%%%%%%%%%%
\section*{Acknowledgments} \label{sec:acks}
%%%%%%%%%%%%%%%%%%%%%%%%%%%%%%%%%%%%%%%%%%%%%%%%%%%%%%%%%%%%



\bibliographystyle{vancouver}

\bibliography{wssspe}
\end{document}
