\documentclass[11pt]{article}
\usepackage{geometry}                % See geometry.pdf to learn the layout options. There are lots.
\geometry{letterpaper}                   % ... or a4paper or a5paper or ... 
%\geometry{landscape}                % Activate for for rotated page geometry
%\usepackage[parfill]{parskip}    % Activate to begin paragraphs with an empty line rather than an indent
\usepackage{graphicx}
\usepackage{amssymb}
\usepackage{epstopdf}
\usepackage{enumitem}

 \geometry{
 letterpaper,
 total={215.9mm,279.4mm},
 left=20mm,
 right=20mm,
 top=20mm,
 bottom=20mm,
 bindingoffset=0mm
 }

\DeclareGraphicsRule{.tif}{png}{.png}{`convert #1 `dirname #1`/`basename #1 .tif`.png}

\title{The 4th Workshop on Sustainable Software: Best Practices and Experiences (WSSSPE4)
}
\author{Gabrielle Allen, University of Illinois Urbana--Champaign \\
	Daniel S.\ Katz, University of Illinois Urbana--Champaign \\
	Kyle E.\ Niemeyer, Oregon State University
	}
\date{}                                           % Activate to display a given date or no date

\pagestyle{empty}  

\parskip .5cm
\parindent 0cm

\begin{document}
\maketitle
%\section{}
%\subsection{}


{\bf \noindent Summary:} The WSSSPE4 workshop is part of a community driven effort to address the many new challenges that are evident for the development, deployment, maintenance, and sustainabiity of reusable software for academic research and education. These challenges include the software development process, the support and maintenance of software, governance and business models particularly for sustainability, the role of software in building science communities, the need for software to lead to science which is verifiable and reproducible, policy issues such as how to measure software impact, choose licensing, or assign software credit, and the education and nurturing of early stage researchers in scientific software. 
Based on successful WSSSPE1-WSSSPE3 meetings at SC13, SC14, and NCAR in 2013, 2014, and 2015, the organizers expect over 60 submitted papers and around 75 attendees at the WSSSPE4 meeting that will be held immediately preceding the RSE conference in Manchester, UK, in September 2016.  As with previous events, WSSSPE4 will provide an open and dynamic forum for the community to discuss experiences and approaches in the areas listed above. This proposal requests funds to support attendance at WSSSPE4 for US based researchers, particularly targeted at encouraging attendance by early stage researchers with attention to diversity. 

{\bf \noindent Intellectual Merit:} The participants will be selected using a formal application procedure, with selection criteria based around (1) the potential of their attendance to contribute towards their own research and educational goals; (2) their potential contribution to the workshop; (3) providing a diverse set of attendees with priority to early stage researchers (e.g. students, postdoctoral researchers, early stage faculty). The workshop will include presentations, panels and discussions driven from short peer reviewed papers that will advance understanding in this new area of attention. 

{\bf \noindent Broader Impacts:} All material from the workshop will be openly archived on public popular sites (e.g. Figshare, Slideshare, arXiv), it is anticipated that the workshop will again be tweeted and documented live including contributions from external participants. It is expected that the workshop will lead to a special issue of an open journal. The organizers take seriously the responsibility to encourage participation from underrepresented groups as workshop leaders as well as attendees, and NSF funds will be used to encourage such participation from the US community.  


\newpage

\section{Goals for WSSSPE4}

The first WSSSPE workshop (see Appendix for a description of WSSSPE 1) was held as part of Supercomputing 2013 in Denver during November 2013. The aim on the workshop was to bring together researchers and practitioners to discuss challenges and issues around the development, deployment and maintenance of reusable software for academic research and education. The next workshop, WSSSPE 2, will continue these discussions, focusing on some of the themes identified during WSSSPE~1. WSSSPE~2 is anticipated to be held as a workshop as part of the International IEEE Supercomputing 2014 conference in New Orleans in November 2014. A proposal has been submitted to SC14 which will be reviewed in the next month. 

The goals of WSSSPE 2 are
\begin{itemize}
\item Provide an effective mechanism to further discussions and identify research topics in the development, deployment and maintenance of reusable software
\item Proactively encourage the inclusion of issues related to education and involvement of industry
\item Develop a diverse community of engaged researchers, software developers, industrial representatives, students and educators
\item Make the event as accessible and inclusive as possible, leveraging social media and other technologies
\item Motivate the community in providing outputs including publications and reports
\end{itemize}

\section{Workshop Organization and Planning}

The WSSSPE~2 organizing committee comprises 
\begin{itemize}
\item Co-chairs: Gabrielle Allen (Professor, University of Illinois), Daniel S. Katz (Researcher, University of Chicago)
\item From WSSSPE~1 organizing committee: Manish Parashar (Professor, Rutgers University), Neil Chue Hong (Director, University of Edinburgh), David Proctor (Currently AAAS Fellow, NSF). 
\item New organizing members for WSSSPE 2: Colin Venters (Researcher, University of Leads), Nancy Wilkins-Diehr (Researcher, UCSD), and  Karen Cranston (National Evolutionary Synthesis Center)
\end{itemize}

The organizing committee will in the next months be discussing changes to the program based on last year�s experience, but it is anticipated that the one-day workshop will include
\begin{itemize}
\item Two invited keynote speakers
\item Peer reviewed short papers that will form the basis for discussions or panels
\item Targeted panel discussions
\item Breakout groups to discuss on feed back on identified issues
\end{itemize}

The WSSSPE 1 workshop used some novel features that are described in the arXiv report (http://arxiv.org/format/1311.3523v2 and Appendix) and will probably be repeated for WSSSPE 2. These included
\begin{itemize}
\item Authors submitted papers by placing them on an open archive of their choice and submitting the URL for the document
\item Instead of individual author presentations, the accepted papers where grouped into three panels, with panelists presenting the findings from a set of papers followed by a general discussion with all participants.
\end{itemize}

The tentative agenda for the Workshop is

\begin{tabular}{|l|l|} \hline
9.00 --- 9.15 & Introduction, review of the day's activities \\ \hline
9.15 --- 10.15 & Keynotes \\ \hline
10.15 --- 10.30 & Break \\ \hline
10.30 --- 11.15 & Panel 1 and Audience Discussion \\ \hline
11.15 --- 12.00 & Panel 2 and Audience Discussion \\ \hline
12.00 --- 1.00 & Lunch \\ \hline
1.00 --- 2.30 & Led breakout sessions \\ \hline
2.30 --- 3.30 & Summaries and discussion from breakout discussions \\ \hline
3.30 --- 4.30 & Panel 3 and Audience Discussion \\ \hline
4.30 --- 5.00 & Organize follow on activities  \\ \hline
\end{tabular}


\section{Participant Support}

We are requesting funding from NSF to be able to encourage attendance at the meeting from US participants particularly by underrepresented groups, early stage researchers (students, postdocs, beginning faculty), and to assist in recruiting a diverse set of panelists and keynote speakers. 

We are seeking funding early enough to be able to include a description of the opportunity for travel support in our call for papers and outreach material for the event to make sure that lack of funding is not a barrier to submitting a paper, and to make it clear that participation by early stage researchers is welcomed and encouraged. 

The participants will be selected using a formal application procedure which will be described and advertised on the workshop web page, with selection criteria based around (1) the potential of their attendance to contribute towards their own research and educational goals; (2) their potential contribution to the workshop; (3) providing a diverse set of attendees with priority to early stage researchers (e.g. students, postdoctoral researchers, early stage faculty).

The WSSSPE 2 organizing committee will review the applications and select the participants to be funded through the award. 

\section{Award Administration}

The National Center for Supercomputing Applications (NCSA) at the University of Illinois (U of I) will be the lead fiscal institution for this proposal to the Office of Advanced Cyberinfrastructure (ACI) at the National Science Foundation. NCSA staff will manage the travel for the participants and presenters funded through this proposal. US participants and presenters will arrange their flights through the travel agency of NCSA/University of Illinois, with reimbursed by NCSA/University of Illinois. NCSA/ University of Illinois will cover the hotel and meals for all US and European participants and presenters. The WSSSPE~2 organizing committee will review the applications and select the participants to be funded through this award. 

\section{Broader Impacts} 

The focus of the WSSSPE~2 workshop on sustainable software for science and engineering is a topic of very broad relevance and interest that cuts across many different disciplines and different categories of people from students, developers, faculty, etc. Areas of interest and focus in the workshop include education and career paths of those involved in the development of software.

This proposal is focused primarily on encouraging diverse participation at the workshop, particularly from underrepresented groups and early stage researchers. Funds will be used to help facilitate a diverse set of participants, including panel and keynote speakers. 

\newpage

\section*{Appendix}

The following is a report on the WSSSPE~1 workshop that is available at the arXiv (http://arxiv.org/abs/1311.3523)


\newpage

{\bf UNIVERSITY OF ILLINOIS AT URBANA-CHAMPAIGN}

{\bf BUDGET JUSTIFICATION}

{\bf WSSSPE2}

{\bf Principal Investigator: Gabrielle Allen}

\begin{enumerate}[label=\bf \Alph*.,leftmargin=*]

\item {\bf Senior Personnel: } No funds requested.

\item {\bf Other Personnel: } No funds requested.
\item {\bf Fringe Benefits: }  No funds requested. 
\item {\bf Equipment: }  No funds requested. 
\item {\bf Travel: } \$3,000

{\it National travel for the US based organizers to attend the workshop in New Orleans. Funds will be used to contribute towards airfares and/or the two hotel nights needed for the workshop}

\item {\bf Participant Support: } \$20,000
{\it Funds are requested to support travel for 18 participants and 2 presenters from US institutions across the country to travel to New Orleans. Estimated costs include airfare, 2 nights hotel and workshop fees.}

\item {\bf Other Direct Costs: } No funds requested. 
\item {\bf Indirect Costs: } \$1,755

{\it Indirect costs are assessed at a rate of 58.5\% of Modified Total Direct Costs (MTDC). MTDC is direct costs less equipment, tuition remission, and subawards in excess of \$25,000. }

\item {\bf Total Costs: } \$24,755 

{\it Total funds requested by this proposal. }


\end{enumerate}

\newpage

{\bf Data Management Plan}

As for WSSSPE1, all material (including publications, presentations, meeting minutes, real time discussions) will be open, hosted and archived on existing public platforms (e.g. Slideshare, ArXiv, Google Docs, Twitter). The WSSSPE web page and wiki will be additionally used.

Reviews of paper submissions will not be open. 




\end{document}
