\documentclass[11pt]{article}
\usepackage{geometry}                % See geometry.pdf to learn the layout options. There are lots.
\geometry{letterpaper}                   % ... or a4paper or a5paper or ... 
%\geometry{landscape}                % Activate for for rotated page geometry
%\usepackage[parfill]{parskip}    % Activate to begin paragraphs with an empty line rather than an indent
\usepackage{graphicx}
\usepackage{amssymb}
\usepackage{epstopdf}
\usepackage{enumitem}

 \geometry{
 letterpaper,
 total={215.9mm,279.4mm},
 left=20mm,
 right=20mm,
 top=20mm,
 bottom=20mm,
 bindingoffset=0mm
 }

\DeclareGraphicsRule{.tif}{png}{.png}{`convert #1 `dirname #1`/`basename #1 .tif`.png}

\title{The 4th Workshop on Sustainable Software: Best Practices and Experiences (WSSSPE4)
}
\author{Gabrielle Allen, University of Illinois Urbana--Champaign \\
	Daniel S.\ Katz, University of Illinois Urbana--Champaign \\
	Kyle E.\ Niemeyer, Oregon State University
	}
\date{}                                           % Activate to display a given date or no date

\pagestyle{empty}  

\parskip .5cm
\parindent 0cm

\begin{document}
\maketitle
%\section{}
%\subsection{}


{\bf \noindent Summary:} The WSSSPE4 workshop is part of a community driven effort to address the many new challenges that are evident for the development, deployment, maintenance, and sustainabiity of reusable software for academic research and education. These challenges include the software development process, the support and maintenance of software, governance and business models particularly for sustainability, the role of software in building science communities, the need for software to lead to science which is verifiable and reproducible, policy issues such as how to measure software impact, choose licensing, or assign software credit, and the education and nurturing of early stage researchers in scientific software. 
Based on successful WSSSPE1-WSSSPE3 meetings at SC13, SC14, and NCAR in 2013, 2014, and 2015, the organizers expect over 60 submitted papers and around 75 attendees at the WSSSPE4 meeting that will be held immediately preceding the RSE conference in Manchester, UK, in September 2016.  As with previous events, WSSSPE4 will provide an open and dynamic forum for the community to discuss experiences and approaches in the areas listed above. This proposal requests funds to support attendance at WSSSPE4 for US based researchers, particularly targeted at encouraging attendance by early stage researchers with attention to diversity. 

{\bf \noindent Intellectual Merit:} The participants will be selected using a formal application procedure, with selection criteria based around (1) the potential of their attendance to contribute towards their own research and educational goals; (2) their potential contribution to the workshop; (3) providing a diverse set of attendees with priority to early stage researchers (e.g. students, postdoctoral researchers, early stage faculty). The workshop will include vision and position papers as well as lightning talks and demos, aimed at improving sustainable scientific software today as well as defining and working towards the its future. 

{\bf \noindent Broader Impacts:} WSSSPE4 accepted submissions (except presentations of previously published work) are planned to be published by WSSSPE as a special collection in http://ceur-ws.org/, which is fully indexed.
Other material from the workshop will be openly archived on public popular sites (e.g. Figshare, Slideshare, arXiv). It is anticipated that the workshop will again be tweeted and documented live including contributions from external participants. The organizers take seriously the responsibility to encourage participation from underrepresented groups as workshop leaders as well as attendees, and NSF funds will be used to encourage such participation from the US community.  


\newpage

\section{Goals for WSSSPE4}

Progress in scientific research is dependent on the quality and accessibility of research software at all levels. It is now critical to address many new challenges related to the development, deployment, maintenance, and sustainability of open-use research software: the software upon which specific research results rely.  Open-use software means that the software is widely accessible (whether open source, shareware, or commercial).  Research software means that the choice of software is essential to specific research results; using different software could produce different results.
In addition, it is essential that scientists, researchers, and students are able to learn and adopt a new set of software-related skills and methodologies. Established researchers are already acquiring some of these skills, and in particular, a specialized class of software developers is emerging in academic environments who are an integral and embedded part of successful research teams. WSSSPE provides a forum for discussion of these challenges, including both positions and experiences, and a forum for the community to assemble and act.

This is the fourth major event in the WSSSPE series. The first two WSSSPE workshops were held as part of SC in 2013 and 2014. These one-day workshops brought together researchers and practitioners to discuss challenges and issues around the development, deployment and maintenance of reusable software for academic research and education, and discussed potential solutions, moving from a presentation to a presentation and discussion format.  WSSSPE3, held at NCAR in 2015, grew out of the previous events, and focused almost entirely on facilitated discussions, with the aim of starting working groups to make changes in the practice of scientific software development leading to more sustainable software.  Two of these groups have continued
working together, leading to the FORCE11 Software Citation Principles and an in-progress paper
on best practices for sustainable scientific software development.

In WSSSPE4, held over 2 1/2 days in Manchester, UK, we will include elements from the
previous workshops, with most of the workshop divided into two tracks:

{\bf Track 1 � Building a sustainable future for open-use research software} has the goals of defining a vision of the future of open-use research software, and in the workshop, initiating the activities that are needed to get there.  The idea of this track is to first think about where we want to be 5 to 10 years from now, without being too concerned with where we are today, and then to determine how we can move to this future.

{\bf Track 2 � Practices \& experiences in sustainable scientific software} has the goal of improving the quality of today�s research software and the experiences of its developers by sharing practices and experiences.  This track is focused on the current state of scientific software and what we can do to improve it in the short term, starting with where we are today.

Topics of interest include but are not limited to:
\begin{itemize}
\item Development and Community
\begin{itemize}
\item Best practices for developing sustainable software
\item Models for funding specialist expertise in software collaborations
\item Software tools that aid sustainability
\item Academia/industry interaction
\item Refactoring/improving legacy scientific software
\item Engineering design for sustainable software
\item Metrics for the success of scientific software
\item Adaptation of mainstream software practices for scientific software
\end{itemize}
\item Professionalization
\begin{itemize}
\item Career paths
\item RSE as a brand
\item RSE outside of the UK or Europe
\item Increase incentives in publishing, funding and promotion for better software
\end{itemize}
\item Training
\begin{itemize}
\item Training for developing sustainable software
\item Curriculum for software sustainability
\end{itemize}
\item Credit
\begin{itemize}
\item Making the existing credit and citation ecosystem work better for software
\item Future credit and citation ecosystem
\item Software contributions as a part of tenure review
\item Case studies of receiving credit for software contributions
\item Awards and recognition that encourage sustainable software
\end{itemize}
\item Software publishing
\begin{itemize}
\item Journals and alternative venues for publishing software
\item Review processes for published software
\end{itemize}
\item Software discoverability/reuse
\begin{itemize}
\item Proposals and case studies
\end{itemize}
\item Reproducibility and testing
\begin{itemize}
\item Reproducibility in conferences and journals
\item Best practices for code testing and code review
\end{itemize}
\end{itemize}


Joint sessions of the workshop will include a keynote, a set of lightning talks on topics related to either track, and a panel on best practices for scientific software development.  Tracks 1 and 2 will be held in parallel with each other.

Track 1 will consist of facilitated discussions, both as a single group and as breakouts.  It will also include discussion of Idea papers that presents implementable proposals related to the track.
Track 1 participants will build on the proposals and ideas in these papers, with the goal of initiating the planning, development, and execution of some of the ideas during the workshop itself. Given the magnitude and importance of the task at hand, the WSSSPE4 organizing committee encourages these proposals to be developed on an open, public, and inclusive basis. Submitters are invited to present a vision of some aspect of the future of open-use research software, and a plan of activities to gather and organize the resources needed to get there.

\begin{quote}
Example idea paper topics include:
\begin{itemize}
\item Adaptation of industrial software engineering principles into the research software community with a plan to fund the work
\item Funding and scaling software carpentry style training in advanced topics
Infrastructure and funding for community maintenance of open use research software
\item Scaling the SSI beyond the UK
\item Specific proposals of how to bridging/network the various research software engineering communities in scalable manner without destroying independence and unique foci of each community
\end{itemize}
\end{quote}

Track 2 will consist mostly of presentations, with some discussions, including position papers, experience papers, presentations of previously published work, and demos.  The intent of this
track is for current scientific software developers and those interested in scientific software development to learn from each other improve the current state of the practice in this area.


\section{Workshop Organization and Planning}

The WSSSPE~2 organizing committee comprises 
\begin{itemize}
\item Co-chairs: Gabrielle Allen (Professor, University of Illinois), Daniel S. Katz (Researcher, University of Chicago)
\item From WSSSPE~1 organizing committee: Manish Parashar (Professor, Rutgers University), Neil Chue Hong (Director, University of Edinburgh), David Proctor (Currently AAAS Fellow, NSF). 
\item New organizing members for WSSSPE 2: Colin Venters (Researcher, University of Leads), Nancy Wilkins-Diehr (Researcher, UCSD), and  Karen Cranston (National Evolutionary Synthesis Center)
\end{itemize}

The organizing committee will in the next months be discussing changes to the program based on last year�s experience, but it is anticipated that the one-day workshop will include
\begin{itemize}
\item Two invited keynote speakers
\item Peer reviewed short papers that will form the basis for discussions or panels
\item Targeted panel discussions
\item Breakout groups to discuss on feed back on identified issues
\end{itemize}

The WSSSPE 1 workshop used some novel features that are described in the arXiv report (http://arxiv.org/format/1311.3523v2 and Appendix) and will probably be repeated for WSSSPE 2. These included
\begin{itemize}
\item Authors submitted papers by placing them on an open archive of their choice and submitting the URL for the document
\item Instead of individual author presentations, the accepted papers where grouped into three panels, with panelists presenting the findings from a set of papers followed by a general discussion with all participants.
\end{itemize}

The tentative agenda for the Workshop is

\begin{tabular}{|l|l|} \hline
9.00 --- 9.15 & Introduction, review of the day's activities \\ \hline
9.15 --- 10.15 & Keynotes \\ \hline
10.15 --- 10.30 & Break \\ \hline
10.30 --- 11.15 & Panel 1 and Audience Discussion \\ \hline
11.15 --- 12.00 & Panel 2 and Audience Discussion \\ \hline
12.00 --- 1.00 & Lunch \\ \hline
1.00 --- 2.30 & Led breakout sessions \\ \hline
2.30 --- 3.30 & Summaries and discussion from breakout discussions \\ \hline
3.30 --- 4.30 & Panel 3 and Audience Discussion \\ \hline
4.30 --- 5.00 & Organize follow on activities  \\ \hline
\end{tabular}


\section{Participant Support}

We are requesting funding from NSF to be able to encourage attendance at the meeting from US participants particularly by underrepresented groups, early stage researchers (students, postdocs, beginning faculty), and to assist in recruiting a diverse set of panelists and keynote speakers. 

We are seeking funding early enough to be able to include a description of the opportunity for travel support in our call for papers and outreach material for the event to make sure that lack of funding is not a barrier to submitting a paper, and to make it clear that participation by early stage researchers is welcomed and encouraged. 

The participants will be selected using a formal application procedure which will be described and advertised on the workshop web page, with selection criteria based around (1) the potential of their attendance to contribute towards their own research and educational goals; (2) their potential contribution to the workshop; (3) providing a diverse set of attendees with priority to early stage researchers (e.g. students, postdoctoral researchers, early stage faculty).

The WSSSPE 2 organizing committee will review the applications and select the participants to be funded through the award. 

\section{Award Administration}

The National Center for Supercomputing Applications (NCSA) at the University of Illinois (U of I) will be the lead fiscal institution for this proposal to the Office of Advanced Cyberinfrastructure (ACI) at the National Science Foundation. NCSA staff will manage the travel for the participants and presenters funded through this proposal. US participants and presenters will arrange their flights through the travel agency of NCSA/University of Illinois, with reimbursed by NCSA/University of Illinois. NCSA/ University of Illinois will cover the hotel and meals for all US and European participants and presenters. The WSSSPE~2 organizing committee will review the applications and select the participants to be funded through this award. 

\section{Broader Impacts} 

The focus of the WSSSPE~2 workshop on sustainable software for science and engineering is a topic of very broad relevance and interest that cuts across many different disciplines and different categories of people from students, developers, faculty, etc. Areas of interest and focus in the workshop include education and career paths of those involved in the development of software.

This proposal is focused primarily on encouraging diverse participation at the workshop, particularly from underrepresented groups and early stage researchers. Funds will be used to help facilitate a diverse set of participants, including panel and keynote speakers. 

\newpage

\section*{Appendix}

The following is a report on the WSSSPE~1 workshop that is available at the arXiv (http://arxiv.org/abs/1311.3523)


\newpage

{\bf UNIVERSITY OF ILLINOIS AT URBANA-CHAMPAIGN}

{\bf BUDGET JUSTIFICATION}

{\bf WSSSPE2}

{\bf Principal Investigator: Gabrielle Allen}

\begin{enumerate}[label=\bf \Alph*.,leftmargin=*]

\item {\bf Senior Personnel: } No funds requested.

\item {\bf Other Personnel: } No funds requested.
\item {\bf Fringe Benefits: }  No funds requested. 
\item {\bf Equipment: }  No funds requested. 
\item {\bf Travel: } \$3,000

{\it National travel for the US based organizers to attend the workshop in New Orleans. Funds will be used to contribute towards airfares and/or the two hotel nights needed for the workshop}

\item {\bf Participant Support: } \$20,000
{\it Funds are requested to support travel for 18 participants and 2 presenters from US institutions across the country to travel to New Orleans. Estimated costs include airfare, 2 nights hotel and workshop fees.}

\item {\bf Other Direct Costs: } No funds requested. 
\item {\bf Indirect Costs: } \$1,755

{\it Indirect costs are assessed at a rate of 58.5\% of Modified Total Direct Costs (MTDC). MTDC is direct costs less equipment, tuition remission, and subawards in excess of \$25,000. }

\item {\bf Total Costs: } \$24,755 

{\it Total funds requested by this proposal. }


\end{enumerate}

\newpage

{\bf Data Management Plan}

As for WSSSPE1, all material (including publications, presentations, meeting minutes, real time discussions) will be open, hosted and archived on existing public platforms (e.g. Slideshare, ArXiv, Google Docs, Twitter). The WSSSPE web page and wiki will be additionally used.

Reviews of paper submissions will not be open. 




\end{document}
